\documentclass{beamer}
\title{Normalization for the simply typed \texorpdfstring{$\lambda$}{lambda}-calculus \\ \large A trilogy}
\author{Bálint Kocsis}
\institute{Radboud University}
\date{\today}

\usetheme{Boadilla}

%%%%%%%%%%%%%%%%%%%%%%%%%%%%%%%%%%%%%%%%%%%%%%%%
% Packages
%%%%%%%%%%%%%%%%%%%%%%%%%%%%%%%%%%%%%%%%%%%%%%%%
\usepackage[utf8]{inputenc}
\usepackage[english]{babel}
\usepackage{comment}
\usepackage{amssymb}
\usepackage{amsthm}
\usepackage{amsmath}
\usepackage{fdsymbol}
\usepackage{mathrsfs}
\usepackage{quiver}
\usepackage{mathpartir}
\usepackage{xparse}
\usepackage{hyperref}

%%%%%%%%%%%%%%%%%%%%%%%%%%%%%%%%%%%%%%%%%%%%%%%%
% Global settings
%%%%%%%%%%%%%%%%%%%%%%%%%%%%%%%%%%%%%%%%%%%%%%%%
\parindent = 0.7\parindent
\parskip = 0pt

\tikzcdset{row sep/normal = 2.7em, column sep/normal = 2.7em}

%%%%%%%%%%%%%%%%%%%%%%%%%%%%%%%%%%%%%%%%%%%%%%%%
% Mathematical environments
%%%%%%%%%%%%%%%%%%%%%%%%%%%%%%%%%%%%%%%%%%%%%%%%
\theoremstyle{definition}
\newtheorem{defn}{Definition}[section]
\newtheorem{ex}[defn]{Example}
\newtheorem{notn}[defn]{Notation}
\newtheorem{con}[defn]{Construction}

\theoremstyle{plain}
\newtheorem{thm}[defn]{Theorem}
\newtheorem{prop}[defn]{Proposition}
\newtheorem{lem}[defn]{Lemma}
\newtheorem{cor}[defn]{Corollary}

\theoremstyle{remark}
\newtheorem{rem}[defn]{Remark}

%%%%%%%%%%%%%%%%%%%%%%%%%%%%%%%%%%%%%%%%%%%%%%%%
% Formatting
%%%%%%%%%%%%%%%%%%%%%%%%%%%%%%%%%%%%%%%%%%%%%%%%
\newcommand{\name}{\mathrm}
\renewcommand{\emph}{\textbf}
\newcommand{\sep}{\ |\ }

%%%%%%%%%%%%%%%%%%%%%%%%%%%%%%%%%%%%%%%%%%%%%%%%
% Abbreviations
%%%%%%%%%%%%%%%%%%%%%%%%%%%%%%%%%%%%%%%%%%%%%%%%
\newcommand{\subs}{\subseteq}
\newcommand{\sups}{\supseteq}
\newcommand{\ul}{\underline}
\newcommand{\To}{\Rightarrow}
\newcommand{\xto}[2][]{\xrightarrow[#1]{#2}}
\newcommand{\xfrom}[2][]{\xleftarrow[#1]{#2}}
\renewcommand{\phi}{\varphi}
\renewcommand{\epsilon}{\varepsilon}

%%%%%%%%%%%%%%%%%%%%%%%%%%%%%%%%%%%%%%%%%%%%%%%%
% Brackets
%%%%%%%%%%%%%%%%%%%%%%%%%%%%%%%%%%%%%%%%%%%%%%%%
\newcommand{\ang}[1]{\langle #1 \rangle}
\newcommand{\sem}[1]{\lsem #1 \rsem}

%%%%%%%%%%%%%%%%%%%%%%%%%%%%%%%%%%%%%%%%%%%%%%%%
% Sets & logic
%%%%%%%%%%%%%%%%%%%%%%%%%%%%%%%%%%%%%%%%%%%%%%%%
\renewcommand{\implies}{\To}
\renewcommand{\iff}{\Leftrightarrow}

\newcommand{\nul}{\emptyset}
\newcommand{\singel}{*}
\newcommand{\singset}[1][\singel]{\{#1\}}
\newcommand{\N}{\mathbb{N}}
\newcommand{\setof}[2]{\left\{ #1 \sep #2 \right\}}
\newcommand{\cartprod}[2]{#1 \times #2}
\newcommand{\funcset}[2]{#2^{#1}}
\newcommand{\powset}[1]{\mathcal{P}(#1)}
\newcommand{\finpowset}[1]{\mathcal{P}_{\mathrm{fin}}(#1)}

\newcommand{\pto}{\rightharpoonup}
\newcommand{\dom}[1]{\mathrm{dom}(#1)}

%%%%%%%%%%%%%%%%%%%%%%%%%%%%%%%%%%%%%%%%%%%%%%%%
% Basic category theory
%%%%%%%%%%%%%%%%%%%%%%%%%%%%%%%%%%%%%%%%%%%%%%%%
\newcommand{\cat}[1]{\mathcal{#1}}
\newcommand{\Ob}[1]{\mathrm{Ob}(#1)}
\newcommand{\Hom}[3][\mathrm{Hom}]{#1(#2, #3)}
\NewDocumentCommand{\id}{o}{1\IfValueT{#1}{_{#1}}}
\NewDocumentCommand{\idfunc}{o}{\name{Id}\IfValueT{#1}{_{#1}}}
\NewDocumentCommand{\inv}{s m}{\IfBooleanTF{#1}{(#2)^{-1}}{#2^{-1}}}

\newcommand{\Set}{\mathbf{Set}}
\newcommand{\Cat}{\mathbf{Cat}}
\newcommand{\op}[1]{#1^\mathrm{op}}
\newcommand{\prodcat}[2]{#1 \times #2}
\newcommand{\funccat}[2]{[#1, #2]}
\newcommand{\PSh}[1]{\name{PSh}(#1)}

%%%%%%%%%%%%%%%%%%%%%%%%%%%%%%%%%%%%%%%%%%%%%%%%
% Lambda calculus with variable names
%%%%%%%%%%%%%%%%%%%%%%%%%%%%%%%%%%%%%%%%%%%%%%%%
\newcommand{\Basetypes}{\Sigma}
\newcommand{\functy}[2]{#1 \to #2}
\newcommand{\Ty}{\mathrm{Ty}}

\newcommand{\Var}[1][]{V_{#1}}
\newcommand{\app}[2]{#1 \cdot #2}
\newcommand{\lamv}[3]{\lambda #1^{#2}\hspace{-1pt}.\;#3}
\newcommand{\Tm}[1][]{\Lambda_{#1}}
\newcommand{\typedvar}[2]{#1 : #2}
\newcommand{\typedtm}[2]{#1 : #2}

\newcommand{\FV}[2][]{\mathrm{FV}_{#1}(#2)}
\newcommand{\BV}[1]{\mathrm{BV}(#1)}

\newcommand{\Sub}{\mathrm{Sub}}
\newcommand{\varsubst}[2]{#1 := #2}
\newcommand{\singsubv}[2]{[\varsubst{#1}{#2}]}
\NewDocumentCommand{\substnf}{m m m}
    {[\varsubst{#1_1}{#2_1}, \ldots, \varsubst{#1_{#3}}{#2_{#3}}]}
\newcommand{\updsub}[3]{#1[\varsubst{#2}{#3}]}
\newcommand{\idsub}[1][]{\mathrm{id}_{#1}}
\newcommand{\subst}[2]{#1 #2}

\newcommand{\convrel}[1]{{=}_{#1}}
\newcommand{\conv}[3]{\vdash #1 = #2 : #3}

\NewDocumentCommand{\Ne}{o}{\mathrm{Ne}\IfValueT{#1}{_{#1}}}
\NewDocumentCommand{\Nf}{o}{\mathrm{Nf}\IfValueT{#1}{_{#1}}}

\newcommand{\Con}{\mathrm{Con}}
\newcommand{\cTm}[2]{\Lambda_{#2}(#1)}
\newcommand{\cNf}[2]{\mathrm{Nf}_{#2}(#1)}
\newcommand{\cNe}[2]{\mathrm{Ne}_{#2}(#1)}
\newcommand{\ctypedtm}[3]{#1 \vdash #2 : #3}
\newcommand{\ctypedconv}[4]{#1 \vdash #2 = #3 : #4}

%%%%%%%%%%%%%%%%%%%%%%%%%%%%%%%%%%%%%%%%%%%%%%%%
% Henkin models
%%%%%%%%%%%%%%%%%%%%%%%%%%%%%%%%%%%%%%%%%%%%%%%%
\newcommand{\struct}[1]{\mathcal{#1}}
\newcommand{\scomp}[2]{#1_{#2}}
\NewDocumentCommand{\sapp}{O{} O{}}{\mathrm{app}^{#2}_{#1}}

\NewDocumentCommand{\Env}{O{} O{}}{\mathrm{Env}^{#2}_{#1}}
\newcommand{\typedenv}[2]{#1 \vDash #2}
\newcommand{\updenv}[3]{#1[#2 := #3]}
\newcommand{\empenv}{\nul}
\newcommand{\sint}[3][]{\sem{#2}^{#1}_{#3}}

\newcommand{\stdmod}[1]{\struct{S}(#1)}
\newcommand{\tmmod}{\struct{L}}
\newcommand{\sintsub}[2]{\sint{#1}{#2}}
\newcommand{\modsat}[4]{#1 \vDash #2 = #3 : #4}

\newcommand{\mapenv}[2]{#1 #2}

%%%%%%%%%%%%%%%%%%%%%%%%%%%%%%%%%%%%%%%%%%%%%%%%
% Kripke models
%%%%%%%%%%%%%%%%%%%%%%%%%%%%%%%%%%%%%%%%%%%%%%%%
\newcommand{\kscomp}[3]{#1_{#2}^{#3}}
\NewDocumentCommand{\kstran}{O{} O{}}{\mathrm{i}_{#1}^{#2}}
\NewDocumentCommand{\ksapp}{O{} O{}}{\mathrm{app}_{#1}^{#2}}

\newcommand{\ctypedenv}[3]{#1 \vDash #2 \sep #3}
\newcommand{\kupdenv}[3]{#1[#2 := #3]}
\newcommand{\ksint}[3]{\sem{#1}^{#3}_{#2}}

\newcommand{\kstdmod}[2]{\struct{S}(#1, #2)}
\newcommand{\cmodsat}[5]{#1 \vDash #2 \vdash #3 = #4 : #5}

%%%%%%%%%%%%%%%%%%%%%%%%%%%%%%%%%%%%%%%%%%%%%%%%
% NbE
%%%%%%%%%%%%%%%%%%%%%%%%%%%%%%%%%%%%%%%%%%%%%%%%
\newcommand{\normf}[2]{\name{nf}^{#1}_{#2}}
\newcommand{\fresh}[2]{v_{#1,#2}}
\newcommand{\extfresh}[2]{#1^{+#2}}
\newcommand{\unquotef}[2]{\name{u}^{#1}_{#2}}
\newcommand{\quotef}[2]{\name{q}^{#1}_{#2}}
\newcommand{\idenv}[1]{\eta_{#1}}

%%%%%%%%%%%%%%%%%%%%%%%%%%%%%%%%%%%%%%%%%%%%%%%%
% Categorical preliminaries
%%%%%%%%%%%%%%%%%%%%%%%%%%%%%%%%%%%%%%%%%%%%%%%%
\newcommand{\1}[1][]{\mathbf{1}_{#1}}
\newcommand{\0}[1][]{\mathbf{0}_{#1}}
\NewDocumentCommand{\mterm}{o}{{!}\IfValueT{#1}{_{#1}}}
\NewDocumentCommand{\minit}{o}{{!}\IfValueT{#1}{_{#1}}}
\newcommand{\cprod}[2]{#1 \times #2}
\NewDocumentCommand{\mfst}{o}{\name{fst}\IfValueT{#1}{_{#1}}}
\NewDocumentCommand{\msnd}{o}{\name{snd}\IfValueT{#1}{_{#1}}}
\newcommand{\mpair}[2]{\ang{#1, #2}}
\newcommand{\cexp}[2]{#1 \To #2}
\newcommand{\mev}[1][]{\name{ev}_{#1}}
\newcommand{\mcurry}[1]{\lambda(#1)}

\newcommand{\CCC}{\mathbf{CCC}}

\NewDocumentCommand{\prodcmp}{o o}{s\IfValueT{#1}{_{#1}}\IfValueT{#2}{^{#2}}}
\NewDocumentCommand{\expcmp}{o o}{t\IfValueT{#1}{_{#1}}\IfValueT{#2}{^{#2}}}

\newcommand{\y}{y}
\newcommand{\yap}[1]{\y_{#1}}
\newcommand{\reind}[1]{#1^*}

\newcommand{\comma}[2]{#1 \mathbin{\hspace{-1pt}\downarrow} #2}
\newcommand{\commafst}[2]{P_{#1, #2}}
\newcommand{\commasnd}[2]{Q_{#1, #2}}
\newcommand{\commatrd}[2]{\theta_{#1, #2}}

%%%%%%%%%%%%%%%%%%%%%%%%%%%%%%%%%%%%%%%%%%%%%%%%
% Well-scoped, well-typed lambda calculus
% with explicit substitutions
%%%%%%%%%%%%%%%%%%%%%%%%%%%%%%%%%%%%%%%%%%%%%%%%
\newcommand{\sTy}{\Ty}
\newcommand{\sCon}{\mathrm{Con}}
\newcommand{\sempcon}{{[]}}
\newcommand{\sextcon}[2]{#1, #2}

\newcommand{\sTm}[2]{\mathrm{Tm}(#1; #2)}
\newcommand{\sSub}[2]{\mathrm{Sub}(#1; #2)}
\newcommand{\stypedtm}[3]{#2 \vdash #1 : #3}
\newcommand{\stypedsub}[3]{#2 \vdash #1 : #3}

\NewDocumentCommand{\vz}{o o}{\mathrm{v}\IfValueT{#1}{^{#1}}\IfValueT{#2}{_{#2}}}
\newcommand{\slam}[2]{\lambda^{#1}.\;#2}
\newcommand{\ssubst}[2]{#1 #2}

\newcommand{\sempsub}[1][]{{()}_{#1}}
\newcommand{\sextsub}[2]{(#1, #2)}
\NewDocumentCommand{\sidsub}{o}{\name{id}\IfValueT{#1}{_{#1}}}
\newcommand{\scompsub}[2]{#1 \circ #2}
\NewDocumentCommand{\spsub}{o o}{\name{p}\IfValueT{#1}{^{#1}}\IfValueT{#2}{_{#2}}}

\newcommand{\sVar}[2]{\mathrm{Var}(#1; #2)}

\newcommand{\stmsub}[1]{#1^\bullet}
\newcommand{\ssingsub}[1]{\ang{#1}}
\NewDocumentCommand{\sliftsub}{o m}{#2\IfValueT{#1}{^{#1}}}

\newcommand{\sconvreltm}[2]{{=}_{#1,#2}^\mathrm{Tm}}
\newcommand{\sconvrelsub}[2]{{=}_{#1,#2}^\mathrm{Sub}}
\newcommand{\sconvtm}[4]{#3 \vdash #1 = #2 : #4}
\newcommand{\sconvsub}[4]{#3 \vdash #1 = #2 : #4}

\newcommand{\sNe}[2]{\mathrm{Ne}(#1; #2)}
\newcommand{\sNf}[2]{\mathrm{Nf}(#1; #2)}
\newcommand{\stypedvar}[3]{#2 \vdash^{\mathrm{v}} #1 : #3}
\newcommand{\stypedne}[3]{#2 \vdash^{\mathrm{ne}} #1 : #3}
\newcommand{\stypednf}[3]{#2 \vdash^{\mathrm{nf}} #1 : #3}

%%%%%%%%%%%%%%%%%%%%%%%%%%%%%%%%%%%%%%%%%%%%%%%%
% Scwfs
%%%%%%%%%%%%%%%%%%%%%%%%%%%%%%%%%%%%%%%%%%%%%%%%
\NewDocumentCommand{\SCon}{o}{\mathrm{Con}\IfValueT{#1}{^{#1}}}
\NewDocumentCommand{\STy}{o}{\mathrm{Ty}\IfValueT{#1}{^{#1}}}
\NewDocumentCommand{\Sempcon}{o}{\bullet\IfValueT{#1}{^{#1}}}
\NewDocumentCommand{\Sextcon}{m m o}{#1 \cdot\IfValueT{#3}{^{#3}} #2}

\NewDocumentCommand{\STm}{o m o}
    {\IfValueTF{#1}
        {\mathrm{Tm}\IfValueT{#3}{^{#3}}(#1; #2)}
        {\mathrm{Tm}\IfValueT{#3}{^{#3}}_{#2}}}
\NewDocumentCommand{\SSub}{m m o}
    {\mathrm{Sub}\IfValueT{#3}{^{#3}}(#1; #2)}

\NewDocumentCommand{\Sq}{o o}{\mathrm{q}\IfValueT{#1}{^{#1}}\IfValueT{#2}{_{#2}}}
\newcommand{\Ssubst}[2]{#1[#2]}

\NewDocumentCommand{\Sempsub}{o o}{\etaa\IfValueT{#1}{_{#1}}\IfValueT{#2}{^{#2}}}
    \newcommand{\etaa}{\eta}
\NewDocumentCommand{\Sextsub}{m m o}{#1 .\IfValueT{#3}{^{#3}} #2}
\NewDocumentCommand{\Sidsub}{o o}{\name{id}\IfValueT{#1}{_{#1}}\IfValueT{#2}{^{#2}}}
\NewDocumentCommand{\Scompsub}{m m o}{#1 \circ\IfValueT{#3}{^{#3}} #2}
\NewDocumentCommand{\Spsub}{o o}{\name{p}\IfValueT{#1}{^{#1}}\IfValueT{#2}{_{#2}}}

\newcommand{\Stmsub}[1]{#1^\bullet}
\newcommand{\Ssingsub}[1]{\ang{#1}}
\NewDocumentCommand{\Sliftsub}{o m}{#2\IfValueT{#1}{^{#1}}}

\NewDocumentCommand{\Sfuncty}{m m o}{#1 \To\IfValueT{#3}{^{#3}} #2}
\NewDocumentCommand{\Sapp}{o m m o}
    {#2 \mathbin{\$\IfValueT{#1}{_{#1}}\IfValueT{#4}{^{#4}}} #3}
\NewDocumentCommand{\Slam}{m o m}{\lambdaa^{#1}\IfValueT{#2}{_{#2}}(#3)}
    \newcommand{\lambdaa}{\lambda}

\NewDocumentCommand{\scwfmorcon}{s m}{#2\IfBooleanT{#1}{^\name{Con}}}
\NewDocumentCommand{\scwfmorsub}{s m m m}
    {#2\IfBooleanT{#1}{^\name{Sub}}_{#3,#4}}
\NewDocumentCommand{\scwfmorty}{s m}{#2\IfBooleanT{#1}{^\name{Ty}}}
\NewDocumentCommand{\scwfmortm}{s m o m}
        {#2\IfBooleanT{#1}{^\name{Tm}}_{\IfValueTF{#3}{#3,#4}{#4}}}

\newcommand{\Ldom}{\lambda\mathrm{-}\mathbf{Dom}}

%%%%%%%%%%%%%%%%%%%%%%%%%%%%%%%%%%%%%%%%%%%%%%%%
% Interpretation, abstract syntax
%%%%%%%%%%%%%%%%%%%%%%%%%%%%%%%%%%%%%%%%%%%%%%%%

\NewDocumentCommand{\cint}{o m o}{\sem{#2}\IfValueT{#1}{_{#1}}\IfValueT{#3}{^{#3}}}
\NewDocumentCommand{\cintty}{o m}{\cint[#1]{#2}[\mathrm{Ty}]}
\NewDocumentCommand{\cintcon}{o m}{\cint[#1]{#2}[\mathrm{Con}]}
\NewDocumentCommand{\cinttm}{o m}{\cint[#1]{#2}[\mathrm{Tm}]}
\NewDocumentCommand{\cintsub}{o m}{\cint[#1]{#2}[\mathrm{Sub}]}

\newcommand{\syncat}{\cat{L}}
\newcommand{\canint}{I}
\newcommand{\Mod}{\mathbf{Mod}}

%%%%%%%%%%%%%%%%%%%%%%%%%%%%%%%%%%%%%%%%%%%%%%%%
% Gluing
%%%%%%%%%%%%%%%%%%%%%%%%%%%%%%%%%%%%%%%%%%%%%%%%
\newcommand{\pCon}{\sCon}
\newcommand{\inclcon}{i}
\NewDocumentCommand{\FSub}{o}{\mathrm{Sub}\IfValueT{#1}{_{#1}}}
\newcommand{\PNe}[1]{\mathrm{Ne}_{#1}}
\newcommand{\PNf}[1]{\mathrm{Nf}_{#1}}
\newcommand{\gluecat}{\cat{G}}
\newcommand{\gluefst}{\pi_1}
\newcommand{\gluesnd}{\pi_2}
\NewDocumentCommand{\glueint}{o}{\rhoo\IfValueT{#1}{_{#1}}}
    \newcommand{\rhoo}{\rho}
\newcommand{\gluevar}[1]{\nu_{#1}}
\newcommand{\glueneut}[1]{\mu_{#1}}
\newcommand{\predneut}[1]{m_{#1}}
\newcommand{\gluenorm}[1]{\eta_{#1}}
\newcommand{\prednorm}[1]{n_{#1}}
\newcommand{\gluepsh}[1]{R_{#1}}
\newcommand{\gluepred}[1]{r_{#1}}
\newcommand{\gluemortm}[1]{\tilde{#1}}
\newcommand{\gluemorsub}[1]{\tilde{#1}}
\newcommand{\gquote}[1]{\mathrm{q}^{#1}}
\newcommand{\gunquote}[1]{\mathrm{u}^{#1}}
\newcommand{\nvar}[1]{\mathrm{var}_{#1}}
\newcommand{\napp}[2]{\mathrm{app}_{#1,#2}}
\newcommand{\nlam}[2]{\mathrm{lam}_{#1,#2}}
\newcommand{\gnorm}[2]{\name{nf}^{#1}_{#2}}

%%%%%%%%%%%%%%%%%%%%%%%%%%%%%%%%%%%%%%%%%%%%%%%%
% Uncategorized
%%%%%%%%%%%%%%%%%%%%%%%%%%%%%%%%%%%%%%%%%%%%%%%%
\newcommand{\convset}[1]{C(#1)}
\newcommand{\substruct}{\subs}
\newcommand{\upset}[1]{\uparrow\hspace{-2pt}(#1)}
\newcommand{\singlist}[1]{{[#1]}}
\newcommand{\concat}[2]{#1, #2}


\begin{document}

\maketitle

\begin{frame}{My thesis}
\begin{itemize}
    \item Present 3 normalization proofs
    \begin{itemize}
        \item Tait's method
        \item Normalization by evaluation (NbE)
        \item Categorical gluing
    \end{itemize}
    \item Compare the structure of the proofs
\end{itemize}
\end{frame}

\begin{frame}{This talk}
\begin{itemize}
    \item Quick intro to STLC
    \item Normalization
    \item Present a normalization proof based on Tait's method
    \item Compare the main steps of Tait's method and NbE
    \item Touch on the categorical proof
\end{itemize}
\end{frame}

\begin{frame}{Setting: the simply typed \texorpdfstring{$\lambda$}{lambda}-calculus}
\begin{defn}[Types]
    \[ \sigma, \tau := \beta \sep \functy{\sigma}{\tau} \]
where $\beta \in \Basetypes$ (\emph{base types})
\end{defn}
\begin{defn}[Terms]
$\Tm[\sigma]$ - \emph{terms of type $\sigma$}, defined inductively

We write $\typedtm{t}{\sigma}$ for $t \in \Tm[\sigma]$
\vspace{-12pt}
\begin{figure}
\begin{mathpar}
\inferrule[var]
    {x \in \Var[\sigma]}
    {\typedtm{x}{\sigma}}
\and
\inferrule[app]
    {\typedtm{t}{\functy{\sigma}{\tau}} \and \typedtm{u}{\sigma}}
    {\typedtm{\app{t}{u}}{\tau}}
\and
\inferrule[lam]
    {x \in \Var[\sigma] \and \typedtm{t}{\tau}}
    {\typedtm{\lamv{x}{\sigma}{t}}{\functy{\sigma}{\tau}}}
\end{mathpar}
\end{figure}
where $\Var[\sigma]$ is the set of \emph{variables of type $\sigma$}
\end{defn}
\end{frame}

\begin{comment}
\begin{frame}{Free and bound variables}
Intuition: bound = in the scope of a lambda abstraction; free = not bound
\begin{defn}[Free and bound variables]
The sets of \emph{free variables} $\FV{t}$ and \emph{bound variables} $\BV{t}$ of a term $t$ are defined recursively:
\begin{align*}
\FV{-} &: \Tm[\sigma] \to \finpowset{\Var} &
    \BV{-} &: \Tm[\sigma] \to \finpowset{\Var} \\
\FV{x} &= \singset[x] &
    \BV{x} &= \nul \\
\FV{\app{t}{u}} &= \FV{t} \cup \FV{u} &
    \BV{\app{t}{u}} &= \BV{t} \cup \BV{u} \\
\FV{\lamv{x}{\sigma}{t}} &= \FV{t} \setminus \singset[x] &
    \BV{\lamv{x}{\sigma}{t}} &= \BV{t} \cup \singset[x]
\end{align*}
where $V = \bigcup_{\sigma \in \Ty}{\Var[\sigma]}$
\end{defn}
\end{frame}

\begin{frame}{$\alpha$-conversion, variable convention}
\begin{defn}[$\alpha$-convertibility]
Two terms are \emph{$\alpha$-convertible} if they only differ in the names of bound variables.
\end{defn}
We identify $\alpha$-convertible terms, i.e. we take the quotient by the equivalence relation of $\alpha$-convertibility

When working with representatives of $\alpha$-equivalence classes, we employ Barendregt's \emph{variable convention}: all bound variables of all terms that occur in a certain mathematical context (e.g. definition, proof) are assumed to be distinct from all free variables of the terms
\end{frame}

\begin{frame}{Substitution}
Let $\Tm = \bigcup_{\sigma \in \Ty}{\Tm[\sigma]}$
\begin{defn}[Substitution]
A \emph{substitution} is a partial function $\gamma : \Var \pto \Tm$ with finite domain such that if $\typedvar{x}{\sigma}$ and $x \in \dom{\gamma}$, then $\typedtm{\gamma(x)}{\sigma}$.
\end{defn}
\begin{defn}[Substitution in terms]
We define an operation $\subst{(-)}{(-)} : \cartprod{\Tm[\sigma]}{\Sub} \to \Tm[\sigma]$ by recursion on the first argument:
\begin{align*}
\subst{x}{\gamma} &= \begin{cases}
                       \gamma(x) & \text{if $x \in \dom{\gamma}$} \\
                       x         & \text{otherwise}
                     \end{cases} \\
\subst{(\app{t}{u})}{\gamma} &= \app{\subst{t}{\gamma}}{\subst{u}{\gamma}} \\
\subst{(\lamv{x}{\sigma}{t})}{\gamma} &= \lamv{x}{\sigma}{\subst{t}{(\updsub{\gamma}{x}{x})}}
\end{align*}
\end{defn}
\end{frame}
\end{comment}

\begin{frame}{Conversion}
\begin{defn}[Conversion relation]
We inductively define a family $\convrel{\sigma} \subs \cartprod{\Tm[\sigma]}{\Tm[\sigma]}$ of relations, called the \emph{conversion relation}.

We write $\conv{t}{u}{\sigma}$ for $(t, u) \in \convrel{\sigma}$
\vspace{-12pt}
\begin{figure}
\begin{mathpar}
\inferrule[beta]
    {\typedtm{t}{\tau} \and \typedtm{u}{\sigma}}
    {\conv{\app{(\lamv{x}{\sigma}{t})}{u}}{\subst{t}{\singsubv{x}{u}}}{\tau}}
\\
\inferrule[eta]
    {\typedtm{t}{\functy{\sigma}{\tau}}}
    {\conv{\lamv{x}{\sigma}{\app{t}{x}}}{t}{\functy{\sigma}{\tau}}}
\quad(x \notin \FV{t})
\end{mathpar}
\end{figure}
+ equivalence relation and congruence rules
\end{defn}
\end{frame}

\begin{frame}{Extensionality}
The rule
\begin{mathpar}
\inferrule[eta]
    {\typedtm{t}{\functy{\sigma}{\tau}}}
    {\conv{\lamv{x}{\sigma}{\app{t}{x}}}{t}{\functy{\sigma}{\tau}}}
\quad(x \notin \FV{t})
\end{mathpar}
is equivalent to extensionality:
\begin{mathpar}
\inferrule[ext]
    {\conv{\app{t}{x}}{\app{t'}{x}}{\tau}}
    {\conv{t}{t'}{\functy{\sigma}{\tau}}}
\quad(x \notin \FV{t} \cup \FV{t'})
\end{mathpar}
\end{frame}

\begin{frame}{Traditional approach to normalization: term rewriting}
\begin{figure}
\begin{mathpar}
\inferrule[beta]{}
    {\app{(\lamv{x}{\sigma}{t})}{u} \to \subst{t}{\singsubv{x}{u}}}
\and
\inferrule[eta]{}
    {\lamv{x}{\sigma}{\app{t}{x}} \to t}
\quad(x \notin \FV{t})
\end{mathpar}
\begin{itemize}
    \item Normalization $\sim$ successive rewrite steps
    \item Normal form: no more rewrites are possible
    \item Problem with term rewriting: reductions are performed step-by-step $\to$ can be slow for terms with many redexes
\end{itemize}
\end{figure}
\end{frame}

\begin{frame}{What is reduction-free normalization?}
\begin{itemize}
    \item Use convertibility instead of reduction
    \begin{itemize}
        \item Convertibility is symmetric
    \end{itemize}
    \item Characterize normal forms differently
    \begin{itemize}
        \item Specify how to construct them
    \end{itemize}
\end{itemize}
\end{frame}

\begin{frame}{Normal forms, inductively}
We characterize long $\beta\eta$-normal forms
\begin{defn}[Normal forms, neutral terms]
We define two collections of terms, $\Ne[\sigma]$ and $\Nf[\sigma]$, by mutual induction:
\vspace{-12pt}
\begin{figure}
\begin{mathpar}
\inferrule[var-ne]
    {x \in \Var[\sigma]}
    {x \in \Ne[\sigma]}
\and
\inferrule[app-ne]
  {m \in \Ne[\functy{\sigma}{\tau}] \and n \in \Nf[\sigma]}
  {\app{m}{n} \in \Ne[\tau]}
\\
\inferrule[shift]
    {m \in \Ne[\beta]}
    {m \in \Nf[\beta]}
\quad(\beta \in \Basetypes)
\and
\inferrule[lam-nf]
    {n \in \Nf[\tau]}
    {\lamv{x}{\sigma}{n} \in \Nf[\functy{\sigma}{\tau}]}
\end{mathpar}
\end{figure}
\end{defn}
\end{frame}

\begin{frame}{Normalization proofs}
We look at three proofs:
\begin{itemize}
    \item Tait's method: weak normalization
    \item Normalization by evaluation (NbE)
    \item Categorical gluing
\end{itemize}
\end{frame}

\begin{frame}{Weak normalization}
\begin{thm}[Weak normalization]
For all terms $\typedtm{t}{\sigma}$, there exists an $n \in \Nf[\sigma]$ such that $\conv{t}{n}{\sigma}$.
\end{thm}
\begin{itemize}
    \item Naive proof by induction on $t$ fails (application case)
    \item We employ \textit{induction loading}: prove a stronger property by induction
\end{itemize}
\end{frame}

\begin{frame}{Steps of the proof}
$\mathrm{WN}_\sigma = \setof{\typedtm{t}{\sigma}}{\text{t is convertible to a normal form}}$
\begin{enumerate}
    \item Define $\sem{\sigma} \subs \Tm[\sigma]$: this is the stronger property (Tait: convertibility); also called a \emph{logical predicate}
    \item Show that $t \in \sem{\sigma}$ for all $\typedtm{t}{\sigma}$ (Fundamental Lemma)
    \item Show that $\sem{\sigma} \subs \mathrm{WN}_\sigma$
    \item Weak normalization follows from (2) and (3)
\end{enumerate}
\end{frame}

\begin{frame}{Step 1: Logical predicate}
\begin{defn}[Convertibility predicate]
We define $\sem{\sigma} \subs \Tm[\sigma]$ by induction on $\sigma$:
\begin{align*}
\sem{\beta} &= \mathrm{WN}_\beta \\
            &\quad\text{(the property we wish to show)} \\
\sem{\functy{\sigma}{\tau}} &= \setof{\typedtm{t}{\functy{\sigma}{\tau}}}{\forall \typedtm{u}{\sigma}.\;u \in \sem{\sigma} \implies \app{t}{u} \in \sem{\tau}} \\
            &\quad\text{(mapping convertible inputs to convertible outputs)}
\end{align*}
\end{defn}
\end{frame}

\begin{frame}{Step 2: Fundamental Lemma}
Substitution: type preserving finite mapping from variables to terms
\begin{lem}[Fundamental Lemma]
Let $\typedtm{t}{\sigma}$ and let $\gamma$ be a substitution. If $\gamma(x) \in \sem{\tau}$ for all $x \in \FV[\tau]{t}$, then $\subst{t}{\gamma} \in \sem{\sigma}$.
\end{lem}
\begin{proof}
By induction on $\typedtm{t}{\sigma}$.
\end{proof}
\end{frame}

\begin{frame}{Step 3}
\begin{itemize}
    \item Goal: prove $\sem{\sigma} \subs \mathrm{WN}_\sigma$
    \item To do so, we need another statement
\end{itemize}
$\mathrm{WA}_\sigma = \setof{\typedtm{t}{\sigma}}{t\text{ is convertible to a neutral element}}$
\begin{lem}
For all types $\sigma$, we have
\begin{enumerate}
    \item $\sem{\sigma} \subs \mathrm{WN}_\sigma$
    \item $\mathrm{WA}_\sigma \subs \sem{\sigma}$
\end{enumerate}
\end{lem}
\begin{proof}
Both statements are proved simultaneously by induction on $\sigma$.
\end{proof}
\end{frame}

\begin{frame}{Normalization function}
The previous proof does not give a normalization procedure \\
We want $\name{nf}_\sigma : \Tm[\sigma] \to \Nf[\sigma]$ with the following correctness properties:
\begin{itemize}
    \item (\textit{soundness}) if $\conv{t}{u}{\sigma}$, then $\name{nf}_\sigma(t) = \name{nf}_\sigma(u)$, and
    \item (\textit{completeness}) for all $\typedtm{t}{\sigma}$, we have $\conv{t}{\name{nf}_\sigma(t)}{\sigma}$.
\end{itemize}
\vspace{6pt}
How to implement $\name{nf}_\sigma$?
\end{frame}

\begin{frame}{Normalization by \textit{evaluation} (NbE)}
\begin{itemize}
    \item Idea: normalization is like evaluation but for open terms
    \item So, use denotational semantics
    \item We need to map semantic objects back to normal forms: `quote'
    \item Efficient: does all simplifications ``at once"
\end{itemize}
\end{frame}

%TODO is this slide necessary?
\begin{frame}{A quick intro to denotational semantics}
$(X_\beta)_{\beta \in \Basetypes}$ is a collection of sets
\begin{align*}
    \sem{\beta} &= X_\beta \quad(\beta \in \Basetypes) \\
    \sem{\functy{\sigma}{\tau}} &= \funcset{\sem{\sigma}}{\sem{\tau}}
\end{align*}
We get an interpretation of terms:
\[ \typedtm{t}{\sigma} \mapsto \sem{t} \in \sem{\sigma} \]
\end{frame}

\begin{frame}{Contexts}
\begin{itemize}
    \item We introduce contexts to keep track of free variables
    \item Context: finite set of variables
    \item $\cTm{\Gamma}{\sigma}$, $\cNe{\Gamma}{\sigma}$, $\cNf{\Gamma}{\sigma}$
\end{itemize}
\end{frame}

\begin{frame}{Main steps of NbE}
\begin{enumerate}
    \item Construct a model: $\sem{\sigma}$ $\sim$ logical predicate
    \begin{itemize}
        \item $\sem{\sigma}$ is a family of sets indexed by contexts
    \end{itemize}
    \item Interpretation of terms: $\sem{t}$ $\sim$ Fundamental Lemma
    \item Define $\quotef{\Gamma}{\sigma} : \sem{\sigma} \Gamma \to \cNf{\Gamma}{\sigma}$ $\sim$ $\sem{\sigma} \subs \mathrm{WN}_\sigma$
    \item Derive normalization function from (2) and (3)
\end{enumerate}
\end{frame}
%TODO remark that the definition of the normalization function is analogous to the proof of weak normalization

\begin{frame}{Step 1: Model construction}
Recall the logical predicate:
\begin{align*}
\sem{\beta}
    &= \setof{\typedtm{t}{\beta}}{t\text{ has a normal form}} \\
\sem{\functy{\sigma}{\tau}}
    &= \setof{\typedtm{t}{\functy{\sigma}{\tau}}}{\forall \typedtm{u}{\sigma}.\;u \in \sem{\sigma} \implies \app{t}{u} \in \sem{\tau}} \\
    &\cong \setof{f : \sem{\sigma} \to \sem{\tau}}{\exists t.\forall u.\;f(u) = \app{t}{u}}
\end{align*}
Model for NbE:
\begin{align*}
\sem{\beta} \Gamma &= \cNf{\Gamma}{\beta} \\
\sem{\functy{\sigma}{\tau}} \Gamma &= \setof{f \in \prod_{\Delta \sups \Gamma}{\funcset{\sem{\sigma} \Delta}{\sem{\tau} \Delta}}}{\ldots}
\end{align*}
\end{frame}

\begin{frame}{Step 2: Interpretation of terms}
For details, see thesis
\end{frame}

%TODO remark that to define quote, we need unquote, just like for the auxiliary lemma
\begin{frame}{Step 3: Quote and unquote}
Recall:
\begin{lem}
For all types $\sigma$, we have
\begin{enumerate}
    \item $\sem{\sigma} \subs \mathrm{WN}_\sigma$
    \item $\mathrm{WA}_\sigma \subs \sem{\sigma}$
\end{enumerate}
\end{lem}
The corresponding components of NbE are
\begin{enumerate}
    \item $\quotef{\Gamma}{\sigma} : \sem{\sigma} \Gamma \to \cNf{\Gamma}{\sigma}$
    \item $\unquotef{\Gamma}{\sigma} : \cNe{\Gamma}{\sigma} \to \sem{\sigma} \Gamma$
\end{enumerate}
They are defined mutually recursively, and their definition follows the proof of the above lemma
\end{frame}

\begin{comment}
\begin{frame}
\begin{itemize}
    \item Define $\quotef{\Gamma}{\sigma} : \sem{\sigma}\Gamma \to \cNf{\Gamma}{\sigma}$ and $\unquotef{\Gamma}{\sigma} : \cNe{\Gamma}{\sigma} \to \sem{\sigma}\Gamma$
    \item Define normalization function $\normf{\Gamma}{\sigma} : \cTm{\Gamma}{\sigma} \to \cNf{\Gamma}{\sigma}$
    \begin{align*}
        \normf{\Gamma}{\sigma}(t) &= \quotef{\Gamma}{\sigma}(\ksint{t}{\eta_\Gamma}{\Gamma}), \text{ where}\\
        \idenv{\Gamma}(x, \Gamma') &= \unquotef{\Gamma'}{\sigma}(x)
        \quad \text{for } (x : \sigma) \in \Gamma' \sups \Gamma
    \end{align*}
\end{itemize}
\end{frame}
\end{comment}

%TODO mention categorical proof: categorification of NbE, uses categorical gluing construction, the high level proof structure is essentially the same
%TODO the gluing category can be seen as the 'category of logical predicates'
%TODO evaluation is given the universal property of $\syncat$ (initiality)
\begin{frame}{Main steps of categorical proof}
The structure is the same as NbE!
\begin{itemize}
    \item Model construction: glued category $\gluecat$ $\sim$ category of logical predicates
    \item Evaluation functor $\cint[\glueint]{-}[\gluecat] : \syncat \to \gluecat$, where $\syncat$ is the initial model
    \item Define quote and unquote
    \item Derive normalization function from (2) and (3)
\end{itemize}
\end{frame}
%TODO we skip the details due to lack of time

%TODO summary
%TODO say something about why categorical proof is nice: can be generalized to more complex type theories; the methods can be used to prove other metatheoretic properties (canonicity, parametricity)
\begin{frame}{Conclusion}
\begin{itemize}
    \item We discussed different proofs of normalization (Tait, NbE, categorical)
    \item They look different, but they are similar
\end{itemize}
\end{frame}

\begin{frame}
\centering
\Large Questions?
\end{frame}

\end{document}
