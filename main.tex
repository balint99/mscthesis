% Deze template is gemaakt door Fons van der Plas (f.vanderplas@student.ru.nl) voor het publiek domein en mag gebruikt worden **zonder vermelding van zijn naam**.
% This template was created by Fons van der Plas (f.vanderplas@student.ru.nl) for the public domain, and may be used **without attribution**.
\documentclass{report}
\usepackage[utf8]{inputenc}     % for éô
\usepackage[english]{babel}     % for proper word breaking at line ends
\usepackage[a4paper, left=1.5in, right=1.5in, top=1.5in, bottom=1.5in]{geometry}
                                % for page size and margin settings
\usepackage{graphicx}           % for ?
\usepackage{amsmath,amssymb}    % for better equations
\usepackage{amsthm}             % for better theorem styles
\usepackage{mathtools}          % for greek math symbol formatting
\usepackage{enumitem}           % for control of 'enumerate' numbering
\usepackage{listings}           % for control of 'itemize' spacing
\usepackage{todonotes}          % for clear TODO notes
\usepackage{hyperref}           % page numbers and '\ref's become clickable

%%%%%%%%%%%%%%%%%%%%%%%%%%%%%%%%
%% SET TITLE PAGE VALUES HERE %%
%%%%%%%%%%%%%%%%%%%%%%%%%%%%%%%%
%             ||               %
%             ||               %
%             \/               %

\def\thesistitle{Normalization for the simply typed \texorpdfstring{$\lambda$}{lambda}-calculus}
\def\thesissubtitle{A trilogy}
\def\thesisauthorfirst{Bálint}
\def\thesisauthorsecond{Kocsis}
\def\thesissupervisorfirst{Niels}
\def\thesissupervisorsecond{van der Weide}
\def\thesissecondreaderfirst{Herman}
\def\thesissecondreadersecond{Geuvers}
\def\thesisdate{\today}


%             /\               %
%             ||               %
%             ||               %
%%%%%%%%%%%%%%%%%%%%%%%%%%%%%%%%
%% SET TITLE PAGE VALUES HERE %%
%%%%%%%%%%%%%%%%%%%%%%%%%%%%%%%%


%% FOR PDF METADATA
\title{\thesistitle}
\author{\thesisauthorfirst\space\thesisauthorsecond}
\date{\thesisdate}

%%%%%%%%%%%%%%%%%%%%%%%

\newlist{enum}{enumerate}{3}
\setlist[enum]{label = (\roman*), noitemsep, listparindent = \parindent}
\newlist{items}{itemize}{3}
\setlist[items]{label = $-$, noitemsep, listparindent = \parindent}

%%%%%%%%%%%%%%%%%%%%%%%%%%%%%%%%%%%%%%%%%%%%%%%%
% Packages
%%%%%%%%%%%%%%%%%%%%%%%%%%%%%%%%%%%%%%%%%%%%%%%%
\usepackage[utf8]{inputenc}
\usepackage[english]{babel}
\usepackage{comment}
\usepackage{amssymb}
\usepackage{amsthm}
\usepackage{amsmath}
\usepackage{fdsymbol}
\usepackage{mathrsfs}
\usepackage{quiver}
\usepackage{mathpartir}
\usepackage{xparse}
\usepackage{hyperref}

%%%%%%%%%%%%%%%%%%%%%%%%%%%%%%%%%%%%%%%%%%%%%%%%
% Global settings
%%%%%%%%%%%%%%%%%%%%%%%%%%%%%%%%%%%%%%%%%%%%%%%%
\parindent = 0.7\parindent
\parskip = 0pt

\tikzcdset{row sep/normal = 2.7em, column sep/normal = 2.7em}

%%%%%%%%%%%%%%%%%%%%%%%%%%%%%%%%%%%%%%%%%%%%%%%%
% Mathematical environments
%%%%%%%%%%%%%%%%%%%%%%%%%%%%%%%%%%%%%%%%%%%%%%%%
\theoremstyle{definition}
\newtheorem{defn}{Definition}[section]
\newtheorem{ex}[defn]{Example}
\newtheorem{notn}[defn]{Notation}
\newtheorem{con}[defn]{Construction}

\theoremstyle{plain}
\newtheorem{thm}[defn]{Theorem}
\newtheorem{prop}[defn]{Proposition}
\newtheorem{lem}[defn]{Lemma}
\newtheorem{cor}[defn]{Corollary}

\theoremstyle{remark}
\newtheorem{rem}[defn]{Remark}

%%%%%%%%%%%%%%%%%%%%%%%%%%%%%%%%%%%%%%%%%%%%%%%%
% Formatting
%%%%%%%%%%%%%%%%%%%%%%%%%%%%%%%%%%%%%%%%%%%%%%%%
\newcommand{\name}{\mathrm}
\renewcommand{\emph}{\textbf}
\newcommand{\sep}{\ |\ }

%%%%%%%%%%%%%%%%%%%%%%%%%%%%%%%%%%%%%%%%%%%%%%%%
% Abbreviations
%%%%%%%%%%%%%%%%%%%%%%%%%%%%%%%%%%%%%%%%%%%%%%%%
\newcommand{\subs}{\subseteq}
\newcommand{\sups}{\supseteq}
\newcommand{\ul}{\underline}
\newcommand{\To}{\Rightarrow}
\newcommand{\xto}[2][]{\xrightarrow[#1]{#2}}
\newcommand{\xfrom}[2][]{\xleftarrow[#1]{#2}}
\renewcommand{\phi}{\varphi}
\renewcommand{\epsilon}{\varepsilon}

%%%%%%%%%%%%%%%%%%%%%%%%%%%%%%%%%%%%%%%%%%%%%%%%
% Brackets
%%%%%%%%%%%%%%%%%%%%%%%%%%%%%%%%%%%%%%%%%%%%%%%%
\newcommand{\ang}[1]{\langle #1 \rangle}
\newcommand{\sem}[1]{\lsem #1 \rsem}

%%%%%%%%%%%%%%%%%%%%%%%%%%%%%%%%%%%%%%%%%%%%%%%%
% Sets & logic
%%%%%%%%%%%%%%%%%%%%%%%%%%%%%%%%%%%%%%%%%%%%%%%%
\renewcommand{\implies}{\To}
\renewcommand{\iff}{\Leftrightarrow}

\newcommand{\nul}{\emptyset}
\newcommand{\singel}{*}
\newcommand{\singset}[1][\singel]{\{#1\}}
\newcommand{\N}{\mathbb{N}}
\newcommand{\setof}[2]{\left\{ #1 \sep #2 \right\}}
\newcommand{\cartprod}[2]{#1 \times #2}
\newcommand{\funcset}[2]{#2^{#1}}
\newcommand{\powset}[1]{\mathcal{P}(#1)}
\newcommand{\finpowset}[1]{\mathcal{P}_{\mathrm{fin}}(#1)}

\newcommand{\pto}{\rightharpoonup}
\newcommand{\dom}[1]{\mathrm{dom}(#1)}

%%%%%%%%%%%%%%%%%%%%%%%%%%%%%%%%%%%%%%%%%%%%%%%%
% Basic category theory
%%%%%%%%%%%%%%%%%%%%%%%%%%%%%%%%%%%%%%%%%%%%%%%%
\newcommand{\cat}[1]{\mathcal{#1}}
\newcommand{\Ob}[1]{\mathrm{Ob}(#1)}
\newcommand{\Hom}[3][\mathrm{Hom}]{#1(#2, #3)}
\NewDocumentCommand{\id}{o}{1\IfValueT{#1}{_{#1}}}
\NewDocumentCommand{\idfunc}{o}{\name{Id}\IfValueT{#1}{_{#1}}}
\NewDocumentCommand{\inv}{s m}{\IfBooleanTF{#1}{(#2)^{-1}}{#2^{-1}}}

\newcommand{\Set}{\mathbf{Set}}
\newcommand{\Cat}{\mathbf{Cat}}
\newcommand{\op}[1]{#1^\mathrm{op}}
\newcommand{\prodcat}[2]{#1 \times #2}
\newcommand{\funccat}[2]{[#1, #2]}
\newcommand{\PSh}[1]{\name{PSh}(#1)}

%%%%%%%%%%%%%%%%%%%%%%%%%%%%%%%%%%%%%%%%%%%%%%%%
% Lambda calculus with variable names
%%%%%%%%%%%%%%%%%%%%%%%%%%%%%%%%%%%%%%%%%%%%%%%%
\newcommand{\Basetypes}{\Sigma}
\newcommand{\functy}[2]{#1 \to #2}
\newcommand{\Ty}{\mathrm{Ty}}

\newcommand{\Var}[1][]{V_{#1}}
\newcommand{\app}[2]{#1 \cdot #2}
\newcommand{\lamv}[3]{\lambda #1^{#2}\hspace{-1pt}.\;#3}
\newcommand{\Tm}[1][]{\Lambda_{#1}}
\newcommand{\typedvar}[2]{#1 : #2}
\newcommand{\typedtm}[2]{#1 : #2}

\newcommand{\FV}[2][]{\mathrm{FV}_{#1}(#2)}
\newcommand{\BV}[1]{\mathrm{BV}(#1)}

\newcommand{\Sub}{\mathrm{Sub}}
\newcommand{\varsubst}[2]{#1 := #2}
\newcommand{\singsubv}[2]{[\varsubst{#1}{#2}]}
\NewDocumentCommand{\substnf}{m m m}
    {[\varsubst{#1_1}{#2_1}, \ldots, \varsubst{#1_{#3}}{#2_{#3}}]}
\newcommand{\updsub}[3]{#1[\varsubst{#2}{#3}]}
\newcommand{\idsub}[1][]{\mathrm{id}_{#1}}
\newcommand{\subst}[2]{#1 #2}

\newcommand{\convrel}[1]{{=}_{#1}}
\newcommand{\conv}[3]{\vdash #1 = #2 : #3}

\NewDocumentCommand{\Ne}{o}{\mathrm{Ne}\IfValueT{#1}{_{#1}}}
\NewDocumentCommand{\Nf}{o}{\mathrm{Nf}\IfValueT{#1}{_{#1}}}

\newcommand{\Con}{\mathrm{Con}}
\newcommand{\cTm}[2]{\Lambda_{#2}(#1)}
\newcommand{\cNf}[2]{\mathrm{Nf}_{#2}(#1)}
\newcommand{\cNe}[2]{\mathrm{Ne}_{#2}(#1)}
\newcommand{\ctypedtm}[3]{#1 \vdash #2 : #3}
\newcommand{\ctypedconv}[4]{#1 \vdash #2 = #3 : #4}

%%%%%%%%%%%%%%%%%%%%%%%%%%%%%%%%%%%%%%%%%%%%%%%%
% Henkin models
%%%%%%%%%%%%%%%%%%%%%%%%%%%%%%%%%%%%%%%%%%%%%%%%
\newcommand{\struct}[1]{\mathcal{#1}}
\newcommand{\scomp}[2]{#1_{#2}}
\NewDocumentCommand{\sapp}{O{} O{}}{\mathrm{app}^{#2}_{#1}}

\NewDocumentCommand{\Env}{O{} O{}}{\mathrm{Env}^{#2}_{#1}}
\newcommand{\typedenv}[2]{#1 \vDash #2}
\newcommand{\updenv}[3]{#1[#2 := #3]}
\newcommand{\empenv}{\nul}
\newcommand{\sint}[3][]{\sem{#2}^{#1}_{#3}}

\newcommand{\stdmod}[1]{\struct{S}(#1)}
\newcommand{\tmmod}{\struct{L}}
\newcommand{\sintsub}[2]{\sint{#1}{#2}}
\newcommand{\modsat}[4]{#1 \vDash #2 = #3 : #4}

\newcommand{\mapenv}[2]{#1 #2}

%%%%%%%%%%%%%%%%%%%%%%%%%%%%%%%%%%%%%%%%%%%%%%%%
% Kripke models
%%%%%%%%%%%%%%%%%%%%%%%%%%%%%%%%%%%%%%%%%%%%%%%%
\newcommand{\kscomp}[3]{#1_{#2}^{#3}}
\NewDocumentCommand{\kstran}{O{} O{}}{\mathrm{i}_{#1}^{#2}}
\NewDocumentCommand{\ksapp}{O{} O{}}{\mathrm{app}_{#1}^{#2}}

\newcommand{\ctypedenv}[3]{#1 \vDash #2 \sep #3}
\newcommand{\kupdenv}[3]{#1[#2 := #3]}
\newcommand{\ksint}[3]{\sem{#1}^{#3}_{#2}}

\newcommand{\kstdmod}[2]{\struct{S}(#1, #2)}
\newcommand{\cmodsat}[5]{#1 \vDash #2 \vdash #3 = #4 : #5}

%%%%%%%%%%%%%%%%%%%%%%%%%%%%%%%%%%%%%%%%%%%%%%%%
% NbE
%%%%%%%%%%%%%%%%%%%%%%%%%%%%%%%%%%%%%%%%%%%%%%%%
\newcommand{\normf}[2]{\name{nf}^{#1}_{#2}}
\newcommand{\fresh}[2]{v_{#1,#2}}
\newcommand{\extfresh}[2]{#1^{+#2}}
\newcommand{\unquotef}[2]{\name{u}^{#1}_{#2}}
\newcommand{\quotef}[2]{\name{q}^{#1}_{#2}}
\newcommand{\idenv}[1]{\eta_{#1}}

%%%%%%%%%%%%%%%%%%%%%%%%%%%%%%%%%%%%%%%%%%%%%%%%
% Categorical preliminaries
%%%%%%%%%%%%%%%%%%%%%%%%%%%%%%%%%%%%%%%%%%%%%%%%
\newcommand{\1}[1][]{\mathbf{1}_{#1}}
\newcommand{\0}[1][]{\mathbf{0}_{#1}}
\NewDocumentCommand{\mterm}{o}{{!}\IfValueT{#1}{_{#1}}}
\NewDocumentCommand{\minit}{o}{{!}\IfValueT{#1}{_{#1}}}
\newcommand{\cprod}[2]{#1 \times #2}
\NewDocumentCommand{\mfst}{o}{\name{fst}\IfValueT{#1}{_{#1}}}
\NewDocumentCommand{\msnd}{o}{\name{snd}\IfValueT{#1}{_{#1}}}
\newcommand{\mpair}[2]{\ang{#1, #2}}
\newcommand{\cexp}[2]{#1 \To #2}
\newcommand{\mev}[1][]{\name{ev}_{#1}}
\newcommand{\mcurry}[1]{\lambda(#1)}

\newcommand{\CCC}{\mathbf{CCC}}

\NewDocumentCommand{\prodcmp}{o o}{s\IfValueT{#1}{_{#1}}\IfValueT{#2}{^{#2}}}
\NewDocumentCommand{\expcmp}{o o}{t\IfValueT{#1}{_{#1}}\IfValueT{#2}{^{#2}}}

\newcommand{\y}{y}
\newcommand{\yap}[1]{\y_{#1}}
\newcommand{\reind}[1]{#1^*}

\newcommand{\comma}[2]{#1 \mathbin{\hspace{-1pt}\downarrow} #2}
\newcommand{\commafst}[2]{P_{#1, #2}}
\newcommand{\commasnd}[2]{Q_{#1, #2}}
\newcommand{\commatrd}[2]{\theta_{#1, #2}}

%%%%%%%%%%%%%%%%%%%%%%%%%%%%%%%%%%%%%%%%%%%%%%%%
% Well-scoped, well-typed lambda calculus
% with explicit substitutions
%%%%%%%%%%%%%%%%%%%%%%%%%%%%%%%%%%%%%%%%%%%%%%%%
\newcommand{\sTy}{\Ty}
\newcommand{\sCon}{\mathrm{Con}}
\newcommand{\sempcon}{{[]}}
\newcommand{\sextcon}[2]{#1, #2}

\newcommand{\sTm}[2]{\mathrm{Tm}(#1; #2)}
\newcommand{\sSub}[2]{\mathrm{Sub}(#1; #2)}
\newcommand{\stypedtm}[3]{#2 \vdash #1 : #3}
\newcommand{\stypedsub}[3]{#2 \vdash #1 : #3}

\NewDocumentCommand{\vz}{o o}{\mathrm{v}\IfValueT{#1}{^{#1}}\IfValueT{#2}{_{#2}}}
\newcommand{\slam}[2]{\lambda^{#1}.\;#2}
\newcommand{\ssubst}[2]{#1 #2}

\newcommand{\sempsub}[1][]{{()}_{#1}}
\newcommand{\sextsub}[2]{(#1, #2)}
\NewDocumentCommand{\sidsub}{o}{\name{id}\IfValueT{#1}{_{#1}}}
\newcommand{\scompsub}[2]{#1 \circ #2}
\NewDocumentCommand{\spsub}{o o}{\name{p}\IfValueT{#1}{^{#1}}\IfValueT{#2}{_{#2}}}

\newcommand{\sVar}[2]{\mathrm{Var}(#1; #2)}

\newcommand{\stmsub}[1]{#1^\bullet}
\newcommand{\ssingsub}[1]{\ang{#1}}
\NewDocumentCommand{\sliftsub}{o m}{#2\IfValueT{#1}{^{#1}}}

\newcommand{\sconvreltm}[2]{{=}_{#1,#2}^\mathrm{Tm}}
\newcommand{\sconvrelsub}[2]{{=}_{#1,#2}^\mathrm{Sub}}
\newcommand{\sconvtm}[4]{#3 \vdash #1 = #2 : #4}
\newcommand{\sconvsub}[4]{#3 \vdash #1 = #2 : #4}

\newcommand{\sNe}[2]{\mathrm{Ne}(#1; #2)}
\newcommand{\sNf}[2]{\mathrm{Nf}(#1; #2)}
\newcommand{\stypedvar}[3]{#2 \vdash^{\mathrm{v}} #1 : #3}
\newcommand{\stypedne}[3]{#2 \vdash^{\mathrm{ne}} #1 : #3}
\newcommand{\stypednf}[3]{#2 \vdash^{\mathrm{nf}} #1 : #3}

%%%%%%%%%%%%%%%%%%%%%%%%%%%%%%%%%%%%%%%%%%%%%%%%
% Scwfs
%%%%%%%%%%%%%%%%%%%%%%%%%%%%%%%%%%%%%%%%%%%%%%%%
\NewDocumentCommand{\SCon}{o}{\mathrm{Con}\IfValueT{#1}{^{#1}}}
\NewDocumentCommand{\STy}{o}{\mathrm{Ty}\IfValueT{#1}{^{#1}}}
\NewDocumentCommand{\Sempcon}{o}{\bullet\IfValueT{#1}{^{#1}}}
\NewDocumentCommand{\Sextcon}{m m o}{#1 \cdot\IfValueT{#3}{^{#3}} #2}

\NewDocumentCommand{\STm}{o m o}
    {\IfValueTF{#1}
        {\mathrm{Tm}\IfValueT{#3}{^{#3}}(#1; #2)}
        {\mathrm{Tm}\IfValueT{#3}{^{#3}}_{#2}}}
\NewDocumentCommand{\SSub}{m m o}
    {\mathrm{Sub}\IfValueT{#3}{^{#3}}(#1; #2)}

\NewDocumentCommand{\Sq}{o o}{\mathrm{q}\IfValueT{#1}{^{#1}}\IfValueT{#2}{_{#2}}}
\newcommand{\Ssubst}[2]{#1[#2]}

\NewDocumentCommand{\Sempsub}{o o}{\etaa\IfValueT{#1}{_{#1}}\IfValueT{#2}{^{#2}}}
    \newcommand{\etaa}{\eta}
\NewDocumentCommand{\Sextsub}{m m o}{#1 .\IfValueT{#3}{^{#3}} #2}
\NewDocumentCommand{\Sidsub}{o o}{\name{id}\IfValueT{#1}{_{#1}}\IfValueT{#2}{^{#2}}}
\NewDocumentCommand{\Scompsub}{m m o}{#1 \circ\IfValueT{#3}{^{#3}} #2}
\NewDocumentCommand{\Spsub}{o o}{\name{p}\IfValueT{#1}{^{#1}}\IfValueT{#2}{_{#2}}}

\newcommand{\Stmsub}[1]{#1^\bullet}
\newcommand{\Ssingsub}[1]{\ang{#1}}
\NewDocumentCommand{\Sliftsub}{o m}{#2\IfValueT{#1}{^{#1}}}

\NewDocumentCommand{\Sfuncty}{m m o}{#1 \To\IfValueT{#3}{^{#3}} #2}
\NewDocumentCommand{\Sapp}{o m m o}
    {#2 \mathbin{\$\IfValueT{#1}{_{#1}}\IfValueT{#4}{^{#4}}} #3}
\NewDocumentCommand{\Slam}{m o m}{\lambdaa^{#1}\IfValueT{#2}{_{#2}}(#3)}
    \newcommand{\lambdaa}{\lambda}

\NewDocumentCommand{\scwfmorcon}{s m}{#2\IfBooleanT{#1}{^\name{Con}}}
\NewDocumentCommand{\scwfmorsub}{s m m m}
    {#2\IfBooleanT{#1}{^\name{Sub}}_{#3,#4}}
\NewDocumentCommand{\scwfmorty}{s m}{#2\IfBooleanT{#1}{^\name{Ty}}}
\NewDocumentCommand{\scwfmortm}{s m o m}
        {#2\IfBooleanT{#1}{^\name{Tm}}_{\IfValueTF{#3}{#3,#4}{#4}}}

\newcommand{\Ldom}{\lambda\mathrm{-}\mathbf{Dom}}

%%%%%%%%%%%%%%%%%%%%%%%%%%%%%%%%%%%%%%%%%%%%%%%%
% Interpretation, abstract syntax
%%%%%%%%%%%%%%%%%%%%%%%%%%%%%%%%%%%%%%%%%%%%%%%%

\NewDocumentCommand{\cint}{o m o}{\sem{#2}\IfValueT{#1}{_{#1}}\IfValueT{#3}{^{#3}}}
\NewDocumentCommand{\cintty}{o m}{\cint[#1]{#2}[\mathrm{Ty}]}
\NewDocumentCommand{\cintcon}{o m}{\cint[#1]{#2}[\mathrm{Con}]}
\NewDocumentCommand{\cinttm}{o m}{\cint[#1]{#2}[\mathrm{Tm}]}
\NewDocumentCommand{\cintsub}{o m}{\cint[#1]{#2}[\mathrm{Sub}]}

\newcommand{\syncat}{\cat{L}}
\newcommand{\canint}{I}
\newcommand{\Mod}{\mathbf{Mod}}

%%%%%%%%%%%%%%%%%%%%%%%%%%%%%%%%%%%%%%%%%%%%%%%%
% Gluing
%%%%%%%%%%%%%%%%%%%%%%%%%%%%%%%%%%%%%%%%%%%%%%%%
\newcommand{\pCon}{\sCon}
\newcommand{\inclcon}{i}
\NewDocumentCommand{\FSub}{o}{\mathrm{Sub}\IfValueT{#1}{_{#1}}}
\newcommand{\PNe}[1]{\mathrm{Ne}_{#1}}
\newcommand{\PNf}[1]{\mathrm{Nf}_{#1}}
\newcommand{\gluecat}{\cat{G}}
\newcommand{\gluefst}{\pi_1}
\newcommand{\gluesnd}{\pi_2}
\NewDocumentCommand{\glueint}{o}{\rhoo\IfValueT{#1}{_{#1}}}
    \newcommand{\rhoo}{\rho}
\newcommand{\gluevar}[1]{\nu_{#1}}
\newcommand{\glueneut}[1]{\mu_{#1}}
\newcommand{\predneut}[1]{m_{#1}}
\newcommand{\gluenorm}[1]{\eta_{#1}}
\newcommand{\prednorm}[1]{n_{#1}}
\newcommand{\gluepsh}[1]{R_{#1}}
\newcommand{\gluepred}[1]{r_{#1}}
\newcommand{\gluemortm}[1]{\tilde{#1}}
\newcommand{\gluemorsub}[1]{\tilde{#1}}
\newcommand{\gquote}[1]{\mathrm{q}^{#1}}
\newcommand{\gunquote}[1]{\mathrm{u}^{#1}}
\newcommand{\nvar}[1]{\mathrm{var}_{#1}}
\newcommand{\napp}[2]{\mathrm{app}_{#1,#2}}
\newcommand{\nlam}[2]{\mathrm{lam}_{#1,#2}}
\newcommand{\gnorm}[2]{\name{nf}^{#1}_{#2}}

%%%%%%%%%%%%%%%%%%%%%%%%%%%%%%%%%%%%%%%%%%%%%%%%
% Uncategorized
%%%%%%%%%%%%%%%%%%%%%%%%%%%%%%%%%%%%%%%%%%%%%%%%
\newcommand{\convset}[1]{C(#1)}
\newcommand{\substruct}{\subs}
\newcommand{\upset}[1]{\uparrow\hspace{-2pt}(#1)}
\newcommand{\singlist}[1]{{[#1]}}
\newcommand{\concat}[2]{#1, #2}


%%%%%%%%%%%%%%%%%%%%%%%

\begin{document}
\begin{titlepage}
	\thispagestyle{empty}
	\newcommand{\HRule}{\rule{\linewidth}{0.5mm}}
	\center
	\textsc{\Large Radboud University Nijmegen}\\[.7cm]
	\includegraphics[width=25mm]{img/in_dei_nomine_feliciter.eps}\\[.5cm]
	\textsc{Faculty of Science}\\[0.5cm]
	
	\HRule \\[0.4cm]
	{ \huge \bfseries \thesistitle}\\[0.1cm]
	\textsc{\thesissubtitle}\\
	\HRule \\[.5cm]
	\textsc{\large Thesis MSc Computing Science}\\[.5cm]
	
	\begin{minipage}{0.4\textwidth}
	\begin{flushleft} \large
	\emph{Author:}\\
	\thesisauthorfirst\space \textsc{\thesisauthorsecond}
	\end{flushleft}
	\end{minipage}
	~
	\begin{minipage}{0.4\textwidth}
	\begin{flushright} \large
	\emph{Supervisor:} \\
	\thesissupervisorfirst\space \textsc{\thesissupervisorsecond} \\[1em]
	\emph{Second reader:} \\
	\thesissecondreaderfirst\space \textsc{\thesissecondreadersecond}
	\end{flushright}
	\end{minipage}\\[4cm]
	\vfill
	{\large \thesisdate}\\
	\clearpage
\end{titlepage}

\tableofcontents

\chapter{Introduction} \label{chap:intro}
There are many different kinds of programming languages. One of the most basic ones is the (pure) simply typed $\lambda$-calculus, the basis for all typed functional languages. The simply typed $\lambda$-calculus sits in the intersection of logic and programming: via the Curry-Howard isomorphism, its type system corresponds to minimalistic logic, that is, logic with implication as the only connective.

One of the main aspects of programming language theory is the study of the properties of programming languages as formal systems. This branch is also called the \textit{metatheory} of programming languages. There are several metatheoretic properties one can consider, such as decidable type checking, (strong) normalization, canonicity, and confluence. In some sense, the validity of these properties all contribute to the well-behavedness of a particular programming language.

In this thesis, we are only concerned with the normalization property. Traditionally, normalization has been defined in the context of \textit{term rewriting}. One defines a collection of rewrite rules (also called \textit{reduction} rules) on the terms of a particular programming language, expressing how to simplify programs. In this setting, normalization refers to the process of applying the rewrite rules repeatedly until no more reductions are possible. A program in which no reductions are possible is called a \textit{normal form}.

One proof of normalization for the simply typed $\lambda$-calculus is due to Tait \cite{tait:1967:jsl}. Historically, this proof is important as it crucially relies on the type structure of the simply typed $\lambda$-calculus, instead of only considering the shapes of terms and their behavior under reduction. His method uses a so called \textit{logical predicate}, which is a family of predicates on $\lambda$-terms indexed by types and defined by induction on types.

Another important proof of normalization is due to Berger and Schwichtenberg \cite{DBLP:conf/lics/BergerS91}. They construct a \textit{normalization function} that computes the normal form of a given input term. An interesting aspect of their proof is that it is \textit{reduction-free}: it does not mention any notion of reduction and does not depend on term rewriting. Instead, it uses denotational semantics, that is, evaluation into a suitable model. This is why their method may be termed \textit{normalization by evaluation}. Berger and Schwichtenberg's algorithm, although somewhat technical, is easy to implement and provides an efficient solution to normalization.

\begin{comment}
One of the reasons for considering reduction-free normalization is the extensionality axiom
\begin{mathpar}
\inferrule[ext]
    {\conv{\app{t}{x}}{\app{t'}{x}}{\tau}}
    {\conv{t}{t'}{\functy{\sigma}{\tau}}}
\end{mathpar}
This axiom states that two functions are equal whenever their behaviors are the same, i.e. they give the same output for all inputs. The axiom is desirable since it allows us to reason about functions based on not only their definition but also their behavior.
\end{comment}

There have been multiple accounts of a categorical reconstruction of normalization by evaluation. The first one is due to Altenkirch, Hofmann, and Streicher \cite{altenkirch:1995:ctcs}. They employ a so-called \textit{twisted gluing} category from which both the normalization function and its correctness can be deduced. They note that normalization by evaluation is related to the categorical \textit{Artin-gluing} construction \cite{wraith:1974:jpaa} without making the relation explicit. The connection to categorical gluing has been made precise by Fiore \cite{fiore:2002:ppdp, fiore:2022:mscs}.

In this thesis, we present three normalization proofs for the simply typed $\lambda$-calculus. The first one is based on Tait's idea of a \textit{convertible} term. The second one is a version of normalization by evaluation, similar to Reynolds' work \cite{reynolds1998normalization}. The third one is a categorical proof based on the work of Fiore \cite{fiore:2002:ppdp, fiore:2022:mscs} and Sterling and Spitters \cite{sterling:2018:arxiv}.

Besides presenting the proofs, we provide a comparison of the structure of the proofs. Comparing the proofs adds to the understanding of the individual proofs themselves on their own. The main message of the thesis is that there are close analogies between the structure of all three proofs.

\section{Related work}

Normalization by evaluation was invented by Berger and Schwichtenberg \cite{DBLP:conf/lics/BergerS91}, who called their method `inverting the evaluation functional'. However, their coding mechanism for generating fresh variables is rather elaborate. A simpler numbering scheme was used in \cite{berger:1993:tlca} and \cite{dybjer:2002:appsem}.

Berger \cite{berger:1993:tlca} showed that a version of normalization by evaluation can be extracted from a constructive proof of strong normalization, similar to the proof of weak normalization in this thesis. He used Kreisel's modified realizability interpretation for intuitionistic logic as the program extraction method. Later, he and others \cite{berger:2006:sl} formalized these results in various proof assistants.

Dybjer and Filinski \cite{dybjer:2002:appsem} used normalization by evaluation for simply typed $\lambda$-calculus as the basis for type-directed partial evaluation.

As mentioned already, there have been several attempts at reconstructing Berger and Schwichtenberg's algorithm using the language of category theory, such as \cite{altenkirch:1995:ctcs} and \cite{cubric:1998:mscs}. Streicher's short note \cite{streicher:1998:appsem} summarizes the situation.

The method of normalization by evaluation has been extended to more complex systems. Altenkirch, Hofmann, and Streicher presented reduction-free categorical normalization proofs for a combinator version of System F \cite{altenkirch1996reduction} and then System F itself \cite{altenkirch1996systemf}. Altenkirch et al. \cite{DBLP:conf/lics/AltenkirchDHS01} considered simply typed $\lambda$-calculus with strong binary sums (categorical coproducts), while Altenkirch and Kaposi \cite{DBLP:conf/rta/AltenkirchK16} developed normalization by evaluation for dependent type theory. 

Kovács \cite{kovacs:2017:msc} formalized normalization by evaluation for an intrinsically scoped and typed version of the simply typed $\lambda$-calculus in Agda.

Fiore \cite{fiore:2002:ppdp} was the first to relate normalization by evaluation to categorical gluing. Sterling and Spitters \cite{sterling:2018:arxiv} prove normalization for free $\lambda$-theories generated from many-typed first-order signatures using categorical gluing.

Kaposi et al. \cite{DBLP:conf/rta/KaposiHS19} transfer the categorical gluing method to type theory and develop gluing for categories with families. Coquand \cite{DBLP:journals/tcs/Coquand19} also proves canonicity and normalization for dependent type theory. His construction arises as a special case of the work of Kaposi et al. \cite{DBLP:conf/rta/KaposiHS19}.

%\cite{plotkin:1980, statman:1985:ic, jung:1993:tlca, DBLP:conf/tlca/FioreS99}
%\cite{DBLP:conf/popl/BalatCF04, coquand:1997:mscs, altenkirch:2009}

\section{Overview}

In Chapter~\ref{chap:stlc}, we discuss the syntax and semantics of the simply typed $\lambda$ calculus using a traditional syntactic presentation and model theory for the simply typed $\lambda$-calculus. Then, in Chapter~\ref{chap:norm-stlc}, we discuss the first two proofs of normalization for the simply typed $\lambda$-calculus. In these two chapters, we try to avoid any reference to category theory to make the methods more accessible.

The goal of the following three chapters is to present the categorical proof of normalization. First, in Chapter~\ref{chap:cat-prelims}, we recall some basic definitions, constructions, and theorems from category theory. Then, in Chapter~\ref{chap:stlc-cat}, we discuss the syntax and semantics of the simply-typed lambda from a categorical perspective. Finally, in Chapter~\ref{chap:gluing}, we present the categorical proof in detail.

\section{Acknowledgements}

I thank Niels van der Weide for his guidance throughout the creation of this work. I also thank Márk Széles for his help in writing.

\begin{comment}
\begin{itemize}
    \item Bigger picture regarding research, how is it related to other stuff in computer science, how it fits in the field
    \begin{itemize}
        \item We are interested in all kinds of programming languages. One of the most basic ones is STLC. Intersection of logic and programming. Type system is representative of minimalistic logic. Shares features with programming languages. Notions of program, reduction. The programs can be normalized to get a result. In order to have a suitable language, we also need to incorporate several aspects in the metatheory. We need decidable type checking, (strong) normalization, canonicity, confluence, etc. Expand on normalization: Something about reduction/rewriting and normal forms in the context of reduction/rewriting. Normalization can then be implemented as repeated rewriting.
    \end{itemize}
    \item History of normalization proofs
    \begin{itemize}
        \item Historical of normalization proofs: Tait's proof
        \item new paragraph: Berger and Schwichtenberg (NbE) - say something about reduction-freeness
        \item Motivation for reduction-free normalization:
        \begin{itemize}
            \item More efficient implementation of a normalizer
            \item One advantage of using reduction-free normalization is that we can use extensionality. The extensionality rule says that that terms are equal whenever they have the same behavior. This rule allows us reason about programs by observing their behavior. For this reason, we consider convertibility instead of reduction.
            \item The reason we consider the eta rule is that it is equivalent to extensionality, which allows us to reason about programs based on their behavior.
        \end{itemize}
        \item AHS'95: first categorical reconstruction of NbE using an ad-hoc twisted gluing category, but they observe already that there are connections to categorical gluing
        \item Fiore'02: relates normalization by evaluation explicitly to categorical gluing.
    \end{itemize}
    \item What did I do in my thesis
    \begin{itemize}
        \item I present 3 normalization proofs. The first is based on Tait's idea of a convertible term. The second is a version of normalization by evaluation. The third one is a categorical proof based on Fiore's work.
        \item I also compare the structure of the proofs.
        \item Comparing the proofs adds to the understanding of the proofs themselves on their own.
    \end{itemize}
    \item Related work
    \begin{itemize}
        \item logical relations
        \item normalization by evaluation
        \item categorical gluing: applications of gluing: mention a bunch of papers that use gluing
    \end{itemize}
    \item Overview of the work
    \begin{itemize}
        \item In chapter ..., we do ...
    \end{itemize}
\end{itemize}
\end{comment}

\begin{comment}
Thoughts:
\begin{itemize}
\item Aim: present very simple proof first, then build towards more and more conceptual proofs
\item Provide historical context for the gluing method, and relate parts of gluing proof to Tait/NbE
\item gluing gained traction via considering more complex theories for which the syntactic methods (NbE) were really complicated
\item "The connection between normalization by evaluation, logical predicates and
semantic gluing constructions is a matter of folklore, worked out in varying degrees
within the literature." (Sterling, Spitters 2018)
\item History
\begin{itemize}
 \item AHS'95, SystemF: 1996
 \item lots of work on syntactic side
 \item Fiore: 2002
\end{itemize}
\end{itemize}

Acknowledgements
- Niels for helpful discussions etc.
- Márk and Herman for proofreading and feedback?
\end{comment}


\chapter{Syntax and semantics of the simply typed \texorpdfstring{$\lambda$}{lambda}-calculus} \label{chap:stlc}
%TODO expand the introduction? write separate introductions for subsections?
In this chapter, we discuss the syntax and semantics of the simply typed $\lambda$ calculus as can be found in the early literature on the topic (for instance, \cite{friedman1975equality, plotkin:1980}). The reason for this choice is that this presentation seems to give the simplest proof of normalization.

In Section~\ref{sec:stlc-syntax}, we formally define the syntax of the simply typed $\lambda$-calculus. Besides the language of $\lambda$-calculus, we discuss important syntactic notions such as substitution and $\beta\eta$-conversion.

In Section~\ref{sec:stlc-semantics}, we present some basic model theory for the simply typed $\lambda$-calculus. In particular, we prove a soundness theorem and construct the term model.

\section{Syntax} \label{sec:stlc-syntax}

\subsection{Types and terms}

\begin{defn}[Types] \label{def:stlc-types}
We fix some nonempty set $\Basetypes$ of \emph{base types}. The \emph{types} of the simply typed $\lambda$-calculus are generated by the grammar
\[ \sigma, \tau ::= \beta \sep \functy{\sigma}{\tau}, \]
where $\beta$ ranges over $\Basetypes$.
\end{defn}

The type former $\functy{-}{-}$ associates to the right. The set of types is denoted by $\Ty$.

\begin{defn}[Terms] \label{def:stlc-terms}
We fix a pairwise disjoint collection of countable sets $\Var[\sigma]$ indexed by types. Elements of $\Var[\sigma]$ are called \emph{variables of type $\sigma$}. We inductively generate a family of sets $\Tm[\sigma]$ indexed by types $\sigma$ according to the rules in Figure~\ref{fig:stlc-terms}; that is, $(\Tm[\sigma])_{\sigma \in \Ty}$ is the smallest collection of sets closed under the rules. Elements of $\Tm[\sigma]$ are called \emph{terms of type $\sigma$}.
\begin{figure}[h]
\begin{mathpar}
\inferrule[var]
    {x \in \Var[\sigma]}
    {x \in \Tm[\sigma]}
\and
\inferrule[app]
    {t \in \Tm[\functy{\sigma}{\tau}] \and u \in \Tm[\sigma]}
    {\app{t}{u} \in \Tm[\tau]}
\and
\inferrule[lam]
    {x \in \Var[\sigma] \and t \in \Tm[\tau]}
    {\lamv{x}{\sigma}{t} \in \Tm[\functy{\sigma}{\tau}]}
\end{mathpar}
\caption{Terms of the simply typed $\lambda$-calculus}
\label{fig:stlc-terms}
\end{figure}
\end{defn}

The application operator $\app{-}{-}$ associates to the left. We write $\Var = \bigcup_{\sigma \in \Ty}{\Var[\sigma]}$ and $\Tm = \bigcup_{\sigma \in \Ty}{\Tm[\sigma]}$. If $t \in \Var[\sigma]$ (respectively $x \in \Tm[\sigma]$), we also write $\typedtm{t}{\sigma}$ (respectively $\typedvar{x}{\sigma}$) and read it as "$t$ (respectively $x$) has type $\sigma$". We also say that $t$ is an \emph{inhabitant} of $\sigma$, or that $\sigma$ is inhabited by $t$.

The abstraction operator $\lambda$ binds the variable $x$ in the term $\lamv{x}{\sigma}{t}$, similarly how a quantifier binds a variable in a formula. In other words, $\lamv{x}{\sigma}{t}$ introduces the variable $x$ whose scope extends to $t$. The occurrences of $x$ in $t$ are thus called \emph{bound}. Variables that are not bound are called \emph{free}.

\begin{defn} \label{def:free-and-bound-vars}
\begin{enum}
\item The sets of \emph{free variables} $\FV{t}$ and \emph{bound variables} $\BV{t}$ of a term $t$ are defined recursively:
\begin{align*}
\FV{-} &: \Tm \to \finpowset{\Var} &
    \BV{-} &: \Tm \to \finpowset{\Var} \\
\FV{x} &= \singset[x] &
    \BV{x} &= \nul \\
\FV{\app{t}{u}} &= \FV{t} \cup \FV{u} &
    \BV{\app{t}{u}} &= \BV{t} \cup \BV{u} \\
\FV{\lamv{x}{\sigma}{t}} &= \FV{t} \setminus \singset[x] &
    \BV{\lamv{x}{\sigma}{t}} &= \BV{t} \cup \singset[x]
\end{align*}
\item $\FV[\sigma]{t} = \FV{t} \cap \Var[\sigma]$ is the set of free variables of $t$ of type $\sigma$.
\item A term $t$ is \emph{closed} if $\FV{t} = \nul$.
\end{enum}
\end{defn}

Generally in mathematics, the names of bound variables do not matter. The formulas $\forall x. P(x)$ and $\forall y. P(y)$ both denote the statement ``$P$ holds for all individuals''. Similarly, the terms $\lamv{x}{\sigma}{x}$ and $\lamv{y}{\sigma}{y}$ both denote the identity function that returns its input unchanged.

\begin{defn}[$\alpha$-convertibility]
Two terms are \emph{$\alpha$-convertible} if they only differ in the names of bound variables. Details can be found in Chapter 2 of \cite{barendregt:1984}.
\end{defn}

From now on, we identify $\alpha$-convertible terms. This means that we do not distinguish between $\alpha$-convertible terms and treat them as being equal. Formally, this amounts to taking a quotient of terms by the equivalence relation of $\alpha$-convertibility. A consequence of this is that all operations and properties on terms have to be defined on equivalence classes rather than individual terms. In practice, this means that whenever we use a representative of some equivalence class in a definition, we have to prove the invariance of the definition under $\alpha$-convertibility. The details of verifying such assertions are tedious and not too interesting, so we leave them out. See also the discussion in Appendix C of \cite{barendregt:1984}.

\begin{rem} \label{rem:variable-convention}
When working with representatives of $\alpha$-equivalence classes, we employ Barendregt's \emph{variable convention} \cite{barendregt:1984}. This means that all bound variables of all terms that occur in a certain mathematical context (e.g. definition, proof) are assumed to be distinct from all free variables of the terms. Given a countable collection of terms, it is always possible choose representatives in such a way that this condition is satisfied. The convention allows for a simpler treatment of substitution, see Definition~\ref{def:subst-in-term}.
\end{rem}

\begin{rem}
The base types in $\Basetypes$ have no closed inhabitants, so one might wonder why they are necessary. Indeed, the essence of lambda calculus lies in the function types. However, since types are defined inductively, one needs at least one base type as a base case. Otherwise, the set of types would be empty.
\end{rem}

\subsection{Substitution}

The free variables of a term may be viewed as placeholders in which other terms may be substituted. For instance, if $t = \app{x}{y}$ with $\typedvar{x}{\functy{\sigma}{\sigma}}$ and $u = \lamv{z}{\sigma}{z}$, then we can substitute $u$ for $x$ in $t$, denoted by $\subst{t}{\singsubv{x}{u}}$, to obtain $\app{(\lamv{z}{\sigma}{z})}{y}$.

Instead of defining $\subst{t}{\singsubv{x}{u}}$, we take a more general approach where multiple terms may be substituted simultaneously for distinct free variables. This operation is also referred to as \textit{parallel substitution}. We simply call it substitution.

\begin{defn}[Substitution] \label{def:substitution}
A \emph{substitution} is a partial function $\gamma : \Var \pto \Tm$ with finite domain such that if $\typedvar{x}{\sigma}$ and $x \in \dom{\gamma}$, then $\typedtm{\gamma(x)}{\sigma}$.
\end{defn}

The set of substitutions is denoted by $\Sub$.

\begin{notn} \label{not:substitutions}
\hfill \vspace{-3pt}
\begin{items}
\item The notation $\substnf{x}{t}{n}$ stands for the substitution $\gamma$ with
\[ \dom{\gamma} = \{x_1, \ldots, x_n\} \quad\text{and}\quad
    \gamma(x_i) = t_i \quad (i = 1, \ldots, n). \]

\item For a substitution $\gamma$, variable $\typedvar{x}{\sigma}$, and term $\typedtm{t}{\sigma}$, we write $\updsub{\gamma}{x}{t}$ for the \emph{updated substitution} with $\dom{\updsub{\gamma}{x}{t}} = \dom{\gamma} \cup \singset[x]$ and such that
\[
\updsub{\gamma}{x}{t}(y) = \begin{cases}
                              \gamma(y) & \text{if } y \ne x \\
                              t & \text{if } y = x  
                            \end{cases}.
\]

\item For a finite set of variables $\Gamma \subs \Var$, the \emph{identity substitution $\idsub[\Gamma]$ on $\Gamma$} is given by
\[ \idsub[\Gamma](x) = x \quad (x \in \Gamma). \]

\item If $\gamma$ is a substitution, we write
\[ \FV{\gamma} = \bigcup_{x \in \dom{\gamma}}{\FV{\gamma(x)}}. \]
\end{items}
\end{notn}

\begin{defn}[Substitution in terms] \label{def:subst-in-term}
For $\gamma = \substnf{x}{t}{n}$, the expression $\subst{t}{\gamma}$ denotes the result of simultaneously substituting the terms $t_1, \ldots, t_n$ for the free variables $x_1, \ldots, x_n$ of $t$. Formally, it is defined by recursion:
\begin{align*}
\subst{(-)}{(-)} &: \cartprod{\Tm[\sigma]}{\Sub} \to \Tm[\sigma] \\
\subst{x}{\gamma} &= \begin{cases}
                       \gamma(x) & \text{if $x \in \dom{\gamma}$} \\
                       x         & \text{otherwise}
                     \end{cases} \\
\subst{(\app{t}{u})}{\gamma} &= \app{\subst{t}{\gamma}}{\subst{u}{\gamma}} \\
\subst{(\lamv{x}{\sigma}{t})}{\gamma} &= \lamv{x}{\sigma}{\subst{t}{(\updsub{\gamma}{x}{x})}}
\end{align*}
\end{defn}

Note that, by definition, substitution preserves types: if $\typedtm{t}{\sigma}$, then $\typedtm{\subst{t}{\gamma}}{\sigma}$. Furthermore, we have $\FV{\subst{t}{\gamma}} \subs (\FV{t} \setminus \dom{\gamma}) \cup \FV{\gamma}$.

Note also that we use the term \textit{substitution} in two senses. It refers both to the data specifying the terms to be substituted (Definition~\ref{def:substitution}) and to the operation defined in Definition~\ref{def:subst-in-term}.

\begin{rem}
The variable convention (see Remark~\ref{rem:variable-convention}) ensures that we avoid variable capturing, i.e. when a free variable of some $t_i$ becomes bound in the substituted term. Concretely, when writing $\subst{t}{\gamma}$, we assume $\BV{t} \cap \FV{\gamma} = \nul$.
\end{rem}

\subsection{Conversion}

A key feature of the $\lambda$-calculus is the so called $\beta$-rule: applying a function $\lamv{x}{\sigma}{t}$ to an argument $u$ results in substituting $u$ for $x$ in $t$. This is expressed by an equational theory between $\lambda$-terms.

\begin{defn}[$\beta\eta$-conversion]
The typed \emph{conversion relation} is a family of relations indexed by types, generated inductively by the rules in Figure~\ref{fig:stlc-equations}, i.e. it is the least family of relations closed under the rules. We write $\conv{t}{u}{\sigma}$ to mean $(t, u) \in \convrel{\sigma}$.
\begin{figure}[ht]
\begin{mathpar}
\inferrule[refl]{\typedtm{t}{\sigma}}
    {\conv{t}{t}{\sigma}}
\and
\inferrule[trans]
    {\conv{t}{u}{\sigma} \and \conv{u}{v}{\sigma}}
    {\conv{t}{v}{\sigma}}
\and
\inferrule[sym]
    {\conv{t}{u}{\sigma}}
    {\conv{u}{t}{\sigma}}
\\
\inferrule[cong-app]
    {\conv{t}{t'}{\functy{\sigma}{\tau}} \and \conv{u}{u'}{\sigma}}
    {\conv{\app{t}{u}}{\app{t'}{u'}}{\tau}}
\and
\inferrule[cong-lam]
    {\conv{t}{t'}{\tau}}
    {\conv{\lamv{x}{\sigma}{t}}{\lamv{x}{\sigma}{t'}}{\functy{\sigma}{\tau}}}
\\
\inferrule[beta]
    {\typedtm{t}{\tau} \and \typedtm{u}{\sigma}}
    {\conv{\app{(\lamv{x}{\sigma}{t})}{u}}{\subst{t}{\singsubv{x}{u}}}{\tau}}
\and
\inferrule[eta]
    {\typedtm{t}{\functy{\sigma}{\tau}}}
    {\conv{\lamv{x}{\sigma}{\app{t}{x}}}{t}{\functy{\sigma}{\tau}}}
\quad(x \notin \FV{t})
\end{mathpar}
\caption{Rules for equations between $\lambda$-terms}
\label{fig:stlc-equations}
\end{figure}
\end{defn}

Note that by definition, only terms of the same type can be related. This is in contrast with the approach taken in operational semantics where a reduction relation is defined on untyped terms.

The following lemma states that substitution respects conversion.

\begin{lem} \label{lem:sub-conv}
Suppose $\conv{t}{t'}{\sigma}$ and $\gamma, \gamma' \in \Sub$ such that
\[ \dom{\gamma} = \dom{\gamma'} \quad\text{and}\quad
    \conv{\gamma(x)}{\gamma'(x)}{\tau} \quad \text{for all } \typedvar{x}{\tau}. \]
Then $\conv{\subst{t}{\gamma}}{\subst{t'}{\gamma'}}{\sigma}$.
\begin{proof}
%TODO complete this proof?
By induction on the proof of $\conv{t}{t'}{\sigma}$ using the congruence rules \textsc{cong-app} and \textsc{cong-lam}.
\end{proof}
\end{lem}

\subsection{Normal forms}

The rule \textsc{beta} can also be viewed as a reduction rule: $\app{(\lamv{x}{\sigma}{t})}{u}$ \textit{reduces} to $\subst{t}{\singsubv{x}{u}}$. This rule is then called \textit{$\beta$-reduction}. Introducing a directionality is particularly useful for applications to programming: it allows one to run a program by applying reduction steps in sequence. Evaluation of a program is finished when no more reductions are applicable. In this case, the program is said to be in \textit{$\beta$-normal form}.

In the extensional $\lambda$-calculus ($\lambda$-calculus with the rule \textsc{eta}), it is also natural to consider normality with respect to the \textsc{eta} rule. The natural direction for this rule is reducing $\lamv{x}{\sigma}{\app{t}{x}}$ to $t$ (with $x \notin \FV{t}$). Terms in which no more \textsc{beta} or \textsc{eta}-reductions are applicable are said to be in \textit{$\beta\eta$-normal form}.

In a $\beta\eta$-normal form, functions can be partially applied. For instance, if $\typedvar{f}{\functy{\sigma}{\functy{\sigma}{\sigma}}}$ and $\typedvar{x}{\sigma}$ are variables, then $f$ and $\app{f}{x}$ are both in $\beta\eta$-normal form. It is also possible to consider another notion of canonical form, where all functions are fully applied. This is achieved by applying the rule \textsc{eta} in reverse direction, referred to as \textit{$\eta$-expansion}, until all missing arguments are supplied. For instance, $\app{f}{x}$ is converted into $\lamv{y}{\sigma}{\app{\app{f}{x}}{y}}$ (for a fresh variable $y$) in this process. These canonical forms are called \textit{long $\beta\eta$-normal forms}.

To make the distinction between the two notions of normal form more explicit, the former notion is also referred to as \textit{short $\beta\eta$-normal form}. Alternative names for the two notions include \textit{$\eta$-short $\beta$-normal form} and \textit{$\eta$-long $\beta$-normal form}, respectively. The latter terminology emphasizes the fact that $\eta$-long $\beta$-normal forms are normal with respect to $\beta$-reduction but not $\eta$-reduction.

It turns out that long $\beta\eta$-normal forms can be characterized syntactically by a simple set of rules. Since we are only interested in long $\beta\eta$-normal forms, we simply call them normal forms.

\begin{defn}[Neutral terms, normal forms] \label{def:normal-forms}
We define two families of sets of $\lambda$-terms, the sets of \emph{neutral terms} $\Ne[\sigma] \subs \Tm[\sigma]$ and the sets of \emph{normal forms} $\Nf[\sigma] \subs \Tm[\sigma]$. The two families are generated mutually inductively by the clauses in Figure~\ref{fig:stlc-normal-forms}.
\begin{figure}[ht]
\begin{mathpar}
\inferrule[var-ne]
    {x \in \Var[\sigma]}
    {x \in \Ne[\sigma]}
\and
\inferrule[app-ne]
  {m \in \Ne[\functy{\sigma}{\tau}] \and n \in \Nf[\sigma]}
  {\app{m}{n} \in \Ne[\tau]}
\\
\inferrule[shift]
    {m \in \Ne[\beta]}
    {m \in \Nf[\beta]}
\quad (\beta \in \Basetypes)
\and
\inferrule[lam-nf]
    {n \in \Nf[\tau]}
    {\lamv{x}{\sigma}{n} \in \Nf[\functy{\sigma}{\tau}]}
\end{mathpar}
\caption{Normal forms and neutral terms}
\label{fig:stlc-normal-forms}
\end{figure}
\end{defn}

Neutral terms are an auxiliary class of terms used in the definition of normal forms. They are terms with a variable in head position preventing the application of the rule \textsc{beta}.

\begin{defn} \label{def:has-normal-form}
Let $\typedtm{t}{\sigma}$ be a term. We say that \emph{$t$ has a normal form} if there exists an $n \in \Nf[\sigma]$ such that $\conv{t}{n}{\sigma}$. We then also say that \emph{$n$ is a normal form of $t$} or \emph{$t$ has the normal form $n$}.
\end{defn}

\section{Semantics} \label{sec:stlc-semantics}
%TODO make the proofs more readable (e.g. replace stub words such as 'case' by full sentences or phrases)?

\subsection{Applicative structures, models} \label{sec:models}

We first define an appropriate notion of model for the simply typed $\lambda$-calculus.

\begin{defn}[Applicative structure]
An \emph{applicative structure} $\struct{A}$ consists of
\begin{items}
    \item a family of sets $\scomp{\struct{A}}{\sigma}$ indexed by types $\sigma$, and
    \item a family of maps $\sapp[\sigma,\tau][\struct{A}] : \cartprod{\scomp{\struct{A}}{\functy{\sigma}{\tau}}}{\scomp{\struct{A}}{\sigma}} \to \scomp{\struct{A}}{\tau}$ indexed by pairs of types $\sigma, \tau$.
\end{items}
The maps $\sapp[\sigma, \tau]$ are called \emph{application maps}.
\end{defn}

It is worth noting that even though terms of type $\functy{\sigma}{\tau}$ are thought of as functions, $\scomp{\struct{A}}{\functy{\sigma}{\tau}}$ does not necessarily have to be a set of functions. However, every element $f \in \scomp{\struct{A}}{\functy{\sigma}{\tau}}$ of an applicative structure $\struct{A}$ determines a function
\[ \sapp[\sigma,\tau][\struct{A}](f, -) : \scomp{\struct{A}}{\sigma} \to \scomp{\struct{A}}{\tau}, \quad
    x \mapsto \sapp[\sigma,\tau][\struct{A}](f, x). \]
which represents the functional behavior of $f$.

\begin{notn}
We often omit the superscripts and subscripts of the application maps. Furthermore, we often drop the application map completely and write $fx$ for $\sapp[\sigma,\tau](f, x)$ when this does not cause confusion.
\end{notn}

\begin{defn}[Extensional applicative structure] \label{def:ext-app-struct}
We say that an applicative structure $\struct{A}$ is \emph{extensional} if the following holds for all $f, g \in \scomp{\struct{A}}{\functy{\sigma}{\tau}}$:
\[ (\forall x \in \scomp{\struct{A}}{\sigma}. fx = gx) \quad\text{implies}\quad f = g. \]
\end{defn}

\begin{rem} \label{rem:ext-app-struct}
The assignment $f \mapsto \sapp(f, -)$ gives a map $\scomp{\struct{A}}{\functy{\sigma}{\tau}} \to \funcset{\scomp{\struct{A}}{\sigma}}{\scomp{\struct{A}}{\tau}}$. The extensionality axiom is equivalent to the statement that this map is injective. Thus, in an extensional applicative structure, we may identify $\scomp{\struct{A}}{\functy{\sigma}{\tau}}$ with a subset of $\funcset{\scomp{\struct{A}}{\sigma}}{\scomp{\struct{A}}{\tau}}$. Under this identification, the application map $\sapp[\sigma,\tau]$ becomes the evaluation map $(f, x) \mapsto f(x)$.
\end{rem}

To be able to interpret $\lambda$-terms in an applicative structure, one needs to choose meanings for the free variables of the term. This role is filled by environments.

\begin{defn}[Environment] \label{def:environment}
Let $\struct{A}$ be an applicative structure.
\begin{enum}
\item An \emph{environment for $\struct{A}$} is a partial function $\rho : \Var \pto \bigcup_{\sigma \in \Ty}{\scomp{\struct{A}}{\sigma}}$ with finite domain such that if $\typedvar{x}{\sigma}$ and $x \in \dom{\rho}$, then $\rho(x) \in \scomp{\struct{A}}{\sigma}$.

\item If $\Gamma \subs \Var$ is a finite set of variables, then a \emph{$\Gamma$-environment} is an environment $\rho$ such that $\Gamma \subs \dom{\rho}$.
\end{enum}
\end{defn}

The set of environments for $\struct{A}$ is denoted by $\Env[][\struct{A}]$. We write $\Env[\Gamma][\struct{A}]$ for the set of $\Gamma$-environments. As usual, the superscript can be omitted. If $\rho \in \Env[\Gamma]$, we also write $\typedenv{\rho}{\Gamma}$. Clearly, if $\Gamma \subs \Gamma'$ and $\typedenv{\rho}{\Gamma'}$, then $\typedenv{\rho}{\Gamma}$.

Note the formal similarity between substitutions and environments. Environments can be seen as the semantic counterpart to substitutions. Whereas a substitution maps free variables to terms, an environment maps them to elements of a structure. The two notions are connected in the term model (Definition~\ref{def:term-model}), for which environments are essentially the same as substitutions.

\begin{notn} \label{not:environments}
\hfill \vspace{-3pt}
\begin{items}
\item For an environment $\rho$, variable $\typedvar{x}{\sigma}$, and $a \in \scomp{\struct{A}}{\sigma}$, we write $\updenv{\rho}{x}{a}$ for the \emph{updated environment} with $\dom{\updenv{\rho}{x}{a}} = \dom{\rho} \cup \singset[x]$ and such that
\[ \updenv{\rho}{x}{a}(y) = \begin{cases}
                            \rho(y) & \text{if } y \ne x \\
                            a & \text{if } y = x.
                            \end{cases} \]

\item The \emph{empty environment} $\empenv$ is the empty function.
\end{items}
\end{notn}

The updated environment is the semantic analogue of the updated substitution (Notation~\ref{not:substitutions}).

\begin{defn}[Environment model] \label{def:env-model}
An \emph{environment model} $\struct{A}$ is an extensional applicative structure $\struct{A}$ together with an assignment
\[ (t, \rho) \mapsto \sint[\struct{A}]{t}{\rho} \in \scomp{\struct{A}}{\sigma}
    \quad \text{for }\typedtm{t}{\sigma} \text{ and } \typedenv{\rho}{\FV{t}} \]
such that the following equations are satisfied:
\begin{align}
\sint[\struct{A}]{x}{\rho} &= \rho(x) \\
\sint[\struct{A}]{\app{t}{u}}{\rho} &= \sint[\struct{A}]{t}{\rho}\sint[\struct{A}]{u}{\rho} \\
\sint[\struct{A}]{\lamv{x}{\sigma}{t}}{\rho} a &= \sint[\struct{A}]{t}{\updenv{\rho}{x}{a}} \label{eq:int-lam} \quad (a \in \scomp{\struct{A}}{\sigma})
\end{align}
\end{defn}

The value $\sint[\struct{A}]{t}{\rho}$ is called the \emph{interpretation} of $t$ at environment $\rho$ in the model $\struct{A}$. As before, we usually omit the superscript $\struct{A}$.

\begin{rem} \label{rem:env-model-int-uniq}
It follows by a simple induction on terms and the extensionality of $\struct{A}$ that there is at most one possible value for the interpretation $\sint{t}{\rho}$ of $t$. Hence, the conditions in Definition~\ref{def:env-model} can almost be seen as a recursive definition of the interpretation of $\lambda$-terms. The only problem is that the interpretation of $\lamv{x}{\sigma}{t} : \functy{\sigma}{\tau}$ may not be defined if $\scomp{\struct{A}}{\functy{\sigma}{\tau}}$ does not contain enough elements.

More specifically, recall from Remark~\ref{rem:ext-app-struct} that if $\struct{A}$ is an extensional applicative structure, then $\scomp{\struct{A}}{\functy{\sigma}{\tau}}$ may be identified with a subset of $\funcset{\scomp{\struct{A}}{\sigma}}{\scomp{\struct{A}}{\tau}}$. By (\ref{eq:int-lam}), the interpretation $\sint{\lamv{x}{\sigma}{t}}{\rho}$ has to be equal to the function $\scomp{\struct{A}}{\sigma} \to \scomp{\struct{A}}{\tau}$ given by $a \mapsto \sint{t}{\updenv{\rho}{x}{a}}$. However, it is possible that this function is not in $\scomp{\struct{A}}{\functy{\sigma}{\tau}}$.

An environment model is thus an extensional applicative structure in which it is possible to interpret every $\lambda$-term. Hence, we can say that an extensional applicative structure $\struct{A}$ \textit{is} an environment model if the uniquely defined interpretation function is well-defined.
\end{rem}

The intended semantics for the simply typed $\lambda$-calculus is given by sets and functions.

\begin{ex}[Standard model]
Let $X = (X_\beta)_{\beta \in \Basetypes}$ be a family of sets indexed by base types. We define the model $\stdmod{X}$, called the \emph{standard model (over $X$)}, as follows. The sets $\stdmod{X}_\sigma$ are given by recursion on types:
\begin{align*}
\stdmod{X}_\beta &= X_\beta \quad (\beta \in \Basetypes) \\
\stdmod{X}_{\functy{\sigma}{\tau}} &= \funcset{\stdmod{X}_\sigma}{\stdmod{X}_\tau}
\end{align*}
The application map $\sapp[\sigma,\tau][\stdmod{X}] : \cartprod{\funcset{\stdmod{X}_\sigma}{\stdmod{X}_\tau}}{\stdmod{X}_\sigma} \to \stdmod{X}_\tau$ is the evaluation map $(f, x) \mapsto f(x)$. It is clear that this applicative structure is extensional. The interpretation of $\lambda$-terms is defined recursively according to the conditions in Definition~\ref{def:env-model}.
\end{ex}

%TODO make more clear: why does it feel strange from a model-theoretic perspective?
An applicative structure contains an application map to model application of $\lambda$ terms. On the other hand, $\lambda$-abstraction does not have an immediate semantic counterpart, which explains why it is not always possible to interpret a $\lambda$-abstraction. In Definition~\ref{def:env-model}, this was solved by postulating that the intended interpretation function exists. This might feel strange from a model-theoretic perspective: a semantic structure should be adequate to interpret syntax without us having to verify this explicitly. There is an alternative formulation of when an extensional applicative structure is a model of $\lambda$-calculus which meets this criterion.

\begin{defn}[Combinatory model] \label{def:comb-model}
A \emph{combinatory model} $\struct{A}$ is an extensional applicative structure $\struct{A}$  with distinguished elements
\[ K^\struct{A}_{\sigma,\tau} \in \struct{A}_{\functy{\sigma}{\functy{\tau}{\sigma}}}
    \quad\text{and}\quad
    S^\struct{A}_{\sigma,\tau,\chi} \in \struct{A}_{\functy{(\functy{\sigma}{\functy{\tau}{\chi}})}{(\functy{\functy{\sigma}{\tau})}{\functy{\sigma}{\chi}}}} \]
for every $\sigma, \tau, \chi \in \Ty$ such that
\begin{equation} \label{eq:comb-model-KS}
K^\struct{A}_{\sigma,\tau}xy = x \quad\text{and}\quad
    S^\struct{A}_{\sigma,\tau,\chi}fgx = fx(gx)
\end{equation}
for all $x, y, f, g$ of the appropriate types.
\end{defn}

As for the application maps, we may omit the superscripts and subscripts of $K$ and $S$.

\begin{rem}
Combinatory models are so named because they provide semantics for \textit{typed combinatory logic}, a system equivalent to the simply typed $\lambda$-calculus. Combinatory logic bypasses the use of variables by expressing all functions using certain elementary functions called \textit{combinators}. Specifically, it has constants
\[ \typedtm{K_{\sigma,\tau}}{\functy{\sigma}{\functy{\tau}{\sigma}}} \quad\text{and}\quad
    \typedtm{S_{\sigma,\tau,\chi}}{(\functy{\functy{\sigma}{\functy{\tau}{\chi}})}{\functy{(\functy{\sigma}{\tau})}{\functy{\sigma}{\chi}}}} \]
for every $\sigma, \tau, \chi \in \Ty$, satisfying the axioms
\[ \vdash \app{\app{K}{t}}{u} = t \quad\text{and}\quad
    \vdash \app{\app{\app{S}{f}}{g}}{t} = \app{\app{f}{t}}{(\app{g}{t})} \]
for every appropriately typed $t, u, f, g$. It should be clear that the intended semantics of $K_{\sigma,\tau}$ and $S_{\sigma,\tau,\chi}$ in a combinatory model $\struct{A}$ are $K^\struct{A}_{\sigma,\tau}$ and $S^\struct{A}_{\sigma,\tau,\chi}$, respectively.
\end{rem}

\begin{rem} \label{rem:comb-model-KS-uniq}
As in the case for environment models (see Remark~\ref{rem:env-model-int-uniq}), the elements $K$ and $S$ in Definition~\ref{def:comb-model} are uniquely determined by their defining equations (\ref{eq:comb-model-KS}). Thus, every extensional applicative structure $\struct{A}$ can be equipped with at most one choice of $K$ and $S$ to make it a combinatory model. In this case, we may say that $\struct{A}$ \textit{is} a combinatory model.
\end{rem}

\begin{prop} \label{prop:env-comb-mod-equiv}
Suppose $\struct{A}$ is an extensional applicative structure. Then $\struct{A}$ is an environment model if and only if it is a combinatory model.
\begin{proof}
%TODO references for the typed case?
The result follows easily from the syntactic equivalence between simply typed $\lambda$-calculus and typed combinatory logic. For example, given an environment model $\struct{A}$, we can define combinators
\begin{align*}
K_{\sigma,\tau} &= \sint{\lamv{x}{\sigma}{\lamv{y}{\tau}{x}}}{\empenv} \\
S_{\sigma,\tau,\chi} &= \sint{\lamv{f}{\functy{\sigma}{\functy{\tau}{\chi}}}{\lamv{g}{\functy{\sigma}{\tau}}{\lamv{x}{\sigma}{\app{\app{f}{x}}{(\app{g}{x})}}}}}{\empenv}.
\end{align*}
In the other direction, to define $\sint{t}{\rho}$ in a combinatory model, we translate the $\lambda$-closure of $t$ to a combinatory term, take its semantics, and apply the resulting function to the values in the environment $\rho$. Details on the translation to combinators for the untyped case can be found in \cite{barendregt:1984, hindley-seldin:1986}.
\end{proof}
\end{prop}

%TODO confirm this claim
By Proposition~\ref{prop:env-comb-mod-equiv}, we may simply speak of a \textit{model} of $\lambda$-calculus, and use either the environment or the combinatory model formulation to prove that an extensional applicative structure is a model. These kinds of models are also called \emph{Henkin models} in the literature \cite{DBLP:books/el/leeuwen90/Mitchell90}.

\subsection{Basic model theory}
%TODO better title?
We now establish some basic results about Henkin models. The most important are the Soundness Theorem \ref{thm:soundness} and the existence of a term model (Definition~\ref{def:term-model}) which satisfies precisely those equations that are provable in the syntax (Proposition~\ref{prop:term-model-satisfaction}).

\begin{lem} \label{lem:int-different-envs}
Let $\typedtm{t}{\sigma}$. Then for all environments $\typedenv{\rho, \rho'}{\FV{t}}$, we have
\[ (\forall x \in \FV{t}. \rho(x) = \rho'(x)) \quad\text{implies}\quad
    \sint{t}{\rho} = \sint{t}{\rho'}. \]
\begin{proof}
By induction on $t$:
\begin{items}
\item Case $x$:
\[ \sint{x}{\rho} = \rho(x) = \rho'(x) = \sint{x}{\rho'} \]
\item Case $\app{t}{u}$:
\[ \sint{\app{t}{u}}{\rho} = \sint{t}{\rho} \sint{u}{\rho}
    = \sint{t}{\rho'} \sint{u}{\rho'} = \sint{\app{t}{u}}{\rho'} \]
\item Case $\lamv{x}{\sigma}{t}$:
\[ \sint{\lamv{x}{\sigma}{t}}{\rho} a = \sint{t}{\updenv{\rho}{x}{a}}
    = \sint{t}{\updenv{\rho'}{x}{a}} = \sint{\lamv{x}{\sigma}{t}}{\rho'} a \]
In order to apply the induction hypothesis, we have used that $\updenv{\rho}{x}{a}(y) = \updenv{\rho'}{x}{a}(y)$ for all $y \in \FV{t} \cup \singset[x]$. \qedhere
\end{items}
\end{proof}
\end{lem}

\begin{defn}
If $\gamma$ is a substitution and $\rho$ is an environment such that $\typedenv{\rho}{\FV{\gamma}}$, then the environment $\typedenv{\sintsub{\gamma}{\rho}}{\dom{\gamma} \cup \dom{\rho}}$ is defined by
\[ \sintsub{\gamma}{\rho}(x) = \begin{cases}
                               \sint{\gamma(x)}{\rho} & \text{if } x \in \dom{\gamma} \\
                               \rho(x) & \text{if } x \in \dom{\rho} \setminus \dom{\gamma}
                               \end{cases}. \]
\end{defn}

\begin{lem}[Substitution lemma] \label{lem:sub-lemma}
Let $\struct{A}$ be a model. Then for all $\typedtm{t}{\sigma}$, $\gamma \in \Sub$, and $\typedenv{\rho}{(\FV{t} \setminus \dom{\gamma}) \cup \FV{\gamma}}$, we have 
\[ \sint{\subst{t}{\gamma}}{\rho} = \sint{t}{\sintsub{\gamma}{\rho}}. \]
\begin{proof}
By induction on the term $t$:
\begin{items}
\item Case $x$: if $x \in \dom{\gamma}$, then we have
\[ \sint{\subst{x}{\gamma}}{\rho} = \sint{\gamma(x)}{\rho}
    = \sintsub{\gamma}{\rho}(x) = \sint{x}{\sintsub{\gamma}{\rho}}. \]
If $x \notin \dom{\gamma}$, then
\[ \sint{\subst{x}{\gamma}}{\rho} = \sint{x}{\rho} = \rho(x)
    = \sintsub{\gamma}{\rho}(x) = \sint{x}{\sintsub{\gamma}{\rho}}. \]
\item Case $\app{t}{u}$:
\[ \sint{\subst{(\app{t}{u})}{\gamma}}{\rho}
    = \sint{\app{\subst{t}{\gamma}}{\subst{u}{\gamma}}}{\rho}
    = \sint{\subst{t}{\gamma}}{\rho} \sint{\subst{u}{\gamma}}{\rho}
    = \sint{t}{\sint{\gamma}{\rho}} \sint{u}{\sint{\gamma}{\rho}}
    = \sint{\app{t}{u}}{\sint{\gamma}{\rho}} \]

\item Case $\lamv{x}{\sigma}{t}$: Extensionality implies that it suffices to show
\[ \sint{\subst{(\lamv{x}{\sigma}{t})}{\gamma}}{\rho} a
    = \sint{\lamv{x}{\sigma}{t}}{\sintsub{\gamma}{\rho}} a \]
for all $a \in \scomp{\struct{A}}{\sigma}$.
\begin{align*}
\sint{\subst{(\lamv{x}{\sigma}{t})}{\gamma}}{\rho} a
    &= \sint{\lamv{x}{\sigma}{\subst{t}{(\updsub{\gamma}{x}{x})}}}{\rho} a
     = \sint{\subst{t}{(\updsub{\gamma}{x}{x})}}{\updenv{\rho}{x}{a}} \\
    &= \sint{t}{\sint{\updsub{\gamma}{x}{x}}{\updenv{\rho}{x}{a}}}
     = \sint{t}{\updenv{\sint{\gamma}{\rho}}{x}{a}}
     = \sint{\lamv{x}{\sigma}{t}}{\sint{\gamma}{\rho}} a.
\end{align*}
In order to apply the induction hypothesis, we have used that %TODO show this?
\[ \typedenv{\updenv{\rho}{x}{a}}{(\FV{t} \setminus \dom{\updsub{\gamma}{x}{x}}) \cup \FV{\updsub{\gamma}{x}{x}}}. \] 

In the second line, we have used that $\sintsub{\updsub{\gamma}{x}{x}}{\updenv{\rho}{x}{a}} = \updenv{\sintsub{\gamma}{\rho}}{x}{a}$. This is true since
\begin{align*}
\sintsub{\updsub{\gamma}{x}{x}}{\updenv{\rho}{x}{a}}(x)
    &= \sint{\updsub{\gamma}{x}{x}(x)}{\updenv{\rho}{x}{a}}
     = \sint{x}{\updenv{\rho}{x}{a}} \\
    &= \updenv{\rho}{x}{a}(x) = a = \updenv{\sintsub{\gamma}{\rho}}{x}{a}(x),
\end{align*}
\begin{align*}
\sintsub{\updsub{\gamma}{x}{x}}{\updenv{\rho}{x}{a}}(y)
    &= \updenv{\rho}{x}{a}(y) = \rho(y)
     = \sintsub{\gamma}{\rho}(y)
     = \updenv{\sintsub{\gamma}{\rho}}{x}{a}(y)
\end{align*}
if $y \notin \dom{\gamma} \cup \singset[x] = \dom{\updsub{\gamma}{x}{x}}$, and
\begin{align*}
\sintsub{\updsub{\gamma}{x}{x}}{\updenv{\rho}{x}{a}}(y)
    &= \sint{\updsub{\gamma}{x}{x}(y)}{\updenv{\rho}{x}{a}}
     = \sint{\gamma(y)}{\updenv{\rho}{x}{a}} \\
    &= \sint{\gamma(y)}{\rho}
     = \sintsub{\gamma}{\rho}(y)
     = \updenv{\sintsub{\gamma}{\rho}}{x}{a}(y)
\end{align*}
if $y \in \dom{\gamma} \setminus \singset[x]$. In the last case, the equality $\sint{\gamma(y)}{\updenv{\rho}{x}{a}} = \sint{\gamma(y)}{\rho}$ follows from Lemma~\ref{lem:int-different-envs} because $x \notin \FV{\gamma(y)}$. \qedhere
%TODO: make this more readable?
\end{items}
\end{proof}
\end{lem}

\begin{cor} \label{cor:sub-lemma-single-var}
Let $\typedtm{t}{\tau}$, $\typedvar{x}{\sigma}$, and $\typedtm{u}{\sigma}$. Then for all environments $\typedenv{\rho}{(\FV{t} \setminus \singset[x]) \cup \FV{u}}$, we have
\[ \sint{\subst{t}{\singsubv{x}{u}}}{\rho}
    = \sint{t}{\updenv{\rho}{x}{\sint{u}{\rho}}}. \]
\begin{proof}
Follows immediately from Lemma~\ref{lem:sub-lemma} taking $\gamma = \singsubv{x}{u}$, using the equation
\[ \sintsub{\singsubv{x}{u}}{\rho} = \updenv{\rho}{x}{\sint{u}{\rho}}. \]
The latter holds since
\[ \sintsub{\singsubv{x}{u}}{\rho}(x) = \sint{\singsubv{x}{u}(x)}{\rho}
    = \sint{u}{\rho} = \updenv{\rho}{x}{\sint{u}{\rho}}(x) \]
and
\[ \sintsub{\singsubv{x}{u}}{\rho}(y) = \rho(y) = \updenv{\rho}{x}{\sint{u}{\rho}}(y) \]
for $y \in \dom{\rho} \setminus \singset[x]$.
\end{proof}
\end{cor}

\begin{defn}[Satisfaction] \label{def:satisfaction}
Let $\struct{A}$ be a model.
\begin{enum}
\item Suppose $\typedtm{t, u}{\sigma}$, and $\typedenv{\rho}{\FV{t} \cup \FV{u}}$. We say that the equation $\conv{t}{u}{\sigma}$ \emph{holds at $\rho$ in $\struct{A}$}, written as $\modsat{\struct{A}, \rho}{t}{u}{\sigma}$, if $\sint[\struct{A}]{t}{\rho} = \sint[\struct{A}]{u}{\rho}$.

\item The model $\struct{A}$ \emph{satisfies} the equation $\conv{t}{u}{\sigma}$, written as $\modsat{\struct{A}}{t}{u}{\sigma}$, if $\modsat{\struct{A}, \rho}{t}{u}{\sigma}$ for all $\typedenv{\rho}{\FV{t} \cup \FV{u}}$.
\end{enum}
\end{defn}

\begin{thm}[Soundness] \label{thm:soundness}
Let $\struct{A}$ be a model. Then
\[ \conv{t}{u}{\sigma} \quad\text{implies}\quad \modsat{\struct{A}}{t}{u}{\sigma} \]
for all $\typedtm{t, u}{\sigma}$.
\begin{proof}
By induction on the derivation of $\conv{t}{u}{\sigma}$. The cases \textsc{refl}, \textsc{trans}, and \textsc{sym} follow from the fact that $\modsat{\struct{A}}{-}{-}{\sigma}$ is an equivalence relation. For the cases \textsc{cong-app} and \textsc{cong-lam} we use the compositionality of the interpretation of $\lambda$-terms: the meaning of compound terms is defined in terms of the meanings of the subterms. Hence, if the latter are equal at all environments, so are the former.

For \textsc{beta}, we use Corollary~\ref{cor:sub-lemma-single-var}. Let $\typedenv{\rho}{(\FV{t} \setminus \singset[x]) \cup \FV{u}}$. Then
\[ \sint{\app{(\lamv{x}{\sigma}{t})}{u}}{\rho} = \sint{\lamv{x}{\sigma}{t}}{\rho} \sint{u}{\rho} = \sint{t}{\updenv{\rho}{x}{\sint{u}{\rho}}} = \sint{\subst{t}{\singsubv{x}{u}}}{\rho}. \]

Finally, \textsc{eta} follows from the extensionality of $\struct{A}$ and Lemma~\ref{lem:int-different-envs}:
\[ \sint{\lamv{x}{\sigma}{\app{t}{x}}}{\rho} a = \sint{\app{t}{x}}{\updenv{\rho}{x}{a}} = \sint{t}{\updenv{\rho}{x}{a}} \sint{x}{\updenv{\rho}{x}{a}} = \sint{t}{\rho} a. \]
Note that to apply Lemma~\ref{lem:int-different-envs}, we use the assumption that $x \notin \FV{t}$.
\end{proof}
\end{thm}

\begin{defn}[Term model] \label{def:term-model}
The \emph{term model} $\tmmod$ is defined as follows. The set $\tmmod_\sigma$ is the set of terms of type $\sigma$ quotiented by convertibility:
\[ \tmmod_\sigma = \Tm[\sigma] / \convrel{\sigma}. \]
That is, $\tmmod_\sigma = \setof{[t]}{\typedtm{t}{\sigma}}$,
where
\[ [t] = \setof{\typedtm{u}{\sigma}}{\conv{t}{u}{\sigma}} \]
denotes the $\beta\eta$-equivalence class of $t$. The application map is given by
\[ [t] [u] = [\app{t}{u}] \]
for $\typedtm{t}{\functy{\sigma}{\tau}}$ and $\typedtm{u}{\sigma}$.

By definition, an environment for $\tmmod$ is a mapping from variables to equivalences classes of terms. If $\gamma$ is a substitution, then $[\gamma]$ denotes the environment given by $[\gamma](x) = [\gamma(x)]$ for all $x \in \dom{\gamma}$. Every $\rho \in \Env[][\tmmod]$ can be written as $[\gamma]$ for some substitution $\gamma$ by picking representatives in $\rho(x)$ for each $x \in \dom{\rho}$. That said, we define the interpretation into $\tmmod$ as
\[ \sint[\tmmod]{t}{[\gamma]} = [\subst{t}{\gamma}]. \]
\end{defn}

\begin{prop} \label{prop:tm-model}
$\tmmod$ is a well-defined model.
\begin{proof}
To show well-definedness, we need to check that our definitions are independent of the choices of representatives. For the application map, suppose we pick representatives $t', u'$ with $\conv{t}{t'}{\functy{\sigma}{\tau}}$ and $\conv{u}{u'}{\sigma}$. By rule \textsc{cong-app}, we have $\conv{\app{t}{u}}{\app{t'}{u'}}{\tau}$, hence $[\app{t}{u}] = [\app{t'}{u'}]$.

Now suppose $\gamma, \gamma'$ are substitutions such that $[\gamma] = [\gamma']$. Then we have $[\gamma(x)] = [\gamma'(x)]$, hence $\conv{\gamma(x)}{\gamma'(x)}{\sigma}$ for all $\typedvar{x}{\sigma} \in \dom{\gamma}$. By Lemma~\ref{lem:sub-conv}, this implies $\conv{\subst{t}{\gamma}}{\subst{t}{\gamma'}}{\sigma}$, and thus $[\subst{t}{\gamma}] = [\subst{t}{\gamma'}]$. This shows the well-definedness of the interpretation.

To show that the applicative structure $\tmmod$ is extensional, let $[t], [t'] \in \tmmod_{\functy{\sigma}{\tau}}$ and assume that $[t][u] = [t'][u]$ for all $\typedtm{u}{\sigma}$. Let $\typedvar{x}{\sigma}$ be a variable such that $x \notin \FV{t} \cup \FV{u}$. Then we have $[\app{t}{x}] = [t][x] = [t'][x] = [\app{t'}{x}]$ by assumption, hence $\conv{\app{t}{x}}{\app{t'}{x}}{\tau}$. By \textsc{cong-lam}, we derive $\conv{\lamv{x}{\sigma}{\app{t}{x}}}{\lamv{x}{\sigma}{\app{t'}{x}}}{\functy{\sigma}{\tau}}$. Applying \textsc{eta} twice, we get $\conv{t}{t'}{\functy{\sigma}{\tau}}$. Thus, $[t] = [t']$ as desired.

It remains to check that the interpretation satisfies the equations in Definition~\ref{def:env-model}:
\begin{align*}
\sint{x}{[\gamma]} &= [\subst{x}{\gamma}] = [\gamma(x)] = [\gamma](x)
    \quad \text{(note that $x \in \dom{[\gamma]} = \dom{\gamma}$)} \\
\sint{\app{t}{u}}{[\gamma]}
    &= [\subst{(\app{t}{u})}{\gamma}]
     = [\app{\subst{t}{\gamma}}{\subst{u}{\gamma}}]
     = [\subst{t}{\gamma}] [\subst{u}{\gamma}]
     = \sint{t}{[\gamma]} \sint{u}{[\gamma]} \\
\sint{\lamv{x}{\sigma}{t}}{[\gamma]} [u]
    &= [\subst{(\lamv{x}{\sigma}{t})}{\gamma}] [u]
     = [\app{(\lamv{x}{\sigma}{\subst{t}{(\updsub{\gamma}{x}{x})}})}{u}] \\
    &= [\subst{t}{\updsub{(\updsub{\gamma}{x}{x})}{x}{u}}]
     = [\subst{t}{(\updsub{\gamma}{x}{u})}]
     = \sint{t}{[\updsub{\gamma}{x}{u}]}
     = \sint{t}{\updenv{[\gamma]}{x}{[u]}}
\end{align*}
Note that in the last case, $\subst{\subst{t}{(\updsub{\gamma}{x}{x})}}{\singsubv{x}{u}} = \subst{t}{(\updsub{\gamma}{x}{u})}$ only because $x \notin \FV{\gamma}$.
\end{proof}
\end{prop}

%TODO can we call this completeness?
\begin{thm} \label{prop:term-model-satisfaction}
For all $t$ and $u$, $\conv{t}{u}{\sigma}$ if and only if $\modsat{\tmmod}{t}{u}{\sigma}$. 
\begin{proof}
The only if part is a corollary of Theorem~\ref{thm:soundness} and Proposition~\ref{prop:tm-model}. If $\modsat{\tmmod}{t}{u}{\sigma}$, then $\sint[\tmmod]{t}{[\idsub]} = \sint[\tmmod]{u}{[\idsub]}$, where $\idsub$ is the identity substitution on $\FV{t}$. Then
\[ [t] = [\subst{t}{(\idsub)}] = \sint{t}{[\idsub]} = \sint{u}{[\idsub]}
    = [\subst{u}{(\idsub)}] = [u], \]
and hence $\conv{t}{u}{\sigma}$.
\end{proof}
\end{thm}

\begin{comment}
NOTE: product structure is only a model of all the components of the two factors are nonempty.

\begin{def}[Product of models]
Let $\struct{A}$ and $\struct{B}$ be models. The \emph{product model} $\cprod{\struct{A}}{\struct{B}}$ is defined as follows:
\begin{items}
\item $(\cprod{\struct{A}}{\struct{B}})_\sigma = \cartprod{\scomp{\struct{A}}{\sigma}}{\scomp{\struct{B}}{\sigma}}$;
\item $\sapp[\sigma,\tau][\cprod{\struct{A}}{\struct{B}}]((f, g), (x, y))
        = (\sapp[\sigma,\tau][\struct{A}](f, x), \sapp[\sigma,\tau][\struct{B}](g, y))$;
\item $\sint[\cprod{\struct{A}}{\struct{B}}]{t}{\rho}
        = (\sint[\struct{A}]{t}{\rho}, \sint[\struct{B}]{t}{\rho})$.
\end{items}
\end{def}

\begin{prop}
If $\scomp{\struct{A}}{\sigma}$ and $\scomp{\struct{B}}{\sigma}$ are nonempty for all $\sigma \in \Ty$, then $\cprod{\struct{A}}{\struct{B}}$ is a model.
\begin{proof}
Extensionality: if $(f, g)(x, y) = (f', g')(x, y)$ for all $x, y$, then $(fx, gy) = (f'x, g'y)$ for all $x, y$, i.e. $fx = f'x$ and $gy = g'y$ for all
\end{proof}
\end{prop}
\end{comment}

\subsection{Homomorphisms}

When studying the properties of a class of mathematical structures, it is worthwhile to consider structure-preserving mappings between the structures. These mappings are usually called \textit{homomorphisms}. For example, functions between groups preserving the group operation are called \textit{group homomorphisms} and they are an essential concept in group theory.

We can apply the same idea to the notions of model defined in Section~\ref{sec:models}. Although we do not make use of homomorphisms between Henkin models in this thesis, we believe it is worth mentioning them to make it easier to relate this theory to the categorical semantics to be introduced in Chapter~\ref{chap:stlc-cat}.

\begin{defn}[Homomorphism of applicative structures]
Let $\struct{A}$ and $\struct{B}$ be applicative structures. A \emph{homomorphism of applicative structures} $h : \struct{A} \to \struct{B}$ is a family of maps $(h_\sigma : \scomp{\struct{A}}{\sigma} \to \scomp{\struct{B}}{\sigma})_{\sigma \in \Ty}$ such that
\[ h_\tau(\sapp[\sigma,\tau][\struct{A}](f, x)) =
    \sapp[\sigma,\tau][\struct{B}](h_{\functy{\sigma}{\tau}}(f), h_\sigma(x)) \]
for all $\sigma, \tau \in \Ty$, $f \in \scomp{\struct{A}}{\functy{\sigma}{\tau}}$, and $x \in \scomp{\struct{A}}{\sigma}$.
\end{defn}

\begin{notn}
Let $h : \struct{A} \to \struct{B}$ be a homomorphism of applicative structures.
\begin{items}
\item If $x \in \scomp{\struct{A}}{\sigma}$, then we may write $h(x)$ for $h_\sigma(x)$, omitting the type $\sigma$.
\item If $\rho \in \Env[][\struct{A}]$ is an environment for $\struct{A}$, then $\mapenv{h}{\rho} \in \Env[][\struct{B}]$ is the environment defined by $(\mapenv{h}{\rho})(x) = h_\sigma(\rho(x))$ for all $x \in \dom{\rho} \cap \Var[\sigma]$.
\end{items}
\end{notn}

\begin{defn}[Homomorphism of models]
Let $\struct{A}$ and $\struct{B}$ be models. We say that a homomorphism of applicative structures $h : \struct{A} \to \struct{B}$ is a \emph{homomorphism of models} if
\[ h_\sigma(\sint[\struct{A}]{t}{\rho}) = \sint[\struct{B}]{t}{\mapenv{h}{\rho}} \]
for all $\typedtm{t}{\sigma}$ and $\typedenv{\rho}{\FV{t}}$.
\end{defn}

Equivalently, using the combinatory model formulation, a homomorphism of models $h : \struct{A} \to \struct{B}$ can be defined as a homomorphism of applicative structures such that
\[ h(K^\struct{A}_{\sigma,\tau}) = K^\struct{B}_{\sigma,\tau} \quad\text{and}\quad
    h(S^\struct{A}_{\sigma,\tau,\chi}) = S^\struct{B}_{\sigma,\tau,\chi}. \]

%TODO examples?

%TODO what else to say about this category?
For every model $\struct{A}$, there is an identity homomorphism $\id[\struct{A}] : \struct{A} \to \struct{A}$ which is the identity function at every type. Furthermore, homomorphisms $f : \struct{A} \to \struct{B}$ and $g : \struct{B} \to \struct{C}$ can be composed pointwise to yield a homomorphism $gf : \struct{A} \to \struct{C}$. Thus, Henkin models and homomorphisms of models form a category.

\begin{comment}
How useful is this example?
\begin{ex}
Let $X = (X_\beta)_{\beta \in \Basetypes}$ and $Y = (Y_\beta)_{\beta \in \Basetypes}$ be to families of sets indexed by base types, and suppose we have a family of bijections $(p_\beta : X_\beta \to Y_\beta)_{\beta \in \Basetypes}$ between the corresponding sets. This gives rise to a homomorphism $\stdmod{p} : \stdmod{X} \to \stdmod{Y}$ from the standard model over $X$ to the standard model over Y. The maps $\stdmod{p}_\sigma$ are defined by recursion on types:
\begin{align*}
\stdmod{p}_\beta(x) &= p_\beta(x) \\
\stdmod{p}_{\functy{\sigma}{\tau}}(f) &= \stdmod{p}_\tau \circ f \circ (\stdmod{p}_\sigma)^{-1}
\end{align*}
\end{ex}
\end{comment}


\chapter{Normalization for the simply typed \texorpdfstring{$\lambda$}{lambda}-calculus} \label{chap:norm-stlc}
Normalization refers to the process of finding a normal form for a given term. Traditionally, normalization and the notion of normal form have been defined in the context of \textit{term rewrite systems}. A term rewrite system is given by a set of terms and a set of rewrite rules, also called reduction rules. In this context, a normal form is a term in which no reductions are possible, and normalizing a term consists of applying the rewrite rules repeatedly until a normal form is reached.

There is a computational reading of this process: terms represent programs, and applying rewrite rules corresponds to running a program. The execution is finished when no more reductions are possible, in which case the final term represents the value of the program. Thus, normal forms correspond to values that a program can output, and normalization corresponds to execution of programs. In particular, in the (simply typed) $\lambda$-calculus, the most important rewrite rule is $\beta$-reduction (the rule \textsc{beta} applied from left to right).

%TODO too many i.e.-s
An interesting question to ask about a term rewrite system is whether every term can be normalized, i.e. whether every term can be reduced to a normal form. A rewrite system in which this is possible is said to satisfy the \textit{weak normalization property}. There is a stronger property known as \textit{strong normalization} which states that every possible reduction path is finite, i.e. no infinite reductions are possible.

In a programming language that satisfies the strong normalization property, any sequence of execution steps will eventually lead to a final result, i.e. no program can run forever. In programming language terminology, we say that the programming language is \textit{terminating}: all programs are guaranteed to terminate and return a result. In contrast, if a language only satisfies weak normalization, then we only know that there is \textit{some} execution path that leads to a result, but not necessarily that each of them does. However, if evaluation in the language is deterministic (that is, there is at most one possible execution step at each stage), then this property already implies that the language is terminating. Thus, weak normalization can still be a useful property.

In a reduction free setting, we cannot formalize the strong normalization property. However, the weak normalization property still makes sense if reduction is replaced by convertibility. The property then states that every term has a normal form (see Definition~\ref{def:has-normal-form}). Fiore \cite{fiore:2002:ppdp, fiore:2022:mscs} calls this the \textit{extensional normalization problem}. Furthermore, it is convenient to characterize normal forms syntactically without reference to any reduction rules. This was done in Definition~\ref{def:normal-forms} for long $\beta\eta$-normal forms.

For practical purposes, the weak normalization property is not sufficient. Instead, we need an effective procedure for determining the normal form of a term, that is, a function that maps terms to their normal forms. Fiore \cite{fiore:2002:ppdp, fiore:2022:mscs} calls this the \textit{intensional normalization problem}. Such a normalization function can then be used for example to decide the convertibility of two terms (Corollary~\ref{cor:conv-dec}). This is interesting from a logical point of view: the simply typed $\lambda$-calculus can be viewed as an equational theory, and thus the validity of formulas in this theory is decidable.

%TODO who actually invented the method, Gödel or Tait?
In Section~\ref{sec:weak-norm}, we prove weak normalization for the simply typed $\lambda$-calculus using the method of \textit{computability predicates} invented by Tait \cite{tait:1967:jsl} and Gödel \cite{godel1958}. We then provide an analysis of this proof in terms of the notion of \textit{logical relation} introduced by Statman \cite{statman:1985:ic}, which is a generalization of Tait's predicates.

In Section~\ref{sec:nbe}, we consider \textit{normalization by evaluation}, a technique invented by Berger and Schwichtenberg \cite{DBLP:conf/lics/BergerS91}, providing a solution for the intensional normalization problem. Normalization by evaluation lies on the observation that normalization is essentially evaluation for open terms. However, instead of an operational approach, it employs denotational semantics for evaluation, thereby bypassing reduction. Hence, normalization of a term may be implemented by evaluating it in a context where the free variables are mapped to themselves, and then turning the semantic object back into a normal form.

%TODO what to say, where to put it?
There is a close analogy between Tait's proof and normalization by evaluation. In fact, Berger showed \cite{berger:1993:tlca, berger:2006:sl} that a version of normalization by evaluation can be obtained from a Tait-like proof of strong normalization using a mechanized procedure known as \textit{program extraction}. In Section~\ref{sec:nbe-alg}, we highlight the correspondences between the two proofs.

\section{Weak normalization} \label{sec:weak-norm}

In this section, we prove the following theorem:

\begin{thm}[Weak normalization] \label{thm:weak-norm}
All terms $\typedtm{t}{\sigma}$ have a normal form.
\end{thm}

A naive attempt of a proof of this theorem would try to prove the statement directly by induction on terms. However, this approach gets stuck at the application case: we cannot prove that $\app{t}{u}$ has a normal form if $t$ and $u$ have one. A counterexample in the untyped $\lambda$-calculus is the term $(\lambda x.\; x\; x)\; (\lambda x.\; x\; x)$.

%TODO explain what Gödel wanted to show using computable functionals?
In Section~\ref{sec:Tait-proof}, we present a proof of Theorem~\ref{thm:weak-norm} based on the idea of a \textit{convertible term} by Tait \cite{tait:1967:jsl}, who used it to prove normalization of terms in Gödel's system T \cite{godel1958}. Tait's method itself was inspired by Gödel's notion of a \textit{computable/reckonable functional} (German: \textit{berechenbare Funktion}) \cite{godel1958}.

%TODO reference for induction loading?
The essential idea of the proof is that we define when a term is convertible by induction on the type of the term (Definition~\ref{def:conv-pred}). A term of base type is convertible iff it has a normal form; a term of function type is convertible iff it maps convertible inputs to convertible outputs. Now proving that every term is convertible is straightforward using induction on terms (Lemma~\ref{lem:fundamental-lem-norm}), solving the issue of the naive attempt. Furthermore, every convertible term has a normal form (Lemma~\ref{lem:computability-normal-forms}); hence, the theorem follows. On a methodological level, the proof considers a stronger property (convertibility) in order to provide a stronger induction hypothesis, thus making the induction go through. This method is sometimes referred to as \textit{induction loading}.

%TODO add references to more applications? (e.g. canonicity, parametricity, normalization/termination, memory safety)
In Section~\ref{sec:logical-relations}, we introduce the notion of \textit{logical predicate}, which is a special case of the notion of \textit{logical relation} introduced by Statman \cite{statman:1985:ic}. Precursors of logical relations and extensions thereof have been widely used in the literature to prove (among other things) definability results about the simply typed $\lambda$-calculus \cite{plotkin:1980, statman:1985:ic, jung:1993:tlca, DBLP:conf/tlca/FioreS99}. The point of logical predicates is that they generalize the core idea behind Tait's convertibility predicate, namely, that the property we wish to prove should be defined by induction on the type structure. We also prove a generalization of Lemma~\ref{lem:fundamental-lem-norm} (Theorem~\ref{thm:fundamental-thm-log-rel}), dubbed the \textit{Fundamental Theorem of Logical Relations} by Statman \cite{statman:1985:ic}.

%TODO abstract away sounds weird perhaps
Furthermore, we introduce the notion of \textit{bilogical predicate} (Definition~\ref{def:bilog-rel}) and prove a corresponding fundamental theorem (Theorem~\ref{thm:fundamental-thm-bilog-rel}). The fundamental theorem abstracts away the ingredients of the proof in Section~\ref{sec:Tait-proof}. This development was inspired by Fiore's Basic Lemma \cite{fiore:2002:ppdp, fiore:2022:mscs} and it is essentially an instantiation of his result phrased in a simpler framework.

Finally, in Section~\ref{sec:log-pred-proof}, we give a very short proof of Theorem~\ref{thm:weak-norm} employing the results of Section~\ref{sec:logical-relations}. This proof is essentially a rephrasing of the proof in Section~\ref{sec:Tait-proof}.

\subsection{``Syntactic" proof} \label{sec:Tait-proof}
%TODO better title

\begin{defn}
Let $T \subs \Tm[\sigma]$. We define
\[ \convset{T} = \setof{\typedtm{t}{\sigma}}{\exists u \in T. \conv{t}{u}{\sigma}}. \]
\end{defn}

That is, $\convset{T}$ is the set of terms of type $\sigma$ that are convertible to some term in $T$. Note that $\convset{\Nf[\sigma]}$ is the set of terms of type $\sigma$ that have a normal form.

\begin{defn}[Convertibility predicate] \label{def:conv-pred}
We define a family of predicates $R_\sigma \subs \Tm[\sigma]$ on $\lambda$-terms by recursion on types:
\begin{align*}
R_\beta &= \convset{\Nf[\beta]} \quad (\beta \in \Basetypes) \\
R_{\functy{\sigma}{\tau}} &= \setof{\typedtm{t}{\functy{\sigma}{\tau}}}
    {\forall u \in R_\sigma.\ \app{t}{u} \in R_\tau}
\end{align*}
\end{defn}

\begin{lem} \label{lem:conversion-computability}
If $\conv{t}{t'}{\sigma}$ and $t' \in R_\sigma$, then $t \in R_\sigma$.
\begin{proof}
By induction on $\sigma$:
\begin{items}
\item Case $\beta \in \Basetypes$:
Suppose $t' \in R_\beta$, i.e. there is a normal form $\typedtm{n}{\beta}$ such that $\conv{t'}{n}{\beta}$. Since $\conv{t}{t'}{\beta}$, transitivity implies $\conv{t}{n}{\beta}$. Hence, $t \in \convset{\Nf[\beta]} = R_\beta$.

\item Case $\functy{\sigma}{\tau}$:
Suppose $t' \in R_{\functy{\sigma}{\tau}}$, and let $u \in R_\sigma$. Then $\app{t'}{u} \in R_\tau$. Since $\conv{t}{t'}{\functy{\sigma}{\tau}}$, we derive $\conv{\app{t}{u}}{\app{t'}{u}}{\tau}$. Thus, we can apply the induction hypothesis to conclude that $\app{t}{u} \in R_\tau$. Hence, $t \in R_{\functy{\sigma}{\tau}}$. \qedhere
\end{items}
\end{proof}
\end{lem}

\begin{lem}[Fundamental Lemma] \label{lem:fundamental-lem-norm}
Let $\typedtm{t}{\sigma}$ and $\gamma \in \Sub$ such that $\FV{t} \subs \dom{\gamma}$. If $\gamma(x) \in R_\tau$ for all $x \in \FV[\tau]{t}$, then $\subst{t}{\gamma} \in R_\sigma$.
\begin{proof}
By induction on $t$:
\begin{items}
\item Case $x$:
$\subst{x}{\gamma} = \gamma(x) \in R_\tau$ by assumption.
\item Case $\app{t}{u}$:
We have $\subst{t}{\gamma} \in R_{\functy{\sigma}{\tau}}$ and $\subst{u}{\gamma} \in R_\sigma$ by the induction hypotheses. Thus, $\subst{(\app{t}{u})}{\gamma} = \app{\subst{t}{\gamma}}{\subst{u}{\gamma}} \in R_\tau$ by the definition or $R_{\functy{\sigma}{\tau}}$.
\item Case $\lamv{x}{\sigma}{t}$:
We need to show that $\subst{(\lamv{x}{\sigma}{t})}{\gamma} = \lamv{x}{\sigma}{\subst{t}{(\updsub{\gamma}{x}{x})}} \in R_{\functy{\sigma}{\tau}}$. Let $u \in R_\sigma$. Then $\updsub{\gamma}{x}{u}$ satisfies the conditions of the lemma, and hence we can apply the induction hypothesis to obtain $\subst{t}{(\updsub{\gamma}{x}{u})} \in R_\tau$. Since $\conv{\app{(\lamv{x}{\sigma}{\subst{t}{(\updsub{\gamma}{x}{x})}})}{u}}{\subst{t}{(\updsub{\gamma}{x}{u})}}{\tau}$, we infer $\app{(\lamv{x}{\sigma}{\subst{t}{(\updsub{\gamma}{x}{x})}})}{u} \in R_\tau$ by Lemma~\ref{lem:conversion-computability}. \qedhere
\end{items}
\end{proof}
\end{lem}

\begin{lem} \label{lem:computability-normal-forms}
The following two statements hold for all types $\sigma$:
\begin{enum}
\item $R_\sigma \subs \convset{\Nf[\sigma]}$;
\item $\convset{\Ne[\sigma]} \subs R_\sigma$.
\end{enum}
\begin{proof}
We prove both statements simultaneously by induction on $\sigma$.
\begin{items}
\item Case $\beta \in \Basetypes$: (i) holds by definition. (ii) holds since $\Ne[\beta] \subs \Nf[\beta]$, and hence $\convset{\Ne[\beta]} \subs \convset{\Nf[\beta]}$.

\item Case $\functy{\sigma}{\tau}$: For (i), let $t \in R_{\functy{\sigma}{\tau}}$ and let $\typedvar{x}{\sigma}$ such that $x \notin \FV{t}$. Since $\conv{x}{x}{\sigma}$ and $x \in \Ne[\sigma]$, we have $x \in \convset{\Ne[\sigma]}$. Thus, by the induction hypothesis for (ii), we get $x \in R_\sigma$. By the definition of $R_{\functy{\sigma}{\tau}}$, this implies $\app{t}{x} \in R_\tau$. Applying the induction hypothesis for (i), we get that $\app{t}{x} \in \convset{\Nf[\tau]}$, and hence there is a normal form $\typedtm{n}{\tau}$ such that $\conv{\app{t}{x}}{n}{\tau}$. Using \textsc{cong-lam} and \textsc{eta}, we derive $\conv{t}{\lamv{x}{\sigma}{n}}{\functy{\sigma}{\tau}}$. Since $\lamv{x}{\sigma}{n}$ is a normal form, this means $t \in \convset{\Nf[\functy{\sigma}{\tau}]}$, as desired.

For (ii), suppose $t \in \convset{\Ne[\functy{\sigma}{\tau}]}$, i.e. there is a neutral term $\typedtm{m}{\functy{\sigma}{\tau}}$ such that $\conv{t}{m}{\functy{\sigma}{\tau}}$. To prove $t \in R_{\functy{\sigma}{\tau}}$, we need to show that $\app{t}{u} \in R_\tau$ for all $u \in R_\sigma$. If $u \in R_\sigma$, then the induction hypothesis for (i) implies that there is a normal form $\typedtm{n}{\sigma}$ such that $\conv{u}{n}{\sigma}$. Using \textsc{cong-app}, we derive $\conv{\app{t}{u}}{\app{m}{n}}{\tau}$. Since $\app{m}{n}$ is a neutral term, this means $\app{t}{u} \in \convset{\Ne[\tau]}$. Then, by the induction hypothesis for (ii), we conclude $\app{t}{u} \in R_\tau$. \qedhere
\end{items}
\end{proof}
\end{lem}

\begin{proof}[Proof of Theorem~\ref{thm:weak-norm}]
Let $\idsub$ be the identity substitution on $\FV{t}$. By Lemma~\ref{lem:computability-normal-forms} (ii), $\idsub(x) = x \in R_\sigma$ for all $x \in \FV{t}$. Hence, $t = \subst{t}{(\idsub)} \in R_\sigma$ by Lemma~\ref{lem:fundamental-lem-norm}. By Lemma~\ref{lem:computability-normal-forms} (i), $t \in \convset{\Nf[\sigma]}$, which is what we wanted to prove.
\end{proof}

%TODO say something about how this subsection relates to Tait's original proof?

\subsection{Logical predicates} \label{sec:logical-relations}

%TODO: define n-ary logical relations? Prove the fundamental theorem for n-ary logical relations? This is necessary if we want to prove correctness of NbE using logical relations (it requires binary logical relations)

\begin{defn}[Logical predicate]
Let $\struct{A}$ be an applicative structure. A \emph{logical predicate $\struct{R}$ over $\struct{A}$} is a family of subsets $\scomp{\struct{R}}{\sigma} \subs \scomp{\struct{A}}{\sigma}$ indexed by types $\sigma$ such that for all $f \in \scomp{\struct{A}}{\functy{\sigma}{\tau}}$, we have
\begin{equation} \label{eq:log-pred-functy}
f \in \scomp{\struct{R}}{\functy{\sigma}{\tau}} \iff
    \forall x \in R_\sigma.\ f x \in R_\tau.
\end{equation}
\end{defn}

It is clear that (\ref{eq:log-pred-functy}) can be seen as the definition of the set $\scomp{\struct{R}}{\functy{\sigma}{\tau}}$. Thus, a logical predicate $\struct{R}$ over $\struct{A}$ is completely determined by the sets $\scomp{\struct{R}}{\beta}$ for $\beta \in \Basetypes$.

\begin{defn}
If $\struct{R}$ is a logical predicate over $\struct{A}$, then we say that an environment $\rho \in \Env[][\struct{A}]$ \emph{satisfies} the predicate $\struct{R}$ if $\rho(x) \in \scomp{\struct{R}}{\sigma}$ for all $x \in \dom{\rho} \cap \Var[\sigma]$.
\end{defn}

\begin{thm}[Fundamental Theorem of Logical Relations] \label{thm:fundamental-thm-log-rel}
Suppose $\struct{A}$ is a model and $\struct{R}$ is a logical predicate over $\struct{A}$. Then for every term $\typedtm{t}{\sigma}$ and environment $\rho \in \Env[\FV{t}][\struct{A}]$ satisfying $\struct{R}$, we have $\sint[\struct{A}]{t}{\rho} \in \scomp{\struct{R}}{\sigma}$.
\begin{proof}
By induction on $t$:
\begin{items}
\item Case $\typedvar{x}{\tau}$:
$\sint{x}{\rho} = \rho(x) \in \scomp{\struct{R}}{\tau}$ by assumption.
\item Case $\app{t}{u}$:
We have $\sint{t}{\rho} \in \scomp{\struct{R}}{\functy{\sigma}{\tau}}$ and $\sint{u}{\rho} \in \scomp{\struct{R}}{\sigma}$ by the induction hypotheses. Thus, $\sint{\app{t}{u}}{\rho} = \sint{t}{\rho} \sint{u}{\rho} \in \scomp{\struct{R}}{\tau}$ by the definition or $\scomp{\struct{R}}{\functy{\sigma}{\tau}}$.
\item Case $\lamv{x}{\sigma}{t}$:
Let $a \in \scomp{\struct{R}}{\sigma}$. Then $\updenv{\rho}{x}{a}$ satisfies $\struct{R}$, and hence we can apply the induction hypothesis to obtain $\sint{t}{\updenv{\rho}{x}{a}} \in \scomp{\struct{R}}{\tau}$. Thus, $\sint{\lamv{x}{\sigma}{t}}{\rho} a = \sint{t}{\updenv{\rho}{x}{a}} \in \scomp{\struct{R}}{\tau}$ for all $a \in \scomp{\struct{R}}{\sigma}$. Therefore, by definition, $\sint{\lamv{x}{\sigma}{t}}{\rho} \in \scomp{\struct{R}}{\functy{\sigma}{\tau}}$. \qedhere
\end{items}
\end{proof}
\end{thm}

%TODO say something about how logical predicates/relations relate to (sub)models?
\begin{comment}
\begin{defn}[Substructure]
Let $\struct{A}$ be an applicative structure. Then a \emph{substructure of $\struct{A}$} is an applicative structure $\struct{S}$ such that $\scomp{\struct{S}}{\sigma} \subs \scomp{\struct{A}}{\sigma}$ for all types $\sigma$, and such that the application maps
\[ \sapp[\sigma,\tau][\struct{S}] : \cartprod{\scomp{\struct{S}}{\functy{\sigma}{\tau}}}{\scomp{\struct{S}}{\sigma}} \to \scomp{\struct{S}}{\tau} \]
are the restrictions of
\[ \sapp[\sigma,\tau][\struct{A}] : \cartprod{\scomp{\struct{A}}{\functy{\sigma}{\tau}}}{\scomp{\struct{A}}{\sigma}} \to \scomp{\struct{A}}{\tau} \]
for all pairs of types $\sigma, \tau$. If $\struct{S}$ is a substructure of $\struct{A}$, we write $\struct{S} \substruct \struct{A}$.
\end{defn}

\begin{defn}[Submodel]
Let $\struct{A}$ be a model. Then a \emph{submodel of $\struct{A}$} is a substructure $\struct{S} \substruct \struct{A}$ such that $\struct{S}$ is a model and $\sint[\struct{S}]{t}{\rho} = \sint[\struct{A}]{t}{\rho}$ for all terms $\typedtm{t}{\sigma}$ and environments $\rho \in \Env[\FV{t}][\struct{S}]$.
\end{defn}

\begin{con} \label{con:log-pred-to-submodel}
Suppose $\struct{A}$ a model and $\struct{R}$ is a logical predicate over $\struct{A}$. By the definition of $\scomp{\struct{R}}{\functy{\sigma}{\tau}}$, the application maps of $\struct{A}$ restrict to maps $\sapp[\sigma,\tau][\struct{R}] : \cartprod{\scomp{\struct{R}}{\functy{\sigma}{\tau}}}{\scomp{\struct{R}}{\sigma}} \to \scomp{\struct{R}}{\tau}$. This defines a substructure $\struct{R} \substruct \struct{A}$.
\end{con}

\begin{lem} \label{lem:log-pred-to-submodel}
The substructure $\struct{R} \substruct \struct{A}$ in Construction \ref{con:log-pred-to-submodel} is a submodel.
\begin{proof}
\struct{R} is not extensional :((

An environment for the applicative structure $\struct{R}$ is precisely an environment for $\struct{A}$ that satisfies the logical predicate $\struct{R}$. Hence, by the fundamental theorem (Theorem~\ref{thm:fundamental-thm-log-rel}), we have $\sint[\struct{A}]{t}{\rho} \in \scomp{\struct{R}}{\sigma}$ for all $\typedtm{t}{\sigma}$ and $\rho \in \Env[\FV{t}][\struct{R}]$. This shows that $\struct{R}$ is a model such that $\sint[\struct{R}]{t}{\rho} = \sint[\struct{A}]{t}{\rho}$ for all $t$ and $\rho$.
\end{proof}
\end{lem}

\begin{lem} \label{lem:submodel-to-log-pred}

\end{lem}

\begin{prop}
Suppose $\struct{A}$ is a model. Then there is a bijection between logical predicates over $\struct{A}$ and submodels of $\struct{A}$.
\begin{proof}
Corollary of lemmas \ref{lem:log-pred-to-submodel} and \ref{lem:submodel-to-log-pred}.
\end{proof}
\end{prop}
\end{comment}

%TODO: Come up with a better name for bilogical predicates?

\begin{defn} \label{def:bilog-rel}
Suppose $\struct{A}$ is an applicative structure. A \emph{bilogical predicate over $\struct{A}$} consists of two families of subsets $\scomp{\struct{Q}}{\sigma}, \scomp{\struct{S}}{\sigma} \subs \scomp{\struct{A}}{\sigma}$ indexed by types $\sigma$, satisfying the following conditions:
\begin{enum}
\item $\scomp{\struct{Q}}{\beta} \subs \scomp{\struct{S}}{\beta}$ for all $\beta \in \Basetypes$;
\item $f \in \scomp{\struct{Q}}{\functy{\sigma}{\tau}}$ and $x \in \scomp{\struct{S}}{\sigma}$ implies $f x \in \scomp{\struct{Q}}{\tau}$;
\item if $f x \in \scomp{\struct{S}}{\tau}$ for all $x \in \scomp{\struct{Q}}{\sigma}$, then $f \in \scomp{\struct{S}}{\functy{\sigma}{\tau}}$.
\end{enum}
\end{defn}

\begin{defn}
If $(\struct{Q}, \struct{S})$ is a bilogical predicate over $\struct{A}$, then we say that an environment $\rho \in \Env[][\struct{A}]$ \emph{satisfies} the predicate $(\struct{Q}, \struct{S})$ if $\rho(x) \in \scomp{\struct{Q}}{\sigma}$ for all $x \in \dom{\rho} \cap \Var[\sigma]$.
\end{defn}

%TODO split up statement of theorem into two parts: first is the lemma used in the proof (Q <= R <= S), second is the current statement. Or: the fundamental theorem is only the first part (Q <= R <= S), and the second part is a corollary of the theorem
\begin{thm}[Fundamental Theorem of Bilogical Relations] \label{thm:fundamental-thm-bilog-rel}
Suppose $\struct{A}$ is a model and $(\struct{Q}, \struct{S})$ is a logical predicate over $\struct{A}$. Then for every term $\typedtm{t}{\sigma}$ and environment $\rho \in \Env[\FV{t}][\struct{A}]$ which satisfies $(\struct{Q}, \struct{S})$, we have $\sint[\struct{A}]{t}{\rho} \in \scomp{\struct{S}}{\sigma}$.
\begin{proof}
Define the logical predicate $\struct{R}$ by $\scomp{\struct{R}}{\beta} = \scomp{\struct{S}}{\beta}$ for all $\beta \in \Basetypes$. We show that
\begin{equation} \label{eq:bilog-pred-inclusions}
\scomp{\struct{Q}}{\sigma} \subs \scomp{\struct{R}}{\sigma} \subs \scomp{\struct{S}}{\sigma}
\end{equation}
for all types $\sigma$. The proof proceeds by induction on $\sigma$.
\begin{items}
\item Case $\beta \in \Basetypes$: We have $\scomp{\struct{Q}}{\beta} \subs \scomp{\struct{R}}{\beta} = \scomp{\struct{S}}{\beta}$ by definition of $\struct{R}$ and condition (i) for bilogical predicates.

\item Case $\functy{\sigma}{\tau}$: Let $f \in \scomp{\struct{Q}}{\functy{\sigma}{\tau}}$ and suppose $x \in \scomp{\struct{R}}{\sigma}$. By the induction hypothesis for $\sigma$, $x \in \scomp{\struct{S}}{\sigma}$, so condition (ii) for bilogical predicates implies $f x \in \scomp{\struct{Q}}{\tau}$. Therefore, by the induction hypothesis for $\tau$, $f x \in \scomp{\struct{R}}{\tau}$. Hence, $f x \in \scomp{\struct{R}}{\tau}$ for all $x \in \scomp{\struct{R}}{\sigma}$, and thus $f \in \scomp{\struct{R}}{\functy{\sigma}{\tau}}$ by definition of $\scomp{\struct{R}}{\functy{\sigma}{\tau}}$.

Now let $f \in \scomp{\struct{R}}{\functy{\sigma}{\tau}}$ and suppose $x \in \scomp{\struct{Q}}{\sigma}$. By the induction hypothesis for $\sigma$, we have $x \in \scomp{\struct{R}}{\sigma}$, hence $f x \in \scomp{\struct{R}}{\tau}$. By the induction hypothesis for $\tau$, we get $f x \in \scomp{\struct{S}}{\tau}$. Hence, we have $f x \in \scomp{\struct{S}}{\tau}$ for all $x \in \scomp{\struct{Q}}{\sigma}$, so $f \in \scomp{\struct{S}}{\functy{\sigma}{\tau}}$ by condition (iii) for bilogical predicates.
\end{items}
This concludes the proof of (\ref{eq:bilog-pred-inclusions}). Now let $t$ and $\rho$ be as in the statement of the theorem. Since $\scomp{\struct{Q}}{\tau} \subs \scomp{\struct{R}}{\tau}$ for all types $\tau$, we have that $\rho$ satisfies $\struct{R}$. Hence, by the Fundamental theorem of logical relations (Theorem~\ref{thm:fundamental-thm-log-rel}), we get $\sint[\struct{A}]{t}{\rho} \in \scomp{\struct{R}}{\sigma} \subs \scomp{\struct{S}}{\sigma}$.
\end{proof}
\end{thm}

%TODO say something about how this is a generalization of the proof in the previous section?

\subsection{``Semantic" proof} \label{sec:log-pred-proof}
%TODO better title
%TODO this section is quite short, is that fine?

\begin{lem} \label{lem:neutral-normal-bilog-pred}
Let
\[ \scomp{\struct{M}}{\sigma} = \setof{[m]}{m \in \Ne[\sigma]} \quad\text{and}\quad
   \scomp{\struct{N}}{\sigma} = \setof{[n]}{n \in \Nf[\sigma]} \]
for all $\sigma$.
Then $(\struct{M}, \struct{N})$ is a bilogical predicate over $\tmmod$.
\begin{proof}
We verify the conditions of bilogical predicates.
\begin{enum}
\item $\scomp{\struct{M}}{\beta} \subs \scomp{\struct{N}}{\beta}$ by rule \textsc{shift}.

\item Let $[m] \in \scomp{\struct{M}}{\functy{\sigma}{\tau}}$ and $[n] \in \scomp{\struct{N}}{\sigma}$, where $m \in \Ne[\functy{\sigma}{\tau}]$ and $n \in \Nf[\sigma]$. Then $\app{m}{n}$ by rule \textsc{app-ne}. Hence, $[m][n] = [\app{m}{n}] \in \scomp{\struct{M}}{\tau}$.

\item Let $[t] \in \tmmod_{\functy{\sigma}{\tau}}$ such that $[t][m] \in \scomp{\struct{N}}{\tau}$ for all $[m] \in \scomp{\struct{M}}{\sigma}$. Then, in particular, $[t][x] = [\app{t}{x}] = [n]$ for some $n \in \Nf[\tau]$, where $x$ is a variable of type $\sigma$ such that $x \notin \FV{t}$. This means that $\conv{\app{t}{x}}{n}{\tau}$, and by rules \textsc{cong-lam} and \textsc{eta}, we derive $\conv{t}{\lamv{x}{\sigma}{n}}{\functy{\sigma}{\tau}}$. Hence, $[t] = [\lamv{x}{\sigma}{n}] \in \scomp{\struct{N}}{\functy{\sigma}{\tau}}$. \qedhere
\end{enum}
\end{proof}
\end{lem}

\begin{proof}[Alternative proof of Theorem~\ref{thm:weak-norm}]
Let $(\struct{M}, \struct{N})$ be the bilogical predicate of Lemma~\ref{lem:neutral-normal-bilog-pred} and let $\idsub$ be the identity substitution on $\FV{t}$. Since $x \in \Ne[\sigma]$ for all $\typedvar{x}{\sigma}$ by rule \textsc{var-ne}, $[\idsub]$ satisfies $(\struct{M}, \struct{N})$. Applying the Fundamental theorem of bilogical relations (Theorem~\ref{thm:fundamental-thm-bilog-rel}), we get that $[t] = \sint[\tmmod]{t}{[\idsub]} \in \scomp{\struct{N}}{\sigma}$. This means that $[t] = [n]$ for some $n \in \Nf[\sigma]$, that is, $\conv{t}{n}{\sigma}$.
\end{proof}

\section{Normalization by evaluation} \label{sec:nbe}

In this section, we define a \textit{normalization function} $\name{nf}_\sigma : \Tm[\sigma] \to \Nf[\sigma]$ that returns the normal form of the input term. It has to satisfy the following requirements:
\begin{enum}
\item $\conv{t}{\name{nf}_\sigma(t)}{\sigma}$ for all $\typedtm{t}{\sigma}$, and
\item if $\conv{t}{u}{\sigma}$, then $\name{nf}_\sigma(t) = \name{nf}_\sigma(u)$.
\end{enum}

The normalization function $\name{nf}_\sigma$ will be constructed using the method of \textit{normalization by evaluation (NbE)} invented by Berger and Schwichtenberg \cite{DBLP:conf/lics/BergerS91}. Informally, NbE works by evaluating the term to be normalized in a suitable model, and then mapping the resulting semantic object back into syntax, producing a normal form evaluating to that object. Since we want the normal form to be convertible to the original term, the mapping provides an inverse to the evaluation function modulo convertibility. Hence, NbE can also be described as \textit{inverting the evaluation function} (\cite{DBLP:conf/lics/BergerS91}).

To implement this idea, we have to solve some technicalities. By the definition of long $\beta\eta$-normal forms (Definition~\ref{def:normal-forms}), a normal form of function type has to be a $\lambda$-abstraction. Hence, when inverting a semantic value of type $\functy{\sigma}{\tau}$, we need to produce a term of the form $\lamv{x}{\sigma}{t}$ where $t \in \Nf[\tau]$. To avoid confusion of free and bound variables, the variable $x$ has to be fresh with respect to the whole computation.

%TODO improve the explanation of the structure of the section? add more pointers into the content?
We solve the problem by keeping track of free variables in order to generate fresh variables when necessary (see Notation~\ref{not:fresh-vars} and Definition~\ref{def:quote-unquote}). For this reason, in Section~\ref{sec:contexts}, we introduce \textit{contexts} whose role is to specify the free variables in a term. Thus, terms in this section are indexed by both types and contexts. Accordingly, in Section~\ref{sec:Kripke-models}, we introduce a notion of ``indexed" model, called a \textit{Kripke model} \cite{DBLP:journals/apal/MitchellM91}, to implement the algorithm. In Section~\ref{sec:nbe-alg}, we construct the algorithm (Definition~\ref{def:normalization-function}) and prove its correctness (Theorem~\ref{thm:norm-fun-correctness}).

%TODO pay attention to chapter reference if structure is changed
There are also alternative ways to solve the problem of generating fresh variable names, see for instance \cite{DBLP:conf/lics/BergerS91, dybjer:2002:appsem}. An advantage of our presentation of normalization by evaluation is that it is closer to the more abstract categorical view on normalization to be discussed in Chapter~\ref{chap:gluing}. However, it is also possible to implement NbE without referring to contexts and using simple Henkin models, see Dybjer and Filinski \cite{dybjer:2002:appsem}. An advantage of Dybjer and Filinski's approach is that their normalization algorithm is more efficient and easier to describe. We believe, however, that the algorithm presented here is mathematically more elegant.
%TODO how good is this argument if we already rely on this fact in the proof?
For instance, it gives a stronger specification to the normalization function, guaranteeing that its output is well-typed in the same context as its input.

%TODO perhaps this should go in introduction? remark that not only NbE but other proofs can also be adapted for different syntactic presentations
We remark that NbE can be implemented for all sorts of syntaxes for the simply typed $\lambda$-calculus. For instance, Kovács \cite{kovacs:2017:msc} formalized NbE for a well-scoped version of $\lambda$-calculus with de Bruijn indices in Agda.

%TODO how to include this? I think it's a nice result. Maybe introduction/related work?
\begin{comment}
We can extract an NbE program from the Tait proof
Say something about program extraction (not formally)
Explain the computational content of lemmas in Tait proof by explicitly relating them to the NbE algorithm
\end{comment}

%TODO how to incorporate these thoughts?
\begin{comment}
Since normalization is achieved by using the evaluator of the metalogic to interpret terms in a model, the algorithm is termed "normalization by evaluation". Thus, in effect, the evaluator of the metalogic is what performs the normalization using its computational rules.

Dybjer and Filinksi, page 3 of introduction
Martin-Löf viewed this kind of normalization proof as a kind of normalization
by intuitionistic model construction: he pointed out that equality (convertibility)
in the object-language is modelled by “definitional equality” in the
meta-language \cite{martin-lof:1975:equality}. Thus the method of normalization works because the
simplification according to this definitional equality is carried out by the evaluator
of the intuitionistic (!) meta-language: hence “normalization by evaluation”.
If instead we work in a classical meta-language, then some extra work would be
needed to implement the meta-language function in a programming language.
\end{comment}

\subsection{Adding contexts} \label{sec:contexts}

\begin{defn}[Context]
A \emph{context} $\Gamma$ is a finite set of variables (of any type), that is, an element of $\finpowset{\Var}$.
\end{defn}

\begin{notn} \label{not:contexts}
\hfill \vspace{-6pt}
\begin{items}
    \item The set of contexts is denoted by $\Con$.
    \item Let $\cTm{\Gamma}{\sigma} = \setof{t \in \Tm[\sigma]}{\FV{t} \subs \Gamma}$. Let $\cNe{\Gamma}{\sigma}$ and $\cNf{\Gamma}{\sigma}$ denote the subsets of neutral terms and normal forms, respectively, in $\cTm{\Gamma}{\sigma}$.
    \item If $t \in \cTm{\Gamma}{\sigma}$, then we write $\ctypedtm{\Gamma}{t}{\sigma}$.
    \item Similarly, if $t, u \in \cTm{\Gamma}{\sigma}$ and $\conv{t}{u}{\sigma}$, then we write $\ctypedconv{\Gamma}{t}{u}{\sigma}$.
\end{items}
\end{notn}

We can give ``context-aware" derivation rules for terms, displayed in Figure~\ref{fig:stlc-terms-with-contexts}. The soundness of these rules is easily shown by unfolding the notations in Notation~\ref{not:contexts} and using Definition~\ref{def:stlc-terms} and Definition~\ref{def:free-and-bound-vars}. We write $(\typedvar{x}{\sigma}) \in \Gamma$ to mean $x \in \Gamma \cap \Var[\sigma]$, and we write $\Gamma, \typedvar{x}{\sigma}$ to mean $\Gamma \cup \singset[x]$ with the assumption that $\typedvar{x}{\sigma}$.

\begin{figure}[ht]
\begin{mathpar}
\inferrule[var]
    {(\typedvar{x}{\sigma}) \in \Gamma}
    {\ctypedtm{\Gamma}{x}{\sigma}}
\and
\inferrule[app]
    {\ctypedtm{\Gamma}{t}{\functy{\sigma}{\tau}} \and \ctypedtm{\Gamma}{u}{\sigma}}
    {\ctypedtm{\Gamma}{\app{t}{u}}{\tau}}
\and
\inferrule[lam]
    {\ctypedtm{\Gamma, \typedvar{x}{\sigma}}{t}{\tau}}
    {\ctypedtm{\Gamma}{\lamv{x}{\sigma}{t}}{\functy{\sigma}{\tau}}}
\end{mathpar}
\caption{Terms of the simply typed $\lambda$-calculus with contexts}
\label{fig:stlc-terms-with-contexts}
\end{figure}

With contexts, we can also give a more refined specification to the normalization function.

%TODO add extra requirements?
\begin{defn} \label{def:normalization-function-reqs}
A \emph{normalization function} is a family of (computable) functions
\[ \normf{\Gamma}{\sigma} : \cTm{\Gamma}{\sigma} \to \cNf{\Gamma}{\sigma} \]
indexed by contexts $\Gamma$ and types $\sigma$, satisfying the following requirements:
\begin{enum}
    \item $\ctypedconv{\Gamma}{t}{\normf{\Gamma}{\sigma}(t)}{\sigma}$ for all $t \in \cTm{\Gamma}{\sigma}$, and
    \item if $\ctypedconv{\Gamma}{t}{u}{\sigma}$, then $\normf{\Gamma}{\sigma}(t) = \normf{\Gamma}{\sigma}(u)$.
    %\item $\normf{\Gamma}{\sigma}(n) = n$ for all $n \in \cNf{\Gamma}{\sigma}$, and
    %\item if $\ctypedtm{\Gamma}{t}{\sigma}$ and $\Gamma \subs \Gamma'$, then $\normf{\Gamma}{\sigma}(t) = \normf{\Gamma'}{\sigma}(t)$.
\end{enum}
\end{defn}

%TODO explain what the requirements mean, and give them names (completeness, soundness[, stability, naturality/compatibility]).

\begin{rem}
Let us explain why we put parentheses around the word \textit{computable} in Definition~\ref{def:normalization-function-reqs}. Constructing a program that implements a normalization function amounts to showing that it is computable. Hence, for a practical implementation of normalization, we need to make sure that the normalization function is computable.

If we work in a constructive metatheory, then the condition is automatically fulfilled since the definitions directly give rise to a functional program. In a classical metatheory, some additional work could be necessary to implement the mathematical normalization function in a programming language and prove that it corresponds to the mathematical definition.
\end{rem}

\subsection{Kripke models} \label{sec:Kripke-models}
%TODO add remarks about category theory?
%TODO add some more explanatory text

For a more thorough discussion of Kripke models, see Mitchell and Moggi \cite{DBLP:journals/apal/MitchellM91}.

\begin{defn}[Kripke applicative structure]
Let $(W, \le)$ be a poset. A \emph{Kripke applicative structure $\struct{A}$ over $W$} consists of
\begin{items}
    \item a family $\kscomp{\struct{A}}{\sigma}{w}$ of sets indexed by $\sigma \in \Ty$ and $w \in W$,
    \item a family of \emph{transition maps} $\kstran[\sigma][w,w'] : \kscomp{\struct{A}}{\sigma}{w} \to \kscomp{\struct{A}}{\sigma}{w'}$ indexed by $\sigma \in \Ty$ and $w, w' \in W$ such that $w \le w'$, and
    \item a family of \emph{application maps} $\ksapp[\sigma,\tau][w] : \cartprod{\kscomp{\struct{A}}{\functy{\sigma}{\tau}}{w}}{\kscomp{\struct{A}}{\sigma}{w}} \to \kscomp{\struct{A}}{\tau}{w}$
\end{items}
such that
\begin{items}
    \item $\kstran[\sigma][w,w]$ is the identity function on $\kscomp{\struct{A}}{\sigma}{w}$,
    \item $\kstran[\sigma][w',w''] \circ \kstran[\sigma][w,w'] = \kstran[\sigma][w,w'']$, and
    \item $\ksapp[\sigma,\tau][w'](\kstran[\functy{\sigma}{\tau}][w,w'](f), \kstran[\sigma][w,w'](x)) = \kstran[\tau][w,w'](\ksapp[\sigma,\tau][w](f, x))$
\end{items}
for all $\sigma, \tau \in \Ty$ and $w \le w' \le w''$.
\end{defn}

\begin{notn}
As before, we may drop the types when they are understood from context, and we may write $fx$ for $\ksapp[\sigma,\tau][w](f, x)$.
\end{notn}

The poset $W$ is viewed as a set of \textit{possible worlds} partially ordered by accessibility. The transition map $\kstran[\sigma][w,w']$ allows us to transport an element at world $w$ to any ``future" world $w'$.

%TODO explain the axioms?

%TODO remark that the A_\sigma^w and i_\sigma^{w,w'} for fixed \sigma form a functor A_\sigma : W -> Set and that the application maps form a natural transformation A_{\functy{\sigma}{\tau}} \times A_\sigma \to A_\tau

For the remainder of this section, we fix some poset $(W, \le)$.
%TODO where to introduce this notation?
The notation $\upset{w}$ stands for the the set $\setof{w' \in W}{w' \ge w}$.

%TODO make this a remark to be able to refer to it?
Similarly to standard applicative structures, every element $f \in \kscomp{\struct{A}}{\sigma}{w}$ determines a function $\kscomp{\struct{A}}{\sigma}{w} \to \kscomp{\struct{A}}{\tau}{w}$ sending $x$ to $f x$. With Kripke applicative structures, more is true: we can use the transition map $\kstran[\functy{\sigma}{\tau}][w,w']$ to view $f$ at world $w'$. Hence, there is a family of functions $\kscomp{\struct{A}}{\sigma}{w'} \to \kscomp{\struct{A}}{\tau}{w'}$ given by $x \mapsto \kstran[][w,w'](f) x$ for every $w' \ge w$, representing the applicative behaviour of $f$ in all future worlds.

\begin{defn}[Extensional Kripke applicative structure]
We say that a Kripke applicative structure $\struct{A}$ over $W$ is \emph{extensional} if the following holds for all $f, g \in \kscomp{\struct{A}}{\functy{\sigma}{\tau}}{w}$:
\[ (\forall w' \ge w. \forall x \in \kscomp{\struct{A}}{\sigma}{w'}.
        \kstran[][w,w'](f)x = \kstran[][w,w'](g)x)
    \quad\text{implies}\quad f = g. \]
\end{defn}

\begin{rem}
Compare this definition with Definition~\ref{def:ext-app-struct}. The property states that if the applicative behaviour of two elements is the same, then the elements must be identical.
\end{rem}

For the definition of a Kripke model, we need the following auxiliary notion.

\begin{defn}[Global element]
Let $\struct{A}$ be a Kripke applicative structure over $W$. A \emph{global element} $a$ of $\struct{A}$ of type $\sigma$ is a family $a_w \in \kscomp{\struct{A}}{\sigma}{w}$ indexed by $w \in W$ such that $\kstran[\sigma][w,w'](a_w) = a_w'$ for all $w \le w'$.
\end{defn}

%TODO remark about terminology? (global element in category theory)

If $a$ is a global element of $\struct{A}$ of type $\sigma$, we write $a \in \kscomp{\struct{A}}{\sigma}{}$ to express this fact. Furthermore, we write $a_w$ for the component of $a$ at $w$.

\begin{defn}[Kripke model]
A \emph{Kripke model $\struct{A}$ over $W$} is an extensional Kripke applicative structure $\struct{A}$ over $W$ with distinguished global elements
\[ K^{\sigma,\tau} \in \kscomp{\struct{A}}{\functy{\sigma}{\functy{\tau}{\sigma}}}{} \quad\text{and}\quad
    S^{\sigma,\tau,\chi} \in \kscomp{\struct{A}}{\functy{(\functy{\sigma}{\functy{\tau}{\chi}})}{\functy{(\functy{\sigma}{\tau})}{\functy{\sigma}{\chi}}}}{} \]
for every $\sigma, \tau, \chi \in \Ty$ such that
\[ K^{\sigma,\tau}_w x y = x \quad\text{and}\quad
    S^{\sigma,\tau,\chi}_w f g x = fx(gx) \]
for all $x, y, f, g$ of the appropriate types.
\end{defn}
\begin{rem}
This is an indexed version of Definition~\ref{def:comb-model}.
\end{rem}

%TODO ugly notation for families
%TODO terminology: standard model over (X, j)?
\begin{ex}[Standard model] \label{ex:standard-Kripke-model}
Let $X = (\kscomp{X}{\beta}{w})_{\beta \in \Basetypes, w \in W}$ be a family of base sets and $j = (\kscomp{j}{\beta}{w,w'} : \kscomp{X}{\beta}{w} \to \kscomp{X}{\beta}{w'})_{\beta \in \Basetypes, w \le w'}$ a family of base transition maps such that
\begin{items}
    \item $\kscomp{j}{\beta}{w,w}$ is the identity on $\kscomp{X}{\beta}{w}$,
    \item $\kscomp{j}{\beta}{w',w''} \circ \kscomp{j}{\beta}{w,w'} = \kscomp{j}{\beta}{w,w''}$
\end{items}
for all $\beta \in \Basetypes$ and $w \le w' \le w''$. We define the model $\kstdmod{X}{j}$, called the \emph{standard (Kripke) model}, as follows. The sets $\kscomp{\kstdmod{X}{j}}{\sigma}{w}$ and the transition maps $\kstran[\sigma][w,w']$ are defined simultaneously by recursion on types:
\begin{align*}
\kscomp{\kstdmod{X}{j}}{\beta}{w}
    &= \kscomp{X}{\beta}{w} \quad (\beta \in \Basetypes) \\
\kstran[\beta][w,w'](x) &= \kscomp{j}{\beta}{w,w'}(x) \quad (\beta \in \Basetypes) \\
%TODO fix set builder notation
\kscomp{\kstdmod{X}{j}}{\functy{\sigma}{\tau}}{w}
    &= \setof{f \in \prod_{w' \ge w}{\funcset{\kscomp{\kstdmod{X}{j}}{\sigma}{w'}}{{\kscomp{\kstdmod{X}{j}}{\tau}{w'}}}}}{\forall w' \ge w. \forall w'' \ge w'. \forall x. f_{w''}(\kstran(x)) = \kstran(f_{w'}(x))} \\
\kstran[\functy{\sigma}{\tau}][w,w']((f)_{w'' \ge w}) &= (f_{w''})_{w'' \ge w'}
\end{align*}
The application map $\ksapp[\sigma,\tau][w]$ is given by $(f, x) \mapsto f_w(x)$. The global elements $K$ and $S$ are defined as follows:
\begin{align*}
((K^{\sigma,\tau}_w)_{w'}(x))_{w''}(y) &= \kstran[][w',w''](x) \\
(((S^{\sigma,\tau,\chi}_w)_{w'}(f))_{w''}(g))_{w'''}(x) &= (f_{w'''}(x))_{w'''}(g_{w'''}(x)).
\end{align*}

%TODO: verify that this is a model, or indicate how to do it.
\end{ex}

%TODO give motivation for the definition of standard Kripke model at function types (refer to the discussion about the applicative behaviour of elements at future worlds). Remark that categorically it is the exponential in the functor category [W, Set] and that the application is the evaluation map (just like for the standard Henkin model in the category Set)

%TODO: Is it nice to do this?
In what follows, $\struct{A}$ denotes a Kripke model over $W$.

\begin{defn}[Environment]
\hfill \vspace{-6pt}
\begin{enum}
\item An \emph{environment for $\struct{A}$} is a partial function $\rho : \cartprod{\Var}{W} \pto \bigcup_{\sigma \in \Ty, w \in W}{\kscomp{\struct{A}}{\sigma}{w}}$ with finite domain such that
\begin{items}
    \item if $\typedvar{x}{\sigma}$ and $(x, w) \in \dom{\rho}$, then $\rho(x, w) \in \kscomp{\struct{A}}{\sigma}{w}$, and
    %TODO the second condition might be called compatibility
    \item if $(x, w) \in \dom{\rho}$ and $w \le w'$, then $(x, w') \in \dom{\rho}$ and $\rho(x, w') = \kstran[][w,w'](\rho(x, w))$.
\end{items}
%TODO: Note: weird terminology? doesn't agree with Kripke model paper
\item If $\Gamma \in \Con$ and $w \in W$, then an environment $\rho$ is said to \emph{satisfy $\Gamma$ at $w$}, notation $\ctypedenv{\rho}{\Gamma}{w}$, if $\cartprod{\Gamma}{\singset[w]} \subs \dom{\rho}$.
\end{enum}
\end{defn}
\begin{rem}
Compare with Definition~\ref{def:environment}.
%TODO explain the comparison, e.g. that \rho(x, w) doesn't have to be defined for all w, and what satisfaction means
\end{rem}

%TODO (figure out and) remark what an environment is categorically

\begin{notn}
For an environment $\rho$, variable $\typedvar{x}{\sigma}$, and $a \in \kscomp{\struct{A}}{\sigma}{w}$, we write $\kupdenv{\rho}{x}{a}$ for the \emph{updated environment} with $\dom{\kupdenv{\rho}{x}{a}} = \dom{\rho} \cup \cartprod{\singset[x]}{\upset{w}}$ and such that
\[ \kupdenv{\rho}{x}{a}(y, w') =
    \begin{cases}
    \rho(y, w') & \text{if } y \ne x \\
    \kstran[][w,w'](a) & \text{if } y = x
    \end{cases}. \]
\end{notn}
\begin{rem}
Compare with Notation~\ref{not:environments}.
%TODO explain the comparison?
\end{rem}

%TODO improve wording
\begin{prop}[Interpretation of $\lambda$-calculus in a Kripke model] \label{prop:Kripke-model-int}
There is an assignment
\[ (t, \rho) \mapsto \ksint{t}{\rho}{w} \in \kscomp{\struct{A}}{\sigma}{w}
    \quad \text{for }\ctypedtm{\Gamma}{t}{\sigma} \text{ and } \ctypedenv{\rho}{\Gamma}{w} \]
such that the following equations are satisfied:
\begin{align}
\ksint{x}{\rho}{w} &= \rho(x, w) \\
\ksint{\app{t}{u}}{\rho}{w} &= \ksint{t}{\rho}{w} \ksint{u}{\rho}{w} \\
\kstran[][w,w'](\ksint{\lamv{x}{\sigma}{t}}{\rho}{w}) a &= \ksint{t}{\kupdenv{\rho}{x}{a}}{w'} \quad (a \in \kscomp{\struct{A}}{\sigma}{w'}, w' \ge w)
\end{align}
\end{prop}

\begin{rem}
Similarly to Henkin models, the clauses in Proposition~\ref{prop:Kripke-model-int} uniquely define the interpretation of a term $t$ in an environment $\rho$. Hence, it can be seen as the definition of interpretation. The well-definedness of the assignment is guaranteed by the existence of the interpretations for the combinators $K$ and $S$. Compare with Remark~\ref{rem:env-model-int-uniq}, Remark~\ref{rem:comb-model-KS-uniq}, and Proposition~\ref{prop:env-comb-mod-equiv}.
\end{rem}

\begin{ex}
Specializing Proposition~\ref{prop:Kripke-model-int} to the standard Kripke model (Example~\ref{ex:standard-Kripke-model}), we get the following \textit{standard interpretation function}:
\begin{align*}
\ksint{x}{\rho}{w} &= \rho(x, w) \\
\ksint{\app{t}{u}}{\rho}{w} &= (\ksint{t}{\rho}{w})_w(\ksint{u}{\rho}{w}) \\
(\ksint{\lamv{x}{\sigma}{t}}{\rho}{w})_{w'}(a) &= \ksint{t}{\kupdenv{\rho}{x}{a}}{w'} \quad (a \in \kscomp{\struct{A}}{\sigma}{w'}, w' \ge w)
\end{align*}
\end{ex}

\begin{defn}[Satisfaction]
\hfill \vspace{-6pt}
\begin{enum}
\item Suppose $\ctypedtm{\Gamma}{t}{\sigma}$, $\ctypedtm{\Gamma}{u}{\sigma}$ and $\ctypedenv{\rho}{\Gamma}{w}$. We say that the equation $\ctypedconv{\Gamma}{t}{u}{\sigma}$ \emph{holds at $w$ and $\rho$ in $\struct{A}$}, written as $\cmodsat{\struct{A}, \rho}{\Gamma}{t}{u}{\sigma}$, if $\ksint{t}{\rho}{w} = \ksint{u}{\rho}{w}$.

\item The model $\struct{A}$ \emph{satisfies} the equation $\ctypedconv{\Gamma}{t}{u}{\sigma}$, written as $\cmodsat{\struct{A}}{\Gamma}{t}{u}{\sigma}$, if $\cmodsat{\struct{A}, \rho}{\Gamma}{t}{u}{\sigma}$ for all $w \in W$ and $\ctypedenv{\rho}{\Gamma}{w}$.
\end{enum}
\end{defn}
\begin{rem}
Compare with Definition~\ref{def:satisfaction}.
\end{rem}

\begin{thm}[Soundness] \label{thm:kripke-soundness}
For all $\ctypedtm{\Gamma}{t}{\sigma}$ and $\ctypedtm{\Gamma}{u}{\sigma}$, we have
\[ \ctypedconv{\Gamma}{t}{u}{\sigma} \quad\text{implies}\quad
    \cmodsat{\struct{A}}{\Gamma}{t}{u}{\sigma}. \]
\begin{proof}
By induction on the proof of $\ctypedconv{\Gamma}{t}{u}{\sigma}$. The proof is similar to that of Theorem~\ref{thm:soundness}.
\end{proof}
\end{thm}

\subsection{The algorithm} \label{sec:nbe-alg}
%TODO better title?
%TODO add some more explanatory text

We are now ready to define the algorithm. Our algorithm is similar to that of Reynolds \cite{reynolds1998normalization}. The main steps of the algorithm are as follows:
\begin{enum}
    \item We construct a suitable model $\struct{S}$ (Definition~\ref{def:nbe-model}).
    \item The model gives rise to an interpretation $\sem{-}$ of terms in the model.
    \item We define a family of functions $\quotef{\Gamma}{\sigma} : \kscomp{\struct{S}}{\sigma}{\Gamma} \to \cNf{\Gamma}{\sigma}$ indexed by types and contexts by recursion on $\sigma$. These functions provide an inverse to the interpretation/evaluation function $\sem{-}$. Due to the contravariance of function types in the first argument, we also need a family of functions $\unquotef{\Gamma}{\sigma} : \cNe{\Gamma}{\sigma} \to \kscomp{\struct{S}}{\sigma}{\Gamma}$ in the other direction, embedding neutral terms (in particular, variables) into the semantics.
    \item We derive the normalization function $\normf{\Gamma}{\sigma}$ from the ingredients above. For this, we need a special environment $\idenv{\Gamma}$ (Definition~\ref{def:idenv}) that maps all variables in $\Gamma$ to their semantic counterpart.
\end{enum}

There is a close analogy between these steps and the structure of the proof of Section~\ref{sec:Tait-proof}.
\begin{enum}
    \item The model $\struct{S}$ corresponds to the convertibility predicate $R$ (Definition~\ref{def:conv-pred}). In particular, both $\kscomp{\struct{S}}{\sigma}{\Gamma}$ and $R_\sigma$ are defined by induction on the type $\sigma$.
    \item The fact that there is an interpretation $\sem{-}$ of terms in the model corresponds the Fundamental Lemma (Lemma~\ref{lem:fundamental-lem-norm}).
    \item The functions $\quotef{\Gamma}{\sigma}$ and $\unquotef{\Gamma}{\sigma}$ correspond to the two parts of Lemma~\ref{lem:computability-normal-forms}, respectively.
    \item The derivation of the normalization function follows the proof of weak normalization in Section~\ref{sec:Tait-proof}. In particular, the environment $\idenv{\Gamma}$ corresponds to the fact that the identity substitution on $\Gamma$ satisfies the logical predicate $R$.
\end{enum}

We now present the construction of our normalization function.

\begin{defn}
We write $C$ for the poset of contexts ordered by inclusion.
\end{defn}

\begin{defn} \label{def:nbe-model}
Let $\struct{S}$ be the standard Kripke model over $C$, where the base families $(X, j)$ are given by $\kscomp{X}{\beta}{\Gamma} = \cNf{\Gamma}{\beta}$ and $\kscomp{j}{\beta}{\Gamma,\Gamma'}(n) = n$.
\end{defn}

\begin{notn} \label{not:fresh-vars}
\hfill \vspace{-6pt}
\begin{items}
\item For a context $\Gamma$, let $\fresh{\Gamma}{\sigma}$ denote a fresh variable of type $\sigma$ for $\Gamma$, that is, a variable $\fresh{\Gamma}{\sigma} \in \Var[\sigma]$ such that $\fresh{\Gamma}{\sigma} \notin \Gamma$.

\item Let $\extfresh{\Gamma}{\sigma}$ denote the context $\Gamma \cup \singset[\fresh{\Gamma}{\sigma}]$.
\end{items}
\end{notn}

\begin{defn} \label{def:quote-unquote}
We define two families of functions $\quotef{\Gamma}{\sigma} : \kscomp{\struct{S}}{\sigma}{\Gamma} \to \cNf{\Gamma}{\sigma}$ and $\unquotef{\Gamma}{\sigma} : \cNe{\Gamma}{\sigma} \to \kscomp{\struct{S}}{\sigma}{\Gamma}$ simultaneously by recursion on $\sigma$:
\begin{align*}
\quotef{\Gamma}{\beta}(n) &= n \\
\quotef{\Gamma}{\functy{\sigma}{\tau}}(f) &= \lamv{\fresh{\Gamma}{\sigma}}{\sigma}{\quotef{\extfresh{\Gamma}{\sigma}}{\tau}(f_{\extfresh{\Gamma}{\sigma}}(\quotef{\extfresh{\Gamma}{\sigma}}{\sigma}(\fresh{\Gamma}{\sigma})))} \\
\\
\unquotef{\Gamma}{\beta}(m) &= m \\
(\unquotef{\Gamma}{\functy{\sigma}{\tau}}{m})_{\Gamma'}(a)
    &= \unquotef{\Gamma'}{\tau}(\app{m}{\quotef{\Gamma'}{\sigma}(a)})
\end{align*}
\end{defn}

%TODO remark that \quotef fills the role of the mapping from semantic objects to normal forms referred to in the introduction. Because of the contravariance of function types, we also need a function \unquotef in the other direction that works on neutral terms. This duality is reminiscent of the mutual definition of normal forms and neutral terms.

\begin{comment}
Note the condition "$x$ fresh" in the second case of $\quotef$. This is obviously not a well-defined condition from a mathematical perspective. Rather, it aims to provide computational intuition: when computing the result of $\quotef$ at a function type, we generate a fresh variable name in order avoid capturing the other free variables in the term returned by the recursive call. It is not too difficult to avoid this ambiguity by slightly modifying the interpretation, allowing us to control which variables may occur free in terms.
\end{comment}

%TODO improve the wording of definitions and lemmas from this point

\begin{defn} \label{def:idenv}
For all contexts $\Gamma$, we define the environment $\idenv{\Gamma}$ as follows:
\[ \idenv{\Gamma}(x, \Gamma') = \unquotef{\Gamma'}{\sigma}(x)
    \quad \text{for } (x : \sigma) \in \Gamma' \sups \Gamma \]
\end{defn}

%TODO prove this
We have that $\ctypedenv{\idenv{\Gamma}}{\Gamma}{\Gamma}$.

\begin{defn} \label{def:normalization-function}
We define the function $\normf{\Gamma}{\sigma}$ as follows:
\[ \normf{\Gamma}{\sigma}(t) = \quotef{\Gamma}{\sigma}(\ksint{t}{\idenv{\Gamma}}{\Gamma}). \]
\end{defn}

We now prove the correctness of $\normf{}{}$ by showing that it is a normalization function in the sense of Definition~\ref{def:normalization-function-reqs}.

%TODO
\begin{comment}
add more lemmas, possibly:
- naturality of quote/unquote
- compatibility of \eta (i.e. \eta_{\extfresh{\Gamma}{\sigma}} = \kupdenv{\eta_\Gamma}{\fresh{\Gamma}{\sigma}}{\unquotef{\sigma}(\fresh{\Gamma}{\sigma})}
\end{comment}

\begin{lem} \label{lem:norm-idempotent-nf}
\hfill \vspace{-6pt}
\begin{enum}
\item For all $m \in \cNe{\Gamma}{\sigma}$, we have $\ksint{m}{\idenv{\Gamma}}{\Gamma} = \unquotef{\Gamma}{\sigma}(m)$.
\item For all $n \in \cNf{\Gamma}{\sigma}$, we have $\normf{\Gamma}{\sigma}(n) = n$.
\end{enum}
\begin{proof}
By simultaneous induction on neutral terms and normal forms.
%TODO complete this proof
\end{proof}
\end{lem}

%TODO prove the other requirements too (if they are present)
\begin{thm} \label{thm:norm-fun-correctness}
The function $\normf{\Gamma}{\sigma}$ defined in Definition~\ref{def:normalization-function} satisfies the requirements in Definition~\ref{def:normalization-function-reqs}.
\begin{proof}
Condition (ii) follows from Theorem~\ref{thm:kripke-soundness} and the definition of $\normf{\Gamma}{\sigma}$. By Theorem~\ref{thm:weak-norm}, we know that there is $n \in \Nf[\sigma]$ such that $\conv{t}{n}{\sigma}$. By Condition (ii) and Lemma~\ref{lem:norm-idempotent-nf}, we get $\normf{\Gamma}{\sigma}(t) = \normf{\Gamma}{\sigma}(n) = n$. Hence, $\conv{t}{\normf{\Gamma}{\sigma}(t)}{\sigma}$, showing Condition (i).
%TODO fix this
(Note: technically, we need to show $\ctypedconv{\Gamma}{t}{\normf{\Gamma}{\sigma}(t)}{\sigma}$. For this, we need that $\ctypedtm{\Gamma}{n}{\sigma}$, which can be shown by a syntactic argument).
\end{proof}
\end{thm}

\begin{cor} \label{cor:conv-normal-forms}
For all $\ctypedtm{\Gamma}{t}{\sigma}$ and $\ctypedtm{\Gamma}{u}{\sigma}$, we have $\ctypedconv{\Gamma}{t}{u}{\sigma}$ iff $\normf{\Gamma}{\sigma}(t) = \normf{\Gamma}{\sigma}(u)$.
\begin{proof}
The only if direction is Definition~\ref{def:normalization-function-reqs} (ii). The if direction follows from Definition~\ref{def:normalization-function-reqs} (i), and reflexivity, transitivity, and symmetry of the convertibility relation.
\end{proof}
\end{cor}

\begin{cor} \label{cor:conv-dec}
Given $\ctypedtm{\Gamma}{t}{\sigma}$ and $\ctypedtm{\Gamma}{t}{\sigma}$, it is decidable whether $\ctypedconv{\Gamma}{t}{u}{\sigma}$.
\begin{proof}
By Corollary~\ref{cor:conv-normal-forms}, we can compute $\normf{\Gamma}{\sigma}(t)$ and $\normf{\Gamma}{\sigma}(u)$ and compare the resulting normal forms for syntactic equality.
\end{proof}
\end{cor}

%TODO give a second proof of completeness using binary bilogical relations?
%TODO abstract away quote-unquote shenanigans so that defining nf and proving its completeness follow from the same general theorem? (this would be a proof-relevant version of bilogical relations) Problem: need to introduce Kripke logical relations, difficult to phrase result without categorical language. It might be better to do this for the categorical case


\chapter{Categorical preliminaries} \label{chap:cat-prelims}
%TODO add references to standard category theory literature?
In this chapter, we recall some important categorical notions, constructions, and theorems that are used throughout the upcoming chapters, and we fix the associated notation.

Categories are denoted by uppercase Latin letters in calligraphic font (e.g. $\cat{A}, \cat{B}, \cat{C}$). Specific named categories are written with boldface letters, such as the category $\Set$ of sets and functions and the category $\Cat$ of small categories and functors. The collection of objects of a category $\cat{C}$ is denoted by $\Ob{\cat{C}}$. To indicate that $A$ is an object of $\cat{C}$, we also write $A \in \cat{C}$ instead of $A \in \Ob{\cat{C}}$. For $A, B \in \cat{C}$, the collection of morphisms from $A$ to $B$ is denoted by $\Hom[\cat{C}]{A}{B}$, or $\Hom{A}{B}$ if the category $\cat{C}$ is clear from the context. The identity morphism on $A$ is written as $\id[A]$, and the composition of morphisms $f : B \to C$ and $g : A \to B$ is written as $f \circ g$ or $fg$. The subscript of $\id[A]$ is sometimes dropped when understood from context.

The image of an object $A \in \cat{C}$ under a functor $F : \cat{C} \to \cat{D}$ is written as $F(A)$ or $FA$. Similarly, the image of a morphism $f$ is written as $F(f)$ or $Ff$. The identity functor on $\cat{C}$ is denoted by $\idfunc[\cat{C}]$ or $\id[\cat{C}]$. Just like for morphisms, the composition of functors $F : \cat{B} \to \cat{C}$ and $G : \cat{A} \to \cat{B}$ is written as $F \circ G$ or $FG$.
%TODO just use unambiguous notation?
Note that juxtaposition is thus overloaded to mean both composition and application. However, we try to avoid ambiguity by employing appropriate notation.

Given a natural transformation $\mu : F \to G$ between functors $F, G : \cat{C} \to \cat{D}$, we denote its component at $A \in \cat{C}$ by $\mu_A$. The vertical composition of natural transformations $\mu : G \to H$ and $\nu : F \to G$ is again denoted by $\mu \circ \nu$ or $\mu\nu$. For a natural transformation $\mu : F \to G$ between functors $F, G : \cat{C} \to \cat{D}$ and functor $H : \cat{B} \to \cat{C}$, we write $\mu H : FH \to GH$ for the natural transformation given by $(\mu H)_A = \mu_{HA} : FHA \to GHA$. The operation sending $\mu$ to $\mu H$ is called \textit{whiskering} (on the right).

%TODO: move notation for functor category here?

%TODO structure of chapter?

\section{Universal properties}
%TODO add examples in the category of sets?

\begin{defn} \label{def:terminal-object}
An object $T$ in a category $\cat{C}$ is called \emph{terminal} if for every object $X$ of $\cat{C}$ there is exactly one morphism $h : X \to T$. Diagrammatically:
% https://q.uiver.app/#q=WzAsMixbMCwwLCJYIl0sWzEsMCwiVCJdLFswLDEsIlxcZXhpc3RzISBoIiwwLHsic3R5bGUiOnsiYm9keSI6eyJuYW1lIjoiZGFzaGVkIn19fV1d
\[\begin{tikzcd}
	X & T
	\arrow["{\exists! h}", dashed, from=1-1, to=1-2]
\end{tikzcd}\]
\end{defn}

Given a terminal object $T$, we write $\mterm[X] : X \to T$ for the unique morphism into $T$. If a category $\cat{C}$ has a terminal object, it is denoted by $\1[\cat{C}]$ or simply $\1$.

\begin{ex} \label{ex:set-terminal-object}
In the category $\Set$, every singleton set is terminal. A canonical choice for the terminal object is the set $\singset[\nul]$ containing only the empty set. Since the choice does not matter, we choose some singleton set $\singset$ for the terminal object and use the notation $\singel$ for its only element. The unique morphism $\mterm[X] : X \to \singset$ is the constant map with value $\singel$.
\end{ex}

\begin{ex}
%TODO rephrase?
In the category $\Cat$, any category with exactly one object and one identity morphism on that object is terminal.
\end{ex}

The dual notion of a terminal object is an initial object.

\begin{defn} \label{def:initial-object}
An object $I$ in a category $\cat{C}$ is called \emph{initial} if for every object $X$ of $\cat{C}$ there is exactly one morphism $h : I \to X$. Diagrammatically:
% https://q.uiver.app/#q=WzAsMixbMSwwLCJYIl0sWzAsMCwiSSJdLFsxLDAsIlxcZXhpc3RzISBoIiwwLHsic3R5bGUiOnsiYm9keSI6eyJuYW1lIjoiZGFzaGVkIn19fV1d
\[\begin{tikzcd}
	I & X
	\arrow["{\exists! h}", dashed, from=1-1, to=1-2]
\end{tikzcd}\]
\end{defn}

Given an initial object $I$, we write $\minit[X] : I \to X$ for the unique morphism out of $I$. If a category $\cat{C}$ has an initial object, it is denoted by $\0[\cat{C}]$ or simply $\0$.

\begin{ex} \label{ex:set-initial-object}
The category of sets has exactly one initial object, namely, the empty set $\nul$. The unique morphism $\minit[X] : \nul \to X$ is the empty function.
\end{ex}

\begin{defn} \label{def:binary-products}
Let $A$ and $B$ be objects in $\cat{C}$. Then a \emph{(binary) product} of $A$ and $B$ is an object $P$ together with morphisms $p_1 : P \to A$, $p_2 : P \to B$ satisfying the following universal property: for every pair of morphisms $f : X \to A$, $g : X \to B$ there exists a unique morphism $h : X \to P$ such that 
\[ p_1 \circ h = f \quad\text{and}\quad p_2 \circ h = g. \]
Diagrammatically:
% https://q.uiver.app/#q=WzAsNCxbMSwxLCJQIl0sWzAsMSwiQSJdLFsyLDEsIkIiXSxbMSwwLCJYIl0sWzAsMSwicF8xIl0sWzAsMiwicF8yIiwyXSxbMywxLCJmIiwyXSxbMywyLCJnIl0sWzMsMCwiXFxleGlzdHMhIGgiLDEseyJsYWJlbF9wb3NpdGlvbiI6NDAsInN0eWxlIjp7ImJvZHkiOnsibmFtZSI6ImRhc2hlZCJ9fX1dXQ==
\[\begin{tikzcd}
	& X \\
	A & P & B
	\arrow["{p_1}", from=2-2, to=2-1]
	\arrow["{p_2}"', from=2-2, to=2-3]
	\arrow["f"', from=1-2, to=2-1]
	\arrow["g", from=1-2, to=2-3]
	\arrow["{\exists! h}"{description, pos=0.4}, dashed, from=1-2, to=2-2]
\end{tikzcd}\]
\end{defn}

If $(P, p_1, p_2)$ is a product of $A$ and $B$, then $P$ is denoted by $\cprod{A}{B}$. The maps $p_1$ and $p_2$ are called \emph{projections}, and they are denoted by $\mfst[A,B] : \cprod{A}{B} \to A$ and $\msnd[A,B] : \cprod{A}{B} \to B$, respectively. In the literature, the object $\cprod{A}{B}$ itself is often referred to as the product of $A$ and $B$, leaving the projections implicit; we follow this convention. Given maps $f : X \to A$, $g : X \to B$, the unique morphism arising from the universal property is written as $\mpair{f}{g} : X \to \cprod{A}{B}$ and is called the \emph{pairing} of $f$ and $g$.

\begin{ex} \label{ex:set-products}
The categorical product of sets $A$ and $B$ is the cartesian product $\cartprod{A}{B}$ together with the projection functions $\mfst[A,B] : \cartprod{A}{B} \to A$ and $\msnd[A,B] : \cartprod{A}{B} \to B$ sending $(a,b)$ to $a$ and $b$, respectively. The pairing of functions $f : X \to A$ and $g : X \to B$ is the function $\mpair{f}{g} : X \to \cartprod{A}{B}$ given by $\mpair{f}{g}(x) = (f(x), g(x))$.
\end{ex}

\begin{defn} \label{def:product-functor}
Suppose that for every pair of objects $A, B$ in $\cat{C}$ we have chosen a product $\cprod{A}{B}$ of $A$ and $B$. Then we can define the \emph{product functor} $\cprod{-}{-} : \prodcat{\cat{C}}{\cat{C}} \to \cat{C}$ that sends a pair of objects $(A, B)$ to $\cprod{A}{B}$ and a pair of morphisms $(f : A \to A', g : B \to B')$ to
\[ \cprod{f}{g} = \mpair{f \circ \mfst[A,B]}{g \circ \msnd[A,B]}
    : \cprod{A}{B} \to \cprod{A'}{B'}. \]
\end{defn}
\begin{prop}
The operations defined in Definition~\ref{def:product-functor} constitute a functor.
%TODO proof?
\end{prop}

%TODO what is the precedence of exponentials? should it be parenthesized?
\begin{defn} \label{def:exponentials}
Let $A$ and $B$ be objects in $\cat{C}$ such that for every object $X$ of $\cat{C}$ we have chosen a product $\cprod{X}{A}$ of $X$ and $A$. Then an \emph{exponential} of $A$ and $B$ is an object $E$ together with a map $\epsilon : \cprod{E}{A} \to B$ satisfying the following universal property: for any morphism $f : \cprod{X}{A} \to B$ there exists a unique morphism $h : X \to E$ such that
\[ \epsilon \circ (\cprod{h}{\id[A]}) = f. \]
Diagrammatically:
% https://q.uiver.app/#q=WzAsNSxbMSwxLCJcXGNwcm9ke1h9e0F9Il0sWzIsMCwiQiJdLFsxLDAsIlxcY3Byb2R7RX17QX0iXSxbMCwxLCJYIl0sWzAsMCwiRSJdLFswLDEsImYiLDJdLFsyLDEsIlxcZXBzaWxvbiJdLFswLDIsIlxcY3Byb2R7aH17XFxpZFtBXX0iXSxbMyw0LCJcXGV4aXN0cyEgaCIsMCx7InN0eWxlIjp7ImJvZHkiOnsibmFtZSI6ImRhc2hlZCJ9fX1dXQ==
\[\begin{tikzcd}
	E & {\cprod{E}{A}} & B \\
	X & {\cprod{X}{A}}
	\arrow["f"', from=2-2, to=1-3]
	\arrow["\epsilon", from=1-2, to=1-3]
	\arrow["{\cprod{h}{\id[A]}}", from=2-2, to=1-2]
	\arrow["{\exists! h}", dashed, from=2-1, to=1-1]
\end{tikzcd}\]
\end{defn}

%TODO parentheses around exponential?
%TODO mention that the exponential transpose of h is \mev \circ (\cprod{h}{\id})?
If $(E, \epsilon)$ is an exponential of $A$ and $B$, then $E$ is denoted by $\cexp{A}{B}$. The map $\epsilon$ is called \emph{evaluation}, and it is denoted by $\mev[A,B] : \cprod{(\cexp{A}{B})}{B} \to A$. Similarly to products, the object $\cexp{A}{B}$ itself is often referred to as the exponential of $A$ and $B$ in the literature, leaving the evaluation implicit; we follow this convention. Given a map $f : \cprod{X}{A} \to B$, the unique morphism $h : X \to \cexp{A}{B}$ arising from the universal property is written as $\mcurry{f}$, and the maps $f$ and $h$ are called \emph{exponential transposes} of each other. The operation sending $f$ to $\mcurry{f}$ is also called \emph{currying}.

\begin{ex} \label{ex:set-exponentials}
The exponential $\cexp{A}{B}$ in $\Set$ is given by the set $\funcset{A}{B}$ of functions from $A$ to $B$, and the evaluation map $\mev[A,B] : \cartprod{\funcset{A}{B}}{A} \to B$ sends a pair $(f, x)$ to $f(x)$. The exponential transpose of a function $f : \cartprod{X}{A} \to B$ is the function $\mcurry{f} : X \to \funcset{A}{B}$ that sends an element $x \in X$ to the function $a \mapsto f(x, a)$.
\end{ex}

%TODO parentheses around exponential?
\begin{defn} \label{def:exponential-functor}
Suppose that for every pair of objects $A, B$ in $\cat{C}$ we have chosen an exponential $\cexp{A}{B}$ of $A$ and $B$ (note that this also requires choices for products). Then we can define the \emph{exponential functor} $\cexp{-}{-} : \prodcat{\op{\cat{C}}}{\cat{C}} \to \cat{C}$ that sends a pair of objects $(A,B)$ to $\cexp{A}{B}$ and a pair of morphisms $(f : A' \to A, g : B \to B')$ to
\[ \cexp{f}{g} = \mcurry{g \circ \mev[A,B] \circ (\cprod{\id}{f})}
    : \cexp{A}{B} \to \cexp{A'}{B'}. \]
\end{defn}
\begin{prop}
The operations defined in Definition~\ref{def:exponential-functor} constitute a functor.
%TODO proof?
\end{prop}

\begin{defn}
Let $f : A \to C$ and $g : B \to C$ be morphisms in $\cat{C}$. Then a \emph{pullback} of $f$ and $g$ is an object $P$ together with morphisms $p_1 : P \to A$, $p_2 : P \to B$ satisfying the following universal property: for every pair of morphisms $a : X \to A$, $b : X \to B$ such that $fa = gb$, there exists a unique morphism $h : X \to P$ such that 
\[ p_1 \circ h = a \quad\text{and}\quad p_2 \circ h = b. \]
Diagrammatically:
% https://q.uiver.app/#q=WzAsNSxbMSwxLCJQIl0sWzIsMSwiQiJdLFsyLDIsIkMiXSxbMSwyLCJBIl0sWzAsMCwiWCJdLFsxLDIsImciXSxbMywyLCJmIiwyXSxbMCwzLCJwXzEiXSxbNCwzLCJhIiwyXSxbMCwxLCJwXzIiLDJdLFs0LDEsImIiXSxbNCwwLCJcXGV4aXN0cyEgaCIsMSx7InN0eWxlIjp7ImJvZHkiOnsibmFtZSI6ImRhc2hlZCJ9fX1dXQ==
\[\begin{tikzcd}
	X \\
	& P & B \\
	& A & C
	\arrow["g", from=2-3, to=3-3]
	\arrow["f"', from=3-2, to=3-3]
	\arrow["{p_1}", from=2-2, to=3-2]
	\arrow["a"', from=1-1, to=3-2]
	\arrow["{p_2}"', from=2-2, to=2-3]
	\arrow["b", from=1-1, to=2-3]
	\arrow["{\exists! h}"{description}, dashed, from=1-1, to=2-2]
\end{tikzcd}\]
\end{defn}

The notation and terminology of pullbacks is similar to that of products. If $(P, p_1, p_2)$ is a pullback of $f : A \to C$ and $g : B \to C$, then the maps $p_1$ and $p_2$ are called \emph{projections}. In the literature, the object $P$ itself is often referred to as the pullback of $f$ and $g$, leaving the projections implicit; we follow this convention. Given maps $a : X \to A$, $b : X \to B$, the unique morphism arising from the universal property is written as $\mpair{a}{b} : X \to P$ and is called the \emph{pairing} of $a$ and $b$.

\begin{ex} \label{ex:set-pullbacks}
In the category $\Set$, the pullback of functions $f : A \to C$ and $g : B \to C$ is the set
\[ P = \setof{(x, y) \in \cartprod{A}{B}}{f(x) = g(y)} \]
together with the projection functions $\mfst[A,B]$ and $\msnd[A,B]$ restricted to $P$. Given a commuting square
% https://q.uiver.app/#q=WzAsNCxbMCwwLCJYIl0sWzAsMSwiQSJdLFsxLDAsIkIiXSxbMSwxLCJDIl0sWzEsMywiZiIsMl0sWzIsMywiZyJdLFswLDEsImEiLDJdLFswLDIsImIiXV0=
\[\begin{tikzcd}
	X & B \\
	A & C
	\arrow["f"', from=2-1, to=2-2]
	\arrow["g", from=1-2, to=2-2]
	\arrow["a"', from=1-1, to=2-1]
	\arrow["b", from=1-1, to=1-2]
\end{tikzcd}\]
the pairing $\mpair{a}{b} : X \to \cartprod{A}{B}$ factors through $P$, i.e. we have $\mpair{a}{b}(x) \in P$ for all $x \in X$. This is because $\mpair{a}{b}(x) = (a(x), b(x))$ and $f(a(x)) = (fa)(x) = (gb)(x) = g(b(x))$ for all $x \in X$ due to the commutativity of the square above.
\end{ex}

%TODO add remarks/propositions about the uniqueness of terminal objects/products/exponentials/pullbacks?

%TODO add remarks that arrows with codomain a product/pullback are equal iff the morphisms obtained by postcomposition by the projections are equal

%TODO above lemmas/propositions will be used without explicit reference

\section{Cartesian closed categories and functors}

\begin{defn} \label{def:ccc}
A \emph{cartesian closed category} (or \textit{CCC} for short) is a category equipped with
\begin{items}
    \item a choice of a terminal object $\1$,
    \item an operation which sends a pair of objects $A$ and $B$ to a product $(\cprod{A}{B}, \mfst[A,B], \msnd[A,B])$ of $A$ and $B$, and
    \item an operation which sends a pair of objects $A$ and $B$ to an exponential $(\cexp{A}{B}, \mev[A,B])$ of $A$ and $B$.
\end{items}
\end{defn}

We often omit the objects in the subscripts of the projections and the evaluation when they are understood from context.

\begin{rem} \label{rem:ccc-structure-vs-property}
%TODO add reference to literature?
Definition~\ref{def:ccc} defines a CCC as a category equipped with the additional data of choices of a terminal object and products and exponentials for every pair of objects. That is, a CCC is a 4-tuple $(\cat{C}, \1, \cprod{-}{-}, \cexp{-}{-})$. In the literature, another common definition is to require the \textit{mere existence} of a terminal object, products, and exponentials, without fixing any particular choices. To elucidate the distinction, we may refer to a CCC in the former sense as a \emph{CCC with structure} and to a CCC in the latter sense as a \emph{CCC with property}. The unqualified term CCC refers to a CCC with structure.

Assuming the axiom of choice, the two definitions are equivalent since it is always possible to choose products, exponentials, or other structures defined by universal properties. In constructive mathematics, however, it is often easier to carry around a choice of such structures together with the categories. For instance, to define the product functor $\cprod{-}{-} : \prodcat{\cat{C}}{\cat{C}} \to \cat{C}$ (Definition~\ref{def:product-functor}), it is necessary to have a choice of a product for every pair of objects in $\cat{C}$.

The distinction between \textit{structure} and \textit{property} also becomes important (even in the classical setting) once one considers the notion of morphism between structured categories. In particular, one may consider (at least) two kinds of morphisms between CCCs: \textit{strict} morphisms which preserve the additional structure \textit{on the nose}, i.e. up to equality, and \textit{weak} morphisms which preserve it only up to isomorphism. A more detailed explanation can be found in Remark~\ref{rem:ccc-strict-vs-weak-preservation}.
\end{rem}

\begin{ex} \label{ex:set-is-ccc}
The category $\Set$ is cartesian closed. This follows from examples \ref{ex:set-terminal-object}, \ref{ex:set-products}, and \ref{ex:set-exponentials}.
\end{ex}

\begin{defn} \label{def:strict-preservation}
Let $\cat{C}$ and $\cat{D}$ be cartesian closed categories and let $F : \cat{C} \to \cat{D}$ be a functor. We say that
\begin{enum}
    \item $F$ \emph{strictly preserves the terminal object} if $F(\1[\cat{C}]) = \1[\cat{D}]$;
    \item $F$ \emph{strictly preserves products} if
    \[ F(\cprod{A}{B}) = \cprod{FA}{FB}, \quad
        F(\mfst[A,B]) = \mfst[FA,FB], \quad\text{and}\quad
        F(\msnd[A,B]) = \msnd[FA,FB] \]
    for all $A, B \in \cat{C}$;
    \item $F$ \emph{strictly preserves exponentials} if
    \[ F(\cexp{A}{B}) = \cexp{FA}{FB} \quad\text{and}\quad
        F(\mev[A,B]) = \mev[FA,FB] \]
    for all $A, B \in \cat{C}$.
\end{enum}
\end{defn}

\begin{defn} \label{def:strict-cc-functor}
A functor $F : \cat{C} \to \cat{D}$ between CCCs is called \emph{strict cartesian closed} if it strictly preserves the terminal object, products, and exponentials.    
\end{defn}

\begin{rem} \label{rem:ccc-strict-vs-weak-preservation}
%TODO replace 'make sense' with a more appropriate phrase?
The adjective \textit{strict} in Definition~\ref{def:strict-preservation} indicates that the structure is preserved \textit{up to equality}: the chosen structure on $\cat{C}$ is mapped to the chosen structure on $\cat{D}$. Note that this notion of structure preserving morphism does not make sense when $\cat{C}$ or $\cat{D}$ does not carry a chosen structure. This is our main motivation for adopting CCCs with structure since we need to work with strict morphisms in Chapter~\ref{chap:gluing}.

%TODO reference to literature?
There is another, more widespread, notion of preservation in the literature on category theory, whereby the structure is only preserved \textit{up to isomorphism}. For instance, in the case of products, we could weaken the condition $F(\cprod{A}{B}) = \cprod{FA}{FB}$ to $F(\cprod{A}{B}) \cong \cprod{FA}{FB}$.  We refer to this idea as \textit{weak} preservation to distinguish it from the strict notion. An advantage of weak preservation is that it is possible to generalize it to the case when $\cat{C}$ or $\cat{D}$ does not have all (chosen) products; see Definition~\ref{def:weak-preserve-products}.
\end{rem}

%TODO introduce terminology for categories with terminal object/products/exponentials?
%TODO replace 'make sense' with a more appropriate phrase?
The notions in Definition~\ref{def:strict-preservation} also make sense when $\cat{C}$ and $\cat{D}$ are not necessarily cartesian closed. For instance, if $\cat{C}$ and $\cat{D}$ only have chosen terminal objects $\1[\cat{C}]$ and $\1[\cat{D}]$, we still say that $F$ strictly preserves the terminal object whenever $F(\1[\cat{C}]) = \1[\cat{D}]$. Similarly, we only need to assume the existence of chosen products to state that a functor strictly preserves them.

%TODO is this proposition necessary?
From the universal properties of products and exponentials, it follows that a strict cartesian closed functor strictly preserves pairing and currying too.

\begin{prop}
Let $F : \cat{C}\to \cat{D}$ be a strict cartesian closed functor. Then
\begin{enum}
    \item $F(\mterm[X]) = \mterm[FX]$ for all $X \in \cat{C}$;
    \item $F(\mpair{f}{g}) = \mpair{Ff}{Fg}$ for all $f : X \to A$ and $g : X \to B$;
    \item F($\cprod{f}{g}) = \cprod{Ff}{Fg}$ for all $f : A \to A'$ and $g : B \to B'$;
    \item $F(\mcurry{f}) = \mcurry{Ff}$ for all $f : \cprod{X}{A} \to B$.
\end{enum}
\begin{proof}
\begin{enum}
    \item The equality follows since both sides are maps into the terminal object $F(\1[\cat{C}]) = \1[\cat{D}]$.
    
    \item Note that we have
    \[ \mfst[FA,FB] \circ F(\mpair{f}{g}) = F(\mfst[A,B]) \circ F(\mpair{f}{g})
    = F(\mfst[A,B] \circ \mpair{f}{g}) = Ff \]
    and
    \[ \msnd[FA,FB] \circ F(\mpair{f}{g}) = F(\msnd[A,B]) \circ F(\mpair{f}{g})
    = F(\msnd[A,B] \circ \mpair{f}{g}) = Fg. \]
    However, since $\mpair{Ff}{Fg}$ is the unique morphism $FX \to \cprod{FA}{FB}$ such that $\mfst[FA,FB] \circ \mpair{Ff}{Fg} = Ff$ and $\msnd[FA,FB] \circ \mpair{Ff}{Fg} = Fg$, we must have $F(\mpair{f}{g}) = \mpair{Ff}{Fg}$.
    
    \item We calculate:
    \begin{align*}
    F(\cprod{f}{g})
       &= F(\mpair{f \circ \mfst[A,B]}{g \circ \msnd[A,B]})
        = \mpair{F(f \circ \mfst[A,B])}{F(g \circ \msnd[A,B])} \\
       &= \mpair{Ff \circ \mfst[FA,FB]}{Fg \circ \msnd[FA,FB]}
        = \cprod{Ff}{Fg}.
    \end{align*}
    
    \item We have
    \begin{align*}
    \mev[FA,FB] &\circ (\cprod{F(\mcurry{f})}{\id[FA]})
        = F(\mev[A,B]) \circ (\cprod{F(\mcurry{f})}{F(\id[A])}) \\
       &= F(\mev[A,B]) \circ F(\cprod{\mcurry{f}}{\id[A]})
        = F(\mev[A,B] \circ (\cprod{\mcurry{f}}{\id[A]}))
        = Ff.
     \end{align*}
      Hence, by the universal property of exponentials, $F(\mcurry{f}) = \mcurry{Ff}$. \qedhere
\end{enum}
\end{proof}
\end{prop}

The class of strict cartesian closed functors includes the identity functors and is closed under composition. Hence:

\begin{defn} \label{def:cat-ccc}
Cartesian closed categories and strict cartesian closed functors between them form a category $\CCC$.
\end{defn}

%TODO add some text here about weak preservation?

\begin{defn} \label{def:weak-preserve-terminal}
Let $F : \cat{C} \to \cat{D}$ be a functor and suppose that $T$ is a terminal object in $\cat{C}$. We say that $F$ \emph{weakly preserves the terminal object $T$} if $FT$ is terminal in $\cat{D}$.
\end{defn}

\begin{defn} \label{def:weak-preserve-products}
Let $F : \cat{C} \to \cat{D}$ be a functor.
\begin{enum}
\item Suppose that
% https://q.uiver.app/#q=WzAsMyxbMCwwLCJBIl0sWzEsMCwiUCJdLFsyLDAsIkIiXSxbMSwwLCJwXzEiLDJdLFsxLDIsInBfMiJdXQ==
\[\begin{tikzcd}
	A & P & B
	\arrow["{p_1}"', from=1-2, to=1-1]
	\arrow["{p_2}", from=1-2, to=1-3]
\end{tikzcd}\]
is a product of $A$ and $B$ in $\cat{C}$. We say that $F$ \emph{weakly preserves this product} if
% https://q.uiver.app/#q=WzAsMyxbMCwwLCJGQSJdLFsxLDAsIkZQIl0sWzIsMCwiRkIiXSxbMSwwLCJGcF8xIiwyXSxbMSwyLCJGcF8yIl1d
\[\begin{tikzcd}
	FA & FP & FB
	\arrow["{Fp_1}"', from=1-2, to=1-1]
	\arrow["{Fp_2}", from=1-2, to=1-3]
\end{tikzcd}\]
is a product of $FA$ and $FB$ in $\cat{D}$.
\item We say that $F$ \emph{weakly preserves products} if $F$ weakly preserves all products that exist.
\end{enum}
\end{defn}

%TODO add some text here about the following notations?
%TODO is notation the right environment?
%TODO: are these at the appropriate place?
\begin{notn}
If $\cat{C}$ and $\cat{D}$ have chosen terminal objects and the functor $F : \cat{C} \to \cat{D}$ weakly preserves the terminal object, then the unique map
\[ \mterm[F(\1[\cat{C}])] : F(\1[\cat{C}]) \to \1[\cat{D}] \]
is an isomorphism. We denote its inverse by $\inv{\mterm[F\1]}$.
\end{notn}

\begin{notn}
If $\cat{C}$ and $\cat{D}$ have chosen products and $F : \cat{C} \to \cat{D}$ is a functor, then there is a canonical map
% https://q.uiver.app/#q=WzAsMixbMCwwLCJGKFxcY3Byb2R7QX17Qn0pIl0sWzMsMCwiXFxjcHJvZHtGQX17RkJ9Il0sWzAsMSwiXFxtcGFpcntGKFxcbWZzdCl9e0YoXFxtc25kKX0iXV0=
\[\begin{tikzcd}
	{F(\cprod{A}{B})} &&& {\cprod{FA}{FB}}
	\arrow["{\mpair{F(\mfst)}{F(\msnd)}}", from=1-1, to=1-4]
\end{tikzcd}\]
denoted by $\prodcmp[A,B][F]$, which is natural in $A$ and $B$. Naturality means that if $f : A \to A'$ and $g : B \to B'$, then the square
% https://q.uiver.app/#q=WzAsNCxbMCwwLCJGKFxcY3Byb2R7QX17Qn0pIl0sWzEsMCwiXFxjcHJvZHtGQX17RkJ9Il0sWzAsMSwiRihcXGNwcm9ke0EnfXtCJ30pIl0sWzEsMSwiXFxjcHJvZHtGQSd9e0ZCJ30iXSxbMCwxLCJcXHByb2RjbXBbQSxCXVtGXSJdLFsyLDMsIlxccHJvZGNtcFtBJyxCJ11bRl0iLDJdLFswLDIsIkYoXFxjcHJvZHtmfXtnfSkiLDJdLFsxLDMsIlxcY3Byb2R7RmZ9e0ZnfSJdXQ==
\[\begin{tikzcd}
	{F(\cprod{A}{B})} & {\cprod{FA}{FB}} \\
	{F(\cprod{A'}{B'})} & {\cprod{FA'}{FB'}}
	\arrow["{\prodcmp[A,B][F]}", from=1-1, to=1-2]
	\arrow["{\prodcmp[A',B'][F]}"', from=2-1, to=2-2]
	\arrow["{F(\cprod{f}{g})}"', from=1-1, to=2-1]
	\arrow["{\cprod{Ff}{Fg}}", from=1-2, to=2-2]
\end{tikzcd}\]
commutes. If $F$ weakly preserves products, then $\prodcmp[A,B][F]$ is an isomorphism and we denote its inverse by $\inv*{\prodcmp[A,B][F]}$.
\end{notn}

%TODO parentheses around exponential?
\begin{notn}
If $\cat{C}$ and $\cat{D}$ have chosen exponentials and $F : \cat{C} \to \cat{D}$ is a weakly product preserving functor, then there is a canonical map $F(\cexp{A}{B}) \to \cexp{FA}{FB}$ which is the exponential transpose of the composite
% https://q.uiver.app/#q=WzAsMyxbMCwwLCJcXGNwcm9ke0YoXFxjZXhwe0F9e0J9KX17RkF9Il0sWzQsMCwiRkIiXSxbMiwwLCJGKFxcY3Byb2R7KFxcY2V4cHtBfXtCfSl9e0F9KSJdLFsyLDEsIkYoXFxtZXYpIl0sWzAsMiwiXFxpbnYqe1xccHJvZGNtcFtcXGNleHB7QX17Qn0sQV1bRl19Il1d
\[\begin{tikzcd}
	{\cprod{F(\cexp{A}{B})}{FA}} && {F(\cprod{(\cexp{A}{B})}{A})} && FB
	\arrow["{F(\mev)}", from=1-3, to=1-5]
	\arrow["{\inv*{\prodcmp[\cexp{A}{B},A][F]}}", from=1-1, to=1-3]
\end{tikzcd}\]
We denote this canonical map by $\expcmp[A,B][F]$.
\end{notn}

As usual, the subscripts and superscripts in $\prodcmp[A,B][F]$ and $\expcmp[A,B][F]$ may be omitted.

\section{Presheaves and representables}
%TODO think about the order of results

%TODO make this a definition?
Given two categories $\cat{C}$ and $\cat{D}$, the collections of functors $\cat{C} \to \cat{D}$ and natural transformations between them form a category $\funccat{\cat{C}}{\cat{D}}$ called a \textit{functor category}.

%TODO should $\cat{C}$ be small?
\begin{defn} \label{def:presheaves}
Let $\cat{C}$ be a small category.
\begin{enum}
\item A \emph{presheaf on $\cat{C}$} is a functor $\op{\cat{C}} \to \Set$.
\item The \emph{category of presheaves on $\cat{C}$} is the functor category $\funccat{\op{\cat{C}}}{\Set}$ and is denoted by $\PSh{\cat{C}}$.
\end{enum}
\end{defn}

%TODO add some text here about presheaves in general?

%TODO smallness
%TODO rephrase (hard to read)? change notation?
\begin{defn} \label{def:yoneda-embedding}
For any small category $\cat{C}$, there is a functor
\[ \y : \cat{C} \to \PSh{\cat{C}}, \]
called the \emph{Yoneda-embedding}, defined as follows. It sends an object $C \in \cat{C}$ to the presheaf $\yap{C}$ with $\yap{C}(X) = \Hom{X}{C}$ for each object $X \in \cat{C}$ and where for each morphism $f : Y \to X$ the operation $\yap{C}(f) : \yap{C}(X) \to \yap{C}(Y)$ is given by $\yap{C}(f)(g) = g \circ f$. Furthermore, it sends a morphism $g : C \to D$ to the natural transformation $\yap{g} : \yap{C} \to \yap{D}$ with components $(\yap{g})_X : \yap{C}(X) \to \yap{D}(X)$ given by $(\yap{g})_X(f) = g \circ f$.
\end{defn}

\begin{prop} \label{prop:yoneda-is-embedding}
The Yoneda-embedding $\y$ is fully faithful.
\begin{proof}
For the proof, see \cite{mac2013categories}.
\end{proof}
\end{prop}

\begin{defn} \label{def:reindexing}
Let $F : \cat{C} \to \cat{D}$ be a functor. The \emph{reindexing functor} or \emph{precomposition functor} $\reind{F} : \PSh{\cat{D}} \to \PSh{\cat{C}}$ maps a presheaf $P : \op{\cat{D}} \to \Set$ to $P\op{F} : \op{\cat{C}} \to \Set$ and it maps a natural transformation $\mu : P \to Q$ to the whiskering $\mu\op{F} : P\op{F} \to Q\op{F}$.
\end{defn}

%TODO make nicer
\begin{defn} \label{def:presheaf-cc-structure}
\hfill \vspace{-3pt}
\begin{items}
    \item The terminal presheaf $\1$ is defined by $\1(C) = \singset$ and $\1(f) = \id[\singset]$.
    \item The product $\cprod{P}{Q}$ of presheaves $P$ and $Q$ is pointwise: $(\cprod{P}{Q})(C) = \cprod{PC}{QC}$ and $(\cprod{P}{Q})(f) = \cprod{Pf}{Qf}$. The projections $\mfst[P,Q] : \cprod{P}{Q} \to P$ and $\msnd[P,Q] : \cprod{P}{Q} \to Q$ are also pointwise: $(\mfst[P,Q])_C = \mfst[PC,QC]$ and $(\msnd[P,Q])_C = \msnd[PC,QC]$.
    \item The exponential $\cexp{P}{Q}$ of presheaves $P$ and $Q$ is defined as follows: $(\cexp{P}{Q})(C) = \Hom{\cprod{\yap{C}}{P}}{Q}$ and $(\cexp{P}{Q})(f)(\sigma) = \sigma \circ (\cprod{y_f}{\id[P]})$. The evaluation $\mev[P,Q] : \cprod{(\cexp{P}{Q})}{P} \to Q$ is given by $(\mev[P,Q])_C(\sigma, x) = \sigma_C(\id[C], x)$.
\end{items}
\end{defn}

%TODO rephrase statement of proposition?
\begin{prop} \label{prop:presheaf-cat-is-ccc}
The category $\PSh{\cat{C}}$ is cartesian closed with terminal object, products, and exponentials as in Definition~\ref{def:presheaf-cc-structure}.
\begin{proof}
For a full proof, we refer the reader to standard category theory literature, e.g. \cite{maclane:moerdijk, leinster:basic-ct}. Here, we simply note the following. The pairing $\mpair{\mu}{\nu} : X \to \cprod{P}{Q}$ of natural transformations $\mu : X \to P$ and $\nu : X \to Q$ is pointwise: $\mpair{\mu}{\nu}_C = \mpair{\mu_C}{\nu_C}$. The exponential transpose $\mcurry{\mu} : X \to \cexp{P}{Q}$ of a natural transformation $\mu : \cprod{X}{P} \to Q$ is given by $\mcurry{\mu}_C(x)_D(h, a) = \mu_D(P(h)(x), a)$.
\end{proof}
\end{prop}

\begin{defn} \label{def:presheaf-cat-pullback-structure}
Let $\mu : Q \to S$ and $\nu : R \to S$ be morphisms of presheaves. The pullback of $\mu$ and $\nu$ is the presheaf $P$ given by
\begin{align*}
PC &= \setof{(x, y) \in \cartprod{QC}{RC}}{\mu_C(x) = \nu_C(y)} \\
(Pf)(x, y) &= ((Qf)(x), (Rf)(y))
\end{align*}
together with projections $p_1 : P \to Q$ and $p_2 : P \to R$ given by $(p_1)_C(x, y) = x$ and $(p_2)_C(x, y) = y$.
\end{defn}

\begin{prop} \label{prop:presheaf-cat-pullbacks}
The category $\PSh{\cat{C}}$ has pullbacks as given in Definition~\ref{def:presheaf-cat-pullback-structure}.
\begin{proof}
We again refer the reader to basic category theory literature, e.g. \cite{leinster:basic-ct}.
\end{proof}
\end{prop}

Note that the construction of pullbacks in $\PSh{\cat{C}}$ is also pointwise. That is, the pullback $P$ of $\mu : Q \to S$ and $\nu : R \to S$ in $\PSh{\cat{C}}$ evaluated at $\cat{C}$ is the pullback of $\mu_C : QC \to SC$ and $\nu_C : RC \to SC$ in $\Set$.

\begin{prop} \label{prop:yoneda-preservation}
The Yoneda-embedding weakly preserves the terminal object and products.
\begin{proof}
The proof follows from reformulating the universal property of terminal objects and products by stating that we have a natural bijection on the hom-sets.
\end{proof}
\end{prop}

\begin{prop} \label{prop:reindexing-preservation}
Let $F : \cat{C} \to \cat{D}$ be a functor. The reindexing functor $\reind{F} : \PSh{\cat{D}} \to \PSh{\cat{C}}$ weakly preserves the terminal object and products.
\begin{proof}
The reindexing functor $\reind{F} : \PSh{\cat{D}} \to \PSh{\cat{C}}$ has a left adjoint given by Kan-extension (\cite{mac2013categories}, Chapter X, Section 4, Theorem 1). Hence, it weakly preserves limits, in particular, the terminal object and products.
\end{proof}
\end{prop}

\begin{defn} \label{def:representation}
Let $P : \op{\cat{C}} \to \Set$ be a presheaf.
\begin{enum}
\item We say that $P$ is \emph{representable} if there exists an object $C \in \cat{C}$ such that $\yap{C} \cong P$.
\item If $C \in \cat{C}$ and $\rho : \yap{C} \to P$ is an isomorphism, then we say that the pair $(C, \rho)$ is a \emph{representation} of $P$.
\end{enum}
\end{defn}

Note the difference between the two parts of Definition~\ref{def:representation}. A representation of a presheaf $P$ is \textit{extra structure}, namely the pair ($C, \rho)$, attached to $P$. In contrast, representability is a property of a presheaf $P$, stating that there \textit{merely} exists a representation of $P$. The distinction is analogous to the discussion regarding CCCs with structure versus CCCs with property (Remark~\ref{rem:ccc-structure-vs-property}).

\begin{rem}
%TODO example of such a situation
There may be multiple representations of the same presheaf.
%TODO state this more precisely? prove it?
However, it follows from the properties of the Yoneda-embedding (Proposition~\ref{prop:yoneda-is-embedding}) that any two representations are isomorphic in an appropriate sense.
Hence, if a presheaf is representable, then its representation is \textit{essentially unique}, meaning unique up to isomorphism.
%TODO add this remark?
%Therefore, we also say that a representation is a \textit{property-like structure}.
\end{rem}

\section{Comma categories}

%TODO more introductory text?
%The material in this section is used in Chapter~\ref{chap:gluing}.

%TODO ugly notation for projections
\begin{defn} \label{def:comma-category}
Let
% https://q.uiver.app/#q=WzAsMyxbMCwxLCJcXGNhdHtCfSJdLFsxLDEsIlxcY2F0e0N9Il0sWzEsMCwiXFxjYXR7QX0iXSxbMCwxLCJHIiwyXSxbMiwxLCJGIl1d
\[\begin{tikzcd}
	& {\cat{A}} \\
	{\cat{B}} & {\cat{C}}
	\arrow["G"', from=2-1, to=2-2]
	\arrow["F", from=1-2, to=2-2]
\end{tikzcd}\]
be a diagram of categories and functors. The \emph{comma category} $\comma{F}{G}$ is defined as follows:
\begin{items}
    \item objects are triples $(A, B, p)$ where $A \in \cat{A}$, $B \in \cat{B}$, and $p : FA \to GB$;
    \item morphisms $(A, B, p) \to (A', B', p')$ are pairs $(f, g)$ with $f : A \to A'$ and $g : B \to B'$ such that the square
    % https://q.uiver.app/#q=WzAsNCxbMCwwLCJGQSJdLFsxLDAsIkZBJyJdLFswLDEsIkdCIl0sWzEsMSwiR0InIl0sWzAsMSwiRmYiXSxbMiwzLCJHZyIsMl0sWzAsMiwicCIsMl0sWzEsMywicCciXV0=
    \[\begin{tikzcd}
    	FA & {FA'} \\
    	GB & {GB'}
    	\arrow["Ff", from=1-1, to=1-2]
    	\arrow["Gg"', from=2-1, to=2-2]
    	\arrow["p"', from=1-1, to=2-1]
    	\arrow["{p'}", from=1-2, to=2-2]
    \end{tikzcd}\]
    commutes;
    \item identities and composition are componentwise.
\end{items}
\end{defn}

%TODO are \commatrd and the diagram necessary?
There is a projection functor $\commafst{F}{G} : \comma{F}{G} \to \cat{A}$ sending $(A, B, p)$ to $A$ and $(f, g)$ to $f$. Similarly, $\commasnd{F}{G} : \comma{F}{G} \to \cat{B}$ projects the components in $\cat{B}$. Furthermore, there is a natural transformation $\commatrd{F}{G} : F\commafst{F}{G} \to G\commasnd{F}{G}$ given by $(\commatrd{F}{G})_{(A, B, p)} = p$. These objects can be organized into a diagram
% https://q.uiver.app/#q=WzAsNCxbMCwwLCJcXGNvbW1he0Z9e0d9Il0sWzEsMCwiXFxjYXR7QX0iXSxbMSwxLCJcXGNhdHtDfSJdLFswLDEsIlxcY2F0e0J9Il0sWzMsMiwiRyIsMl0sWzEsMiwiRiJdLFswLDMsIlxcY29tbWFzbmR7Rn17R30iLDJdLFswLDEsIlxcY29tbWFmc3R7Rn17R30iXSxbMSwzLCJcXGNvbW1hdHJke0Z9e0d9IiwwLHsic2hvcnRlbiI6eyJzb3VyY2UiOjMwLCJ0YXJnZXQiOjMwfSwibGV2ZWwiOjJ9XV0=
\[\begin{tikzcd}
	{\comma{F}{G}} & {\cat{A}} \\
	{\cat{B}} & {\cat{C}}
	\arrow["G"', from=2-1, to=2-2]
	\arrow["F", from=1-2, to=2-2]
	\arrow["{\commasnd{F}{G}}"', from=1-1, to=2-1]
	\arrow["{\commafst{F}{G}}", from=1-1, to=1-2]
	\arrow["{\commatrd{F}{G}}", shorten <=8pt, shorten >=8pt, Rightarrow, from=1-2, to=2-1]
\end{tikzcd}\]

There are important special cases of the comma category construction for which we introduce notation.
\begin{notn}
\hfill \vspace{-3pt}
\begin{items}
\item If $F$ or $G$ is the identity functor $\idfunc[\cat{C}]$, then we replace the name of the functor by the category $\cat{C}$. For instance, $\comma{\cat{C}}{G}$ stands for $\comma{\idfunc[\cat{C}]}{G}$.
\item If the domain of $F$ or $G$ is the terminal category $\1$, then the functor can be identified with an object $X$ in $\cat{C}$, and we use the object in place of the functor. For instance, $\comma{F}{X}$ stands for $\comma{F}{G}$ where $G : \1 \to \cat{C}$ sends the unique object of $\1$ to $X$.
\end{items}
\end{notn}

\begin{lem}
Suppose $\cat{C}$ and $\cat{D}$ are categories with a terminal object and $F : \cat{C} \to \cat{D}$ weakly preserves the terminal object. Then $(\1[\cat{D}], \1[\cat{C}], \inv{\mterm[F\1]})$ is a terminal object in $\comma{\cat{D}}{F}$.
\begin{proof}
Given an object $(X, A, p) \in \comma{\cat{D}}{F}$, we have a morphism
% https://q.uiver.app/#q=WzAsNCxbMSwwLCJcXDEiXSxbMSwxLCJGXFwxIl0sWzAsMCwiWCJdLFswLDEsIkZBIl0sWzAsMSwiXFxpbnZ7XFxtdGVybVtGXFwxXX0iXSxbMiwzLCJwIiwyXSxbMywxLCJGKFxcbXRlcm1bQV0pIiwyXSxbMiwwLCJcXG10ZXJtW1hdIl1d
\[\begin{tikzcd}
	X & \1 \\
	FA & F\1
	\arrow["{\inv{\mterm[F\1]}}", from=1-2, to=2-2]
	\arrow["p"', from=1-1, to=2-1]
	\arrow["{F(\mterm[A])}"', from=2-1, to=2-2]
	\arrow["{\mterm[X]}", from=1-1, to=1-2]
\end{tikzcd}\]
in $\comma{\cat{D}}{F}$. The square commutes since $F\1$ is terminal. The uniqueness of this morphism follows from the fact that its components are maps to terminal objects.
\end{proof}
\end{lem}

\begin{comment}
Given an object $(X, A, p) \in \comma{\cat{D}}{G}$, we have a morphism
% https://q.uiver.app/#q=WzAsNCxbMCwwLCJYIl0sWzEsMCwiR1xcMSJdLFsxLDEsIkdcXDEiXSxbMCwxLCJHQSJdLFsxLDIsIlxcaWQiXSxbMCwzLCJwIiwyXSxbMCwxLCJHKFxcbXRlcm1bQV0pIFxcY2lyYyBwIl0sWzMsMiwiRyhcXG10ZXJtW0FdKSIsMl1d
\[\begin{tikzcd}
	X & G\1 \\
	GA & G\1
	\arrow["\id", from=1-2, to=2-2]
	\arrow["p"', from=1-1, to=2-1]
	\arrow["{G(\mterm[A]) \circ p}", from=1-1, to=1-2]
	\arrow["{G(\mterm[A])}"', from=2-1, to=2-2]
\end{tikzcd}\]
in $\comma{\cat{D}}{G}$. Furthermore, if
% https://q.uiver.app/#q=WzAsNCxbMCwwLCJYIl0sWzAsMSwiR0EiXSxbMSwwLCJHXFwxIl0sWzEsMSwiR1xcMSJdLFsyLDMsIlxcaWRbR1xcMV0iXSxbMCwyLCJmIl0sWzEsMywiR2ciLDJdLFswLDEsInAiLDJdXQ==
\[\begin{tikzcd}
	X & G\1 \\
	GA & G\1
	\arrow["{\id[G\1]}", from=1-2, to=2-2]
	\arrow["f", from=1-1, to=1-2]
	\arrow["Gg"', from=2-1, to=2-2]
	\arrow["p"', from=1-1, to=2-1]
\end{tikzcd}\]
is any morphism, then $g = \mterm[A]$, so $f = G(\mterm[A]) \circ p$ by commutativity of the diagram.
\end{comment}

\begin{lem}
Suppose $\cat{C}$ and $\cat{D}$ are categories with products and $F : \cat{C} \to \cat{D}$ weakly preserves products. Then the product of $(X, A, p)$ and $(Y, B, q)$ in $\comma{\cat{D}}{F}$ is
%TODO make this a diagram in $\cat{D}$?
% https://q.uiver.app/#q=WzAsMyxbMywwLCIoXFxjcHJvZHtYfXtZfSwgXFxjcHJvZHtBfXtCfSwgcikiXSxbNiwwLCIoWSwgQiwgcSkiXSxbMCwwLCIoWCwgQSwgcCkiXSxbMCwyLCIoXFxtZnN0W1gsWV0sXFw7XFxtZnN0W0EsQl0pIiwyXSxbMCwxLCIoXFxtc25kW1gsWV0sXFw7XFxtc25kW0EsQl0pIl1d
\[\begin{tikzcd}
	{(X, A, p)} &&& {(\cprod{X}{Y}, \cprod{A}{B}, r)} &&& {(Y, B, q)}
	\arrow["{(\mfst[X,Y],\;\mfst[A,B])}"', from=1-4, to=1-1]
	\arrow["{(\msnd[X,Y],\;\msnd[A,B])}", from=1-4, to=1-7]
\end{tikzcd}\]
where $r$ is the composite
% https://q.uiver.app/#q=WzAsMyxbMCwwLCJcXGNwcm9ke1h9e1l9Il0sWzEsMCwiXFxjcHJvZHtGQX17RkJ9Il0sWzIsMCwiRihcXGNwcm9ke0F9e0J9KSJdLFswLDEsIlxcY3Byb2R7cH17cX0iXSxbMSwyLCJcXGludntcXHByb2RjbXBbQSxCXX0iXV0=
\[\begin{tikzcd}
	{\cprod{X}{Y}} & {\cprod{FA}{FB}} & {F(\cprod{A}{B})}
	\arrow["{\cprod{p}{q}}", from=1-1, to=1-2]
	\arrow["{\inv{\prodcmp[A,B]}}", from=1-2, to=1-3]
\end{tikzcd}\]
\begin{proof}
%TODO spell this out?
The first projection $(\mfst[X,Y], \mfst[A,B])$ is a morphism in $\comma{\cat{D}}{F}$ since
% https://q.uiver.app/#q=WzAsNSxbMCwwLCJcXGNwcm9ke1h9e1l9Il0sWzIsMCwiWCJdLFswLDEsIlxcY3Byb2R7RkF9e0ZCfSJdLFswLDIsIkYoXFxjcHJvZHtBfXtCfSkiXSxbMiwyLCJGQSJdLFsyLDMsIlxcaW52e1xccHJvZGNtcFtBLEJdfSIsMl0sWzAsMiwiXFxjcHJvZHtwfXtxfSIsMl0sWzAsMSwiXFxtZnN0W1gsWV0iXSxbMyw0LCJGKFxcbWZzdFtBLEJdKSIsMl0sWzIsNCwiXFxtZnN0W0ZBLEZCXSIsMV0sWzEsNCwicCJdXQ==
\[\begin{tikzcd}
	{\cprod{X}{Y}} && X \\
	{\cprod{FA}{FB}} \\
	{F(\cprod{A}{B})} && FA
	\arrow["{\inv{\prodcmp[A,B]}}"', from=2-1, to=3-1]
	\arrow["{\cprod{p}{q}}"', from=1-1, to=2-1]
	\arrow["{\mfst[X,Y]}", from=1-1, to=1-3]
	\arrow["{F(\mfst[A,B])}"', from=3-1, to=3-3]
	\arrow["{\mfst[FA,FB]}"{description}, from=2-1, to=3-3]
	\arrow["p", from=1-3, to=3-3]
\end{tikzcd}\]
commutes. The proof for the second projection $(\msnd[X,Y], \msnd[A,B])$ is similar.

Now suppose we have two morphisms $(f, a) : (Z, C, u) \to (X, A, p)$ and $(g, b) : (Z, C, u) \to (Y, B, q)$ in $\comma{\cat{D}}{F}$, i.e. commuting squares
% https://q.uiver.app/#q=WzAsOCxbMywwLCJaIl0sWzMsMSwiRkMiXSxbNCwwLCJZIl0sWzQsMSwiRkIiXSxbMCwwLCJaIl0sWzEsMCwiWCJdLFsxLDEsIkZBIl0sWzAsMSwiRkMiXSxbMCwxLCJ1IiwyXSxbMiwzLCJxIl0sWzAsMiwiZyJdLFsxLDMsIkZiIiwyXSxbNCw3LCJ1IiwyXSxbNSw2LCJwIl0sWzQsNSwiZiJdLFs3LDYsIkZhIiwyXV0=
\[\begin{tikzcd}
	Z & X && Z & Y \\
	FC & FA && FC & FB
	\arrow["u"', from=1-4, to=2-4]
	\arrow["q", from=1-5, to=2-5]
	\arrow["g", from=1-4, to=1-5]
	\arrow["Fb"', from=2-4, to=2-5]
	\arrow["u"', from=1-1, to=2-1]
	\arrow["p", from=1-2, to=2-2]
	\arrow["f", from=1-1, to=1-2]
	\arrow["Fa"', from=2-1, to=2-2]
\end{tikzcd}\]
The pairing $\mpair{(f, a)}{(g, b)} : (Z, C, u) \to (\cprod{X}{Y}, \cprod{A}{B}, r)$ is defined as $(\mpair{f}{g}, \mpair{a}{b})$.
%TODO spell this out?
This is a morphism since
% https://q.uiver.app/#q=WzAsNSxbMCwwLCJaIl0sWzAsMiwiRkMiXSxbMSwwLCJcXGNwcm9ke1h9e1l9Il0sWzEsMSwiXFxjcHJvZHtGQX17RkJ9Il0sWzEsMiwiRihcXGNwcm9ke0F9e0J9KSJdLFswLDEsInUiLDJdLFswLDIsIlxcbXBhaXJ7Zn17Z30iXSxbMiwzLCJcXGNwcm9ke3B9e3F9Il0sWzMsNCwiXFxpbnZ7XFxwcm9kY21wW0EsQl19Il0sWzEsNCwiRihcXG1wYWlye2F9e2J9KSIsMl0sWzEsMywiXFxtcGFpcntGYX17RmJ9IiwxXV0=
\[\begin{tikzcd}
	Z & {\cprod{X}{Y}} \\
	& {\cprod{FA}{FB}} \\
	FC & {F(\cprod{A}{B})}
	\arrow["u"', from=1-1, to=3-1]
	\arrow["{\mpair{f}{g}}", from=1-1, to=1-2]
	\arrow["{\cprod{p}{q}}", from=1-2, to=2-2]
	\arrow["{\inv{\prodcmp[A,B]}}", from=2-2, to=3-2]
	\arrow["{F(\mpair{a}{b})}"', from=3-1, to=3-2]
	\arrow["{\mpair{Fa}{Fb}}"{description}, from=3-1, to=2-2]
\end{tikzcd}\]
commutes.
%TODO show the remaining details?
Using the universal properties of $\cprod{X}{Y}$ and $\cprod{A}{B}$, it can be shown that it is the unique morphism satisfying the universal property of the product.
\end{proof}
\end{lem}

%TODO remark more general lemma? $\comma{\cat{D}}{F}$ has a terminal object already if $\cat{C}$ does and $F$ is any functor
%TODO remark another lemma? $\comma{\cat{D}}{F}$ has products if $\cat{C}$ does, $\cat{D}$ has pullbacks and $F$ is any functor

\begin{lem}
Suppose $\cat{C}$ and $\cat{D}$ are categories with exponentials, $\cat{D}$ has chosen pullbacks, and $F : \cat{C} \to \cat{D}$ weakly preserves products. Then the exponential of $(X, A, p)$ and $(Y, B, q)$ in $\comma{\cat{D}}{F}$ is
% https://q.uiver.app/#q=WzAsMixbMCwwLCJcXGNwcm9keyhSLCBcXGNleHB7QX17Qn0sIHIpfXsoWCwgQSwgcCl9Il0sWzQsMCwiKFksIEIsIHEpIl0sWzAsMSwiKFxcbWV2W1gsWV1cXDooXFxjcHJvZHtrfXtcXGlkfSksXFw7XFxtZXZbQSxCXSkiXV0=
\[\begin{tikzcd}
	{\cprod{(R, \cexp{A}{B}, r)}{(X, A, p)}} &&&& {(Y, B, q)}
	\arrow["{(\mev[X,Y]\:(\cprod{k}{\id}),\;\mev[A,B])}", from=1-1, to=1-5]
\end{tikzcd}\]
where $R$, $r$, and $k$ are given by the pullback diagram
% https://q.uiver.app/#q=WzAsNSxbMCwxLCJGKFxcY2V4cHtBfXtCfSkiXSxbMSwxLCJcXGNleHB7RkF9e0ZCfSJdLFsyLDEsIlxcY2V4cHtYfXtGQn0iXSxbMiwwLCJcXGNleHB7WH17WX0iXSxbMCwwLCJSIl0sWzAsMSwiXFxleHBjbXBbQSxCXSIsMl0sWzEsMiwiXFxjZXhwe3B9e1xcaWR9IiwyXSxbMywyLCJcXGNleHB7XFxpZH17cX0iXSxbNCwwLCJyIiwyXSxbNCwzLCJrIl1d
\begin{equation} \label{diag:comma-exp-pb}
\begin{tikzcd}
	R && {\cexp{X}{Y}} \\
	{F(\cexp{A}{B})} & {\cexp{FA}{FB}} & {\cexp{X}{FB}}
	\arrow["{\expcmp[A,B]}"', from=2-1, to=2-2]
	\arrow["{\cexp{p}{\id}}"', from=2-2, to=2-3]
	\arrow["{\cexp{\id}{q}}", from=1-3, to=2-3]
	\arrow["r"', from=1-1, to=2-1]
	\arrow["k", from=1-1, to=1-3]
\end{tikzcd}
\end{equation}
\begin{proof}
First, we check that the evaluation is a morphism in $\comma{\cat{D}}{F}$, i.e. that the diagram
% https://q.uiver.app/#q=WzAsNixbMCwwLCJcXGNwcm9ke1J9e1h9Il0sWzEsMCwiXFxjcHJvZHsoXFxjZXhwe1h9e1l9KX17WH0iXSxbMiwwLCJZIl0sWzAsMSwiXFxjcHJvZHtGKFxcY2V4cHtBfXtCfSl9e0ZBfSJdLFswLDIsIkYoXFxjcHJvZHsoXFxjZXhwe0F9e0J9KX17QX0pIl0sWzIsMiwiRkIiXSxbMCwxLCJcXGNwcm9ke2t9e1xcaWR9Il0sWzEsMiwiXFxtZXZbWCxZXSJdLFswLDMsIlxcY3Byb2R7cn17cH0iLDJdLFszLDQsIlxcaW52e1xccHJvZGNtcFtcXGNleHB7QX17Qn0sQV19IiwyXSxbMiw1LCJxIl0sWzQsNSwiRihcXG1ldltBLEJdKSIsMl1d
\[\begin{tikzcd}
	{\cprod{R}{X}} & {\cprod{(\cexp{X}{Y})}{X}} & Y \\
	{\cprod{F(\cexp{A}{B})}{FA}} \\
	{F(\cprod{(\cexp{A}{B})}{A})} && FB
	\arrow["{\cprod{k}{\id}}", from=1-1, to=1-2]
	\arrow["{\mev[X,Y]}", from=1-2, to=1-3]
	\arrow["{\cprod{r}{p}}"', from=1-1, to=2-1]
	\arrow["{\inv{\prodcmp[\cexp{A}{B},A]}}"', from=2-1, to=3-1]
	\arrow["q", from=1-3, to=3-3]
	\arrow["{F(\mev[A,B])}"', from=3-1, to=3-3]
\end{tikzcd}\]
commutes.
%TODO spell this out?
This follows from the pullback diagram (\ref{diag:comma-exp-pb}) by taking the exponential transposes of the two composites.

Next, suppose we have a morphism $(f, g) : \cprod{(Z, C, u)}{(X, A, p)} \to (Y, B, q)$ in $\comma{\cat{D}}{F}$, i.e. a commuting diagram
% https://q.uiver.app/#q=WzAsNSxbMCwwLCJcXGNwcm9ke1p9e1h9Il0sWzAsMSwiXFxjcHJvZHtGQ317RkF9Il0sWzAsMiwiRihcXGNwcm9ke0N9e0F9KSJdLFsxLDAsIlkiXSxbMSwyLCJGQiJdLFsxLDIsIlxcaW52e1xccHJvZGNtcFtDLEFdfSIsMl0sWzAsMSwiXFxjcHJvZHt1fXtwfSIsMl0sWzAsMywiZiJdLFszLDQsInEiXSxbMiw0LCJGZyIsMl1d
\[\begin{tikzcd}
	{\cprod{Z}{X}} & Y \\
	{\cprod{FC}{FA}} \\
	{F(\cprod{C}{A})} & FB
	\arrow["{\inv{\prodcmp[C,A]}}"', from=2-1, to=3-1]
	\arrow["{\cprod{u}{p}}"', from=1-1, to=2-1]
	\arrow["f", from=1-1, to=1-2]
	\arrow["q", from=1-2, to=3-2]
	\arrow["Fg"', from=3-1, to=3-2]
\end{tikzcd}\]
%TODO spell this out?
Taking the exponential transposes of the two composites, using the naturality of $\inv{\prodcmp}$, we get that the diagram
% https://q.uiver.app/#q=WzAsNixbMCwwLCJaIl0sWzAsMSwiRkMiXSxbMCwyLCJGKFxcY2V4cHtBfXtCfSkiXSxbMSwyLCJcXGNleHB7RkF9e0ZCfSJdLFsyLDIsIlxcY2V4cHtYfXtGQn0iXSxbMiwwLCJcXGNleHB7WH17WX0iXSxbMCwxLCJ1IiwyXSxbMSwyLCJGKFxcbWN1cnJ5e2d9KSIsMl0sWzIsMywiXFxleHBjbXBbQSxCXSIsMl0sWzMsNCwiXFxjZXhwe3B9e1xcaWR9IiwyXSxbNSw0LCJcXGNleHB7XFxpZH17cX0iXSxbMCw1LCJcXG1jdXJyeXtmfSJdXQ==
\[\begin{tikzcd}
	Z && {\cexp{X}{Y}} \\
	FC \\
	{F(\cexp{A}{B})} & {\cexp{FA}{FB}} & {\cexp{X}{FB}}
	\arrow["u"', from=1-1, to=2-1]
	\arrow["{F(\mcurry{g})}"', from=2-1, to=3-1]
	\arrow["{\expcmp[A,B]}"', from=3-1, to=3-2]
	\arrow["{\cexp{p}{\id}}"', from=3-2, to=3-3]
	\arrow["{\cexp{\id}{q}}", from=1-3, to=3-3]
	\arrow["{\mcurry{f}}", from=1-1, to=1-3]
\end{tikzcd}\]
commutes. Hence, applying the pullback property of (\ref{diag:comma-exp-pb}), we obtain a morphism $\bar{f} : Z \to R$ such that $k\bar{f} = \mcurry{f}$ and such that the diagram
% https://q.uiver.app/#q=WzAsNCxbMCwwLCJaIl0sWzEsMCwiUiJdLFsxLDEsIkYoXFxjZXhwe0F9e0J9KSJdLFswLDEsIkZDIl0sWzAsMSwiXFxiYXJ7Zn0iXSxbMSwyLCJyIl0sWzAsMywidSIsMl0sWzMsMiwiRihcXG1jdXJyeXtnfSkiLDJdXQ==
\[\begin{tikzcd}
	Z & R \\
	FC & {F(\cexp{A}{B})}
	\arrow["{\bar{f}}", from=1-1, to=1-2]
	\arrow["r", from=1-2, to=2-2]
	\arrow["u"', from=1-1, to=2-1]
	\arrow["{F(\mcurry{g})}"', from=2-1, to=2-2]
\end{tikzcd}\]
commutes. This shows that $(\bar{f}, \mcurry{g}) : (Z, C, u) \to (R, \cexp{A}{B}, r)$ is a morphism in $\comma{\cat{D}}{F}$.
%TODO show the remaining details?
Using the universal properties of $R$, $\cexp{X}{Y}$, and $\cexp{A}{B}$, it can be shown that this is the unique morphism satisfying the universal property of the exponential.
\end{proof}
\end{lem}

\begin{prop} \label{prop:gluing-category-ccc}
Let $\cat{C}$ and $\cat{D}$ be cartesian closed categories and $F : \cat{C} \to \cat{D}$ be a functor. Suppose $F$ weakly preserves the terminal object and products, and suppose $\cat{D}$ has chosen pullbacks. Then
\begin{enum}
\item the comma category $\comma{\cat{D}}{F}$ is cartesian closed, and
\item the projection functor $\commasnd{\cat{D}}{F} : \comma{\cat{D}}{F} \to \cat{C}$ is strict cartesian closed.
\end{enum} 
\begin{proof}
The first claim is a corollary of the previous three lemmas. The second claim follows immediately from the definitions of the terminal object, products, and exponentials in $\comma{\cat{D}}{F}$.
\end{proof}
\end{prop}


\chapter{Categorical semantics for the simply typed \texorpdfstring{$\lambda$}{lambda}-calculus} \label{chap:stlc-cat}
\begin{comment}
syntax
  1. well-scoped, well-typed terms and substitutions
  2. equations
  3. normal forms + neutral terms

semantics
  1. - scwfs: motivation, compare with syntax
     - category of scwfs: morphisms are structure preserving maps or translations
  2. initial scwf: what is syntax categorically, theorem + (sketch of) proof
  3. CCCs: going from CCC to scwf -> semantics in CCCs
\end{comment}

\begin{comment}
Chap: Categorical semantics for lambda-calculus
Main goals: category of models, syntax is initial model
Why? needed for gluing which is categorical -> forward pointer to gluing proof, backward pointer to introduction
However: we look at a different syntax to motivate scwfs and for comparison with the previous syntax
\end{comment}

%TODO really ugly notation both for syntax and for scwfs. What are better alternatives?

\begin{comment}
Forward pointers (notion of model/category of models, syntax is initial)

Key points
\begin{items}
\item why categorical semantics (provides a much wider range of interpretations for the $\lambda$-calculus, we need it for gluing proof which is categorical)
\item two ways of categorical semantics for $\lambda$-calculus (CCC, scwf), their relation (possible to convert between the two structures), why we use scwf (syntax of $\lambda$-calculus without product types is not an initial CCC)
\item different syntactic presentation, why (better suited for categorical treatment, motivation for scwfs i.e. notion of model, provide understanding by comparing it to previous syntax)
\end{items}
\end{comment}

In Chapter~\ref{chap:stlc}, we discussed the syntax and semantics of simply typed $\lambda$-calculus in a traditional way. In this chapter, we will discuss the categorical semantics of simply typed $\lambda$-calculus.

To motivate the categorical semantics, we first look at a different version of the syntax. This is done in Section~\ref{sec:stlc2-syntax}. There are two main differences between the new and the old syntax.
\begin{items}
    \item In the new syntax, we use de Bruijn indices instead of variable names. The goal of de Bruijn indices is to have a variable numbering scheme which eliminates the need for $\alpha$-conversion. With this scheme, each occurrence of a variable is replaced by the number of $\lambda$-binders between the occurrence of the variable and the binder the variable refers to. For example, the term
    \[ \lamv{x}{\functy{\sigma}{\tau}}{\lamv{y}{\sigma}{\app{x}{y}}} \]
    is expressed as
    \[ \lambda^{\functy{\sigma}{\tau}}.\;\lambda^\sigma.\;\app{1}{0} \]
    with de Bruijn indices.
    \item Substitutions are made explicit in the new syntax. That is, while in the old syntax, substitution was an operation defined on terms, in the new syntax, substitution in terms is a primitive term former. Accordingly, instead of defining a substitution as a mapping from variables to terms, there are rules for the formation of substitutions in the syntax. Moreover, there are two convertibility relations: one for terms, one for substitutions.
\end{items}

After presenting the new syntax, we discuss the categorical semantics of the simply typed $\lambda$-calculus, including notions of models and morphisms, in Section~\ref{sec:stlc2-semantics}. We also give a general method to construct models from cartesian closed categories.

\section{Simply typed \texorpdfstring{$\lambda$}{lambda}-calculus \`a la de Bruijn with explicit substitutions} \label{sec:stlc2-syntax}

\begin{comment}
Key points
\begin{items}
\item syntax (terms, substitutions, equations)
\item normal forms
\item comparison to previous syntax, explain similarities and differences
\end{items}
\end{comment}

\begin{comment}
Problem with traditional presentation: there are complications with substitution, making the formalism complicated (cf. remark in Chapter 2). Solution: use de Bruijn indices.

Problem: it is still quite technical to define substitution. Solution: make substitution part of the syntax (explicit substitutions). Also need typing for substitutions, and additional conversion laws for terms and substitutions.
Maybe motivation: having substitutions be part of the syntax (and not operations defined on the syntax) makes it clearer how to deal with them semantically.
\end{comment}

Types are as before (Definition~\ref{def:stlc-types}):
\[ \sigma, \tau ::= \beta \sep \functy{\sigma}{\tau} \]
where $\beta$ ranges over $\Basetypes$.

Contexts are defined as follows.

\begin{defn}[Context]
A \emph{context} is a list of types.
\end{defn}

\begin{notn}
\begin{enum}
\item We denote the set of contexts by $\sCon$ and range over its elements by $\Gamma, \Delta, \Theta, \Xi$.
%TODO notations for lists?
%TODO identify singleton list with its element?
\item The \emph{empty context} is denoted by $\sempcon$.
\item If $\Gamma \in \Con$ and $\sigma \in \Ty$, then we write $\sextcon{\Gamma}{\sigma}$ for the \emph{extended context} obtained by appending $\sigma$ at the end of the list $\Gamma$.
\end{enum}
\end{notn}

Now we define terms and substitutions.

\begin{defn}
%TODO better wording
We generate two families $\sTm{\Gamma}{\sigma}$ and $\sSub{\Delta}{\Gamma}$ of sets, called \emph{terms} and \emph{substitutions}, respectively, by mutual induction. Terms are indexed in contexts and types, substitutions are indexed in pairs of contexts. The rules are displayed in Figure~\ref{fig:stlc2-terms-subs}). We write $\stypedtm{t}{\Gamma}{\sigma}$ for $t \in \sTm{\Gamma}{\sigma}$ and $\stypedsub{\gamma}{\Delta}{\Gamma}$ for $\gamma \in \sSub{\Delta}{\Gamma}$.

\begin{figure}[ht]
\begin{mathpar}
\inferrule[vz]{}
    {\stypedtm{\vz[\Gamma][\sigma]}{\sextcon{\Gamma}{\sigma}}{\sigma}}
\and
\inferrule[app]
    {\stypedtm{t}{\Gamma}{\functy{\sigma}{\tau}} \and \stypedtm{u}{\Gamma}{\sigma}}
    {\stypedtm{\app{t}{u}}{\Gamma}{\tau}}
\and
\inferrule[lam]
    {\stypedtm{t}{\sextcon{\Gamma}{\sigma}}{\tau}}
    {\stypedtm{\slam{\sigma}{t}}{\Gamma}{\functy{\sigma}{\tau}}}
\and
\inferrule[subst]
    {\stypedtm{t}{\Gamma}{\sigma} \and \stypedsub{\gamma}{\Delta}{\Gamma}}
    {\stypedtm{\ssubst{t}{\gamma}}{\Delta}{\sigma}}
\and
\inferrule[emp]{}
    {\stypedsub{\sempsub[\Gamma]}{\Gamma}{\sempcon}}
\and
\inferrule[ext]
    {\stypedsub{\gamma}{\Delta}{\Gamma} \and \stypedtm{t}{\Delta}{\sigma}}
    {\stypedsub{\sextsub{\gamma}{t}}{\Delta}{\sextcon{\Gamma}{\sigma}}}
\and
\inferrule[id]{}
    {\stypedsub{\sidsub[\Gamma]}{\Gamma}{\Gamma}}
\and
\inferrule[comp]
    {\stypedsub{\gamma}{\Delta}{\Gamma} \and \stypedsub{\delta}{\Theta}{\Delta}}
    {\stypedsub{\scompsub{\gamma}{\delta}}{\Theta}{\Gamma}}
\and
\inferrule[proj]{}
    {\stypedsub{\spsub[\Gamma][\sigma]}{\sextcon{\Gamma}{\sigma}}{\Gamma}}
\end{mathpar}
\caption{Terms and substitutions}
\label{fig:stlc2-terms-subs}
\end{figure}
\end{defn}

\begin{comment}
%TODO compare to previous syntax, explain term and substitution formers
\begin{items}
\item variable names vs de Bruijn indices, explicit substitution
\item operations on substitutions
\end{items}

\begin{items}
\item Context is a list of types representing for each position their type.
\item A variable is a position in this context.
\end{items}
\end{comment}

As mentioned before, the main difference between this syntax and the previous one is that the new syntax uses de Bruijn indices instead of variable names, and that substitutions are part of the rules for generating the syntax. Note however, that only the zero de Bruijn index $\vz[\Gamma][\sigma]$ is part of the generators for the syntax. The other de Bruijn indices are expressed in terms of this and the weakening substitution $\spsub[\Gamma][\sigma]$ as follows.

\begin{defn}[De Bruijn indices]
%TODO better wording
%TODO just call them variables?
We define a subset $\sVar{\Gamma}{\sigma} \subs \sTm{\Gamma}{\sigma}$ of terms, called \emph{de Bruijn indices}, by induction. The rules are displayed in Figure~\ref{fig:stlc2-de-Bruijn-indices}. We write $\stypedvar{x}{\Gamma}{\sigma}$ for $x \in \sVar{\Gamma}{\sigma}$.

\begin{figure}[ht]
\begin{mathpar}
\inferrule[vz]{}{\stypedvar{\vz[\Gamma][\sigma]}{\sextcon{\Gamma}{\sigma}}{\sigma}}
\and
\inferrule[vs]
    {\stypedvar{x}{\Gamma}{\sigma}}
    {\stypedvar{\ssubst{x}{\spsub[\Gamma][\tau]}}{\sextcon{\Gamma}{\tau}}{\sigma}}
\end{mathpar}
\caption{Rules for de Bruijn indices}
\label{fig:stlc2-de-Bruijn-indices}
\end{figure}
\end{defn}

Thus, a de Bruijn index is of the form $\ssubst{\vz}{\spsub^n}$ for some natural number $n$, where $\spsub^n$ denotes an $n$-fold composition of $\spsub$ of the appropriate types. Such a de Bruijn index points to the $n$-th free variable in the context, counting from zero.

%TODO notation for de Bruijn indices?

%TODO ugly notation
%TODO better terminology?
\begin{notn} \label{not:special-subs}
\begin{enum}
\item For $\stypedtm{t}{\Gamma}{\sigma}$, let $\stmsub{t} \in \sSub{\Gamma}{[\sigma]}$ be the substitution $\sextsub{\sempsub[\Gamma]}{t}$.
\item For $\stypedtm{t}{\Gamma}{\sigma}$, let $\ssingsub{t} \in \sSub{\Gamma}{\sextcon{\Gamma}{\sigma}}$ denote the \emph{singleton substitution} $\sextsub{\sidsub[\Gamma]}{t}$.
\item For $\stypedsub{\gamma}{\Delta}{\Gamma}$ and $\sigma \in \Ty$, let $\sliftsub[\sigma]{\gamma} \in \sSub{\sextcon{\Delta}{\sigma}}{\sextcon{\Gamma}{\sigma}}$ denote the \emph{lifted substitution} $\sextsub{\scompsub{\gamma}{\spsub[\Delta][\sigma]}}{\vz[\Delta][\sigma]}$.
\end{enum}
\end{notn}

The convertibility relation is defined similarly to the traditional presentation. However, we also need to define convertibility for substitutions.

\begin{defn}[Convertibility]
%TODO better wording
We generate two families of equivalence relations
\[ \sconvreltm{\Gamma}{\sigma} \subs \cartprod{\sTm{\Gamma}{\sigma}}{\sTm{\Gamma}{\sigma}} \quad\text{and}\quad
    \sconvrelsub{\Delta}{\Gamma} \subs \cartprod{\sSub{\Delta}{\Gamma}}{\sSub{\Delta}{\Gamma}} \]
by mutual induction. The rules are displayed in Figure~\ref{fig:stlc2-equations}). We write $\sconvtm{t}{t'}{\Gamma}{\sigma}$ for $(t, t') \in \sconvreltm{\Gamma}{\sigma}$ and $\sconvsub{\gamma}{\gamma'}{\Delta}{\Gamma}$ for $(\gamma, \gamma') \in \sconvrelsub{\Delta}{\Gamma}$.

\begin{figure}[t!]
\begin{mathpar}
\inferrule[refl]
    {\stypedtm{t}{\Gamma}{\sigma}}
    {\sconvtm{t}{t}{\Gamma}{\sigma}}
\and
\inferrule[trans]
    {\sconvtm{t}{u}{\Gamma}{\sigma} \and \sconvtm{u}{v}{\Gamma}{\sigma}}
    {\sconvtm{t}{v}{\Gamma}{\sigma}}
\and
\inferrule[sym]
    {\sconvtm{t}{u}{\Gamma}{\sigma}}
    {\sconvtm{u}{t}{\Gamma}{\sigma}}
\and
\inferrule[cong-app]
    {\sconvtm{t}{t'}{\Gamma}{\functy{\sigma}{\tau}} \and \sconvtm{u}{u'}{\Gamma}{\sigma}}
    {\sconvtm{\app{t}{u}}{\app{t'}{u'}}{\Gamma}{\tau}}
\and
\inferrule[cong-lam]
    {\sconvtm{t}{t'}{\sextcon{\Gamma}{\sigma}}{\tau}}
    {\sconvtm{\slam{\sigma}{t}}{\slam{\sigma}{t'}}{\Gamma}{\functy{\sigma}{\tau}}}
\and
\inferrule[cong-subst]
    {\sconvtm{t}{t'}{\Gamma}{\sigma} \and \sconvsub{\gamma}{\gamma'}{\Delta}{\Gamma}}
    {\sconvtm{\ssubst{t}{\gamma}}{\ssubst{t'}{\gamma'}}{\Delta}{\sigma}}
\and
\inferrule[cong-ext]
    {\sconvsub{\gamma}{\gamma'}{\Delta}{\Gamma} \and \sconvtm{t}{t'}{\Delta}{\sigma}}
    {\sconvsub{\sextsub{\gamma}{t}}{\sextsub{\gamma'}{t'}}{\Delta}{\sextcon{\Gamma}{\sigma}}}
\and
\inferrule[cong-comp]
    {\sconvsub{\gamma}{\gamma'}{\Delta}{\Gamma} \and \sconvsub{\delta}{\delta'}{\Theta}{\Delta}}
    {\sconvsub{\scompsub{\gamma}{\delta}}{\scompsub{\gamma'}{\delta'}}{\Theta}{\Gamma}}
\and
\inferrule[beta]
    {\stypedtm{t}{\sextcon{\Gamma}{\sigma}}{\tau} \and \stypedtm{u}{\Gamma}{\sigma}}
    {\sconvtm{\app{(\slam{\sigma}{t})}{u}}{\ssubst{t}{\ssingsub{u}}}{\Gamma}{\tau}}
\and
\inferrule[eta]
    {\stypedtm{t}{\Gamma}{\functy{\sigma}{\tau}}}
    {\sconvtm{\slam{\sigma}{\app{\ssubst{t}{\spsub}}{\vz}}}{t}{\Gamma}{\functy{\sigma}{\tau}}}
\and
\inferrule[emp-eta]
    {\stypedsub{\gamma}{\Gamma}{\sempcon}}
    {\sconvsub{\gamma}{\sempsub[\Gamma]}{\Gamma}{\sempcon}}
\and
\inferrule[id-l]
    {\stypedsub{\gamma}{\Delta}{\Gamma}}
    {\sconvsub{\scompsub{\sidsub[\Gamma]}{\gamma}}{\gamma}{\Delta}{\Gamma}}
\and
\inferrule[id-r]
    {\stypedsub{\gamma}{\Delta}{\Gamma}}
    {\sconvsub{\scompsub{\gamma}{\sidsub[\Delta]}}{\gamma}{\Delta}{\Gamma}}
\and
\inferrule[comp-assoc]
    {\stypedsub{\gamma}{\Delta}{\Gamma} \and \stypedsub{\delta}{\Theta}{\Delta} \and \stypedsub{\theta}{\Xi}{\Theta}}
    {\sconvsub{\scompsub{(\scompsub{\gamma}{\delta})}{\theta}}{\scompsub{\gamma}{(\scompsub{\delta}{\theta})}}{\Xi}{\Gamma}}
\and
\inferrule[proj]
    {\stypedsub{\gamma}{\Delta}{\Gamma} \and \stypedtm{t}{\Delta}{\sigma}}
    {\sconvsub{\scompsub{\spsub}{(\sextsub{\gamma}{t})}}{\gamma}{\Delta}{\Gamma}}
\and
\inferrule[var-subst]
    {\stypedsub{\gamma}{\Delta}{\Gamma} \and \stypedtm{t}{\Delta}{\sigma}}
    {\sconvtm{\ssubst{\vz}{(\sextsub{\gamma}{t})}}{t}{\Delta}{\sigma}}
\and
\inferrule[app-subst]
    {\stypedtm{t}{\Gamma}{\functy{\sigma}{\tau}} \and \stypedtm{u}{\Gamma}{\sigma} \and \stypedsub{\gamma}{\Delta}{\Gamma}}
    {\sconvtm{\ssubst{(\app{t}{u})}{\gamma}}{\app{\ssubst{t}{\gamma}}{\ssubst{u}{\gamma}}}{\Delta}{\tau}}
\and
\inferrule[lam-subst]
    {\stypedtm{t}{\sextcon{\Gamma}{\sigma}}{\tau} \and \stypedsub{\gamma}{\Delta}{\Gamma}}
    {\sconvtm{\ssubst{(\slam{\sigma}{t})}{\gamma}}{\slam{\sigma}{\ssubst{t}{(\sliftsub[\sigma]{\gamma})}}}{\Delta}{\functy{\sigma}{\tau}}}
\and
\inferrule[subst-id]
    {\stypedtm{t}{\Gamma}{\sigma}}
    {\sconvtm{\ssubst{t}{(\sidsub[\Gamma])}}{t}{\Gamma}{\sigma}}
\and
\inferrule[subst-comp]
    {\stypedtm{t}{\Gamma}{\sigma} \and \stypedsub{\gamma}{\Delta}{\Gamma} \and \stypedsub{\delta}{\Theta}{\Delta}}
    {\sconvtm{\ssubst{t}{(\scompsub{\gamma}{\delta})}}{\ssubst{(\ssubst{t}{\gamma})}{\delta}}{\Theta}{\sigma}}
\and
\inferrule[ext-eta]
    {\stypedsub{\gamma}{\Delta}{\sextcon{\Gamma}{\sigma}}}
    {\sconvsub{\sextsub{\scompsub{\spsub}{\gamma}}{\ssubst{\vz}{\gamma}}}{\gamma}{\Delta}{\sextcon{\Gamma}{\sigma}}}
\end{mathpar}
\caption{Equations between terms and substitutions}
\label{fig:stlc2-equations}
\end{figure}
\end{defn}

%TODO list some additional useful equations that can be derived

Finally, we define normal forms and neutral terms. These are defined similarly to the traditional presentation.

\begin{defn}
%TODO better wording
We define two families $\sNe{\Gamma}{\sigma}$ and $\sNf{\Gamma}{\sigma}$ of subsets of $\sTm{\Gamma}{\sigma}$, called \emph{neutral terms} and \emph{normal forms}, respectively, by mutual induction. The rules are displayed in Figure~\ref{fig:stlc2-normal-forms}. We write $\stypedne{m}{\Gamma}{\sigma}$ for $m \in \sNe{\Gamma}{\sigma}$ and $\stypednf{n}{\Gamma}{\sigma}$ for $n \in \sNf{\Gamma}{\sigma}$.

\begin{figure}[ht]
\begin{mathpar}
\inferrule[var]
    {\stypedvar{x}{\Gamma}{\sigma}}
    {\stypedne{x}{\Gamma}{\sigma}}
\and
\inferrule[app]
    {\stypedne{m}{\Gamma}{\functy{\sigma}{\tau}} \and \stypednf{n}{\Gamma}{\sigma}}
    {\stypedne{\app{m}{n}}{\Gamma}{\tau}}
\\
\inferrule[shift]
    {\stypedne{m}{\Gamma}{\beta}}
    {\stypednf{m}{\Gamma}{\beta}}
\quad(\beta \in \Basetypes)
\and
\inferrule[lam]
    {\stypednf{n}{\sextcon{\Gamma}{\sigma}}{\tau}}
    {\stypednf{\slam{\sigma}{n}}{\Gamma}{\functy{\sigma}{\tau}}}
\end{mathpar}
\caption{Normal forms and neutral terms}
\label{fig:stlc2-normal-forms}
\end{figure}
\end{defn}

For the gluing proof, we also need neutral substitutions, which are essentially lists of neutral terms. Note that a substitution is equivalently given by a list of terms.

\begin{defn}
A substitution is neutral if every term in it is neutral.
\end{defn}

The identity substitution is convertible to a neutral by definition, and thus it is neutral as well.

\begin{prop}
The identity substitution is neutral.
\end{prop}
\begin{proof}
The identity substitution can be expressed as a list of variables, all of which are neutral terms.
\end{proof}

\section{Semantics} \label{sec:stlc2-semantics}

\begin{comment}
Key points
\begin{items}
\item notion of model: scwf, relate to syntax
\item morphisms of models, category of models
\item syntax is an initial model, abstract syntax: it is characterized by a universal model-theoretic property, independent of syntactic presentations (hides syntactic details, e.g. variable names vs de Bruijn indices, implicit vs explicit substitutions, preterms vs intrinsically typed/scoped terms), allows us work at an appropriate level of abstraction
\item CCCs are models (connect to previous chapter and to gluing proof)
\end{items}
\end{comment}

In this section, we introduce a notion of model for the simply typed $\lambda$ calculus which we call \textit{$\lambda$-domain}, based on simply typed categories with families. We also define morphisms of such models, leading to the category of $\lambda$-domains. Next, we construct an initial model from the syntax of $\lambda$-calculus. Finally, we show how every cartesian closed category gives rise to a $\lambda$-domain.

\subsection{Simply typed categories with families} \label{sec:scwf}

%TODO alternative presentation: define scwfs as a generalized algebraic structure (collection of sets/families and operations on those subject to equations) motivated by the syntax in the previous section, and remark that a more concise definition is possible.

Categories with families provide a nice categorical setting for dealing with type systems with variable binding, especially those with dependent types. They were originally invented by Dybjer \cite{dybjer:1996:types} to model a basic framework for dependent types (essentially, Martin-Löf type theory without any type formers). The axioms of categories with families thus serve to formalize the notions of context, substitution, and variables corresponding to the judgmental framework of dependent type theories.

%TODO reference for Martin-Löf's substitution calculus
%TODO how to punctuate the second sentence? -- or ; or , or .?
Categories with families can be seen as models of the structural components of dependent type theory. However, they lie rather close to the syntax of dependent type theory -- specifically, Martin-Löf's substitution calculus. Thus, they can act as an intermediary between syntax (formal systems) and semantics (categorical/algebraic notions). This double role enables one to prove equivalences between certain type theories and some classes of models for those type theories. For instance, see \cite{DBLP:journals/mscs/ClairambaultD14} for the case of Martin-Löf type theory, and \cite{castellan:2021:cwf} for results about various simpler type systems.

In this section, we introduce a kind of semantic domain (Definition~\ref{def:lambda-domain}) in which to interpret the simply typed $\lambda$-calculus. The notion is based on \textit{simply typed categories with families} (Definition~\ref{def:scwf}), a simplified version of categories with families, where types do not depend on contexts. This simplified version provides a notion of model for a bare-bones simple type system with no type formers and only variables as terms. As such, simply typed categories with families are a suitable basis for modelling more complex type systems. For instance, we shall see (Definition~\ref{def:function-structure}) how to incorporate function types, which are necessary to model $\lambda$-calculus.

For a more thorough introduction to categories with families in general, we refer the reader to \cite{dybjer:1996:types} and \cite{castellan:2021:cwf}.

\begin{defn}[Scwf] \label{def:scwf}
A \emph{simply typed category with families}, or \emph{scwf}, consists of:
\begin{enum}
    \item a category $\cat{C}$ with a terminal object $\1$,
    \item a set $\STy$,
    \item for every $A \in \STy$, a presheaf $\STm{A} : \op{\cat{C}} \to \Set$, and
    \item for every $\Gamma \in \cat{C}$ and $A \in \STy$, a representation of the presheaf $\cprod{\yap{\Gamma}}{\STm{A}} : \op{\cat{C}} \to \Set$.
\end{enum}
\end{defn}

Definition~\ref{def:scwf} is rather concise, but may be hard to understand for a reader not trained in category theory, also known as \textit{general abstract nonsense}\footnote{\url{https://en.wikipedia.org/wiki/Abstract_nonsense}}. It turns out that the data packed in the definition matches the syntax of simply typed $\lambda$-calculus presented in Section~\ref{sec:stlc2-syntax}, excluding function types, application, and abstraction. Let us unpack the definition to illustrate this point. The notation and terminology regarding scwfs also help make the connection clearer.

%TODO paragraphs inside the list? or just one paragraph for each item?
\begin{enum}
    \item The category $\cat{C}$ is called the \emph{base category} of the scwf. Its objects are referred to as \emph{contexts}, and its morphisms as \emph{substitutions}. For this reason, we also write $\SCon$ for $\Ob{\cat{C}}$ and $\SSub{\Delta}{\Gamma}$ for $\Hom[\cat{C}]{\Delta}{\Gamma}$. Since $\cat{C}$ is a category, we have the usual composition and identity operations on substitutions.
    
    The terminal object $\1$ is called the \emph{empty context} and written as $\Sempcon$. The unique map $\Gamma \to \Sempcon$ is called the \emph{empty substitution} and written as $\Sempsub[\Gamma]$.
    
    We use capital Greek letters ($\Gamma, \Delta, \Theta, \ldots$) to range over contexts, and lowercase Greek letters ($\gamma, \delta, \theta, \ldots$) to range over substitutions.

    \item The elements of the set $\STy$ are referred to as \emph{types}. We use uppercase Latin letters ($A, B, C, \ldots$) to range over types.
    
    \item For $\Gamma \in \SCon$ and $A \in \STy$, the elements of $\STm{A}(\Gamma)$ are called \emph{terms of type $A$ in context $\Gamma$}. The set $\STm{A}(\Gamma)$ is also written as $\STm[\Gamma]{A}$.
    
    For a substitution $\gamma : \Delta \to \Gamma$, the functorial action of $\STm{A}$ gives a function
    \[ \STm{A}(\gamma) : \STm[\Gamma]{A} \to \STm[\Delta]{A} \]
    referred to as \emph{substitution in terms}. If $t \in \STm[\Gamma]{A}$, then we write $\Ssubst{t}{\gamma}$ for $\STm{A}(\gamma)(t)$.

    We use lowercase Latin letters $(t, u, v, \ldots)$ to range over terms.

    \item Given $\Gamma \in \SCon$ and $A \in \STy$, the presheaf $\cprod{\yap{\Gamma}}{\STm{A}} : \op{\cat{C}} \to \Set$ sends a context $\Delta$ to the set $\cartprod{\SSub{\Delta}{\Gamma}}{\STm[\Delta]{A}}$ and a substitution $\delta : \Theta \to \Delta$ to the function
    \[
    \begin{array}{rcl}
    \cartprod{\SSub{\Delta}{\Gamma}}{\STm[\Delta]{A}},
        &\to& \cartprod{\SSub{\Theta}{\Gamma}}{\STm[\Theta]{A}} \\
    (\gamma, t) &\mapsto& (\Scompsub{\gamma}{\delta}, \Ssubst{t}{\delta}).
    \end{array}
    \]
    
    Thus, a representation of this presheaf amounts to a context $\Sextcon{\Gamma}{A}$ together with a substitution $\Spsub[\Gamma][A] : \Sextcon{\Gamma}{A} \to \Gamma$ and a term $\Sq[\Gamma][A] \in \STm[\Sextcon{\Gamma}{A}]{A}$ satisfying the following universal property:
    for every context $\Delta$, substitution $\gamma : \Delta \to \Gamma$, and term $t \in \STm[\Delta]{A}$, there exists a unique substitution $\Sextsub{\gamma}{t} : \Delta \to \Sextcon{\Gamma}{A}$ such that
    \[ \Scompsub{\Spsub[\Gamma][A]}{(\Sextsub{\gamma}{t})} = \gamma
        \quad\text{and}\quad
        \Ssubst{\Sq[\Gamma][A]}{\Sextsub{\gamma}{t}} = t. \]

    We say that the triple $(\Sextcon{\Gamma}{A}, \Spsub[\Gamma][A], \Sq[\Gamma][A])$ is a \emph{context comprehension} of $\Gamma$ and $A$. We also say that $\Sextcon{\Gamma}{A}$ is obtained by \emph{extending} the context $\Gamma$ with the type $A$.
\end{enum}

We emphasize that the constants and operations $\Sempcon$, $\STy$, $\STm{-}$, $\Sextcon{}{}$, $\Spsub$, $\Sq$, and $\Sextsub{}{}$ are part of the \textit{structure} of an scwf. Hence, an scwf is formally a tuple $(\cat{C}, \Sempcon, \STy, \STm{}, \Sextcon{}{}, \Spsub, \Sq, \Sextsub{}{})$. In this thesis, however, we refer to scwfs only by their base category, and leave the other parts of the structure implicit. If necessary for disambiguation, we indicate the base category for the other components. For instance, $\SCon[\cat{C}]$, $\SSub{\Delta}{\Gamma}[\cat{C}]$, $\STy[\cat{C}]$, and $\STm[\Gamma]{A}[\cat{C}]$ denote, respectively, the collections of contexts, substitutions, types, and terms of the scwf $\cat{C}$. As before, we often omit superscripts and subscripts (e.g. in $\Spsub[\Gamma][A]$) to simplify notation.

\begin{ex} \label{ex:set-scwf}
We define the scwf $\Set$ of sets as follows.
\begin{items}
    \item Its base category is the category $\Set$ of sets.
    \item Types are sets.
    \item Terms $\STm{\Gamma}{A}$ are functions $\Gamma \to A$. Substitution in terms is given by precomposition of functions.
    \item Context comprehension is given by cartesian products of sets and the projection functions.
\end{items}
\end{ex}

\begin{rem}
The notion of scwf can also be presented as a generalized algebraic theory \cite{cartmell:1986:apal}. The collections of contexts, substitutions, types, and terms become the sorts, and the empty context, empty substitution, substitution in terms, and context comprehension become the operations of the theory, subject to the equations arising from the functoriality of $\STm{A}$ and the universal properties of the terminal object and representability.

A full presentation of (dependently typed) cwfs as a generalized algebraic theory can be found in \cite{dybjer:1996:types}. A similar presentation for scwfs can be obtained by removing the dependency of types on terms. The resulting formal theory is essentially the same as the syntax of simply typed $\lambda$-calculus presented in Section~\ref{sec:stlc2-syntax} (again, excluding functions).
\end{rem}

%TODO same todos as for \ref{not:special-subs}
%TODO are all of these necessary?
The following notations are the semantic counterpart of Notation~\ref{not:special-subs}.
\begin{notn}
\begin{enum}
\item For $t \in \STm[\Gamma]{A}$, let $\Stmsub{t} : \Gamma \to \Sextcon{\Sempcon}{A}$ be the substitution $\Sextsub{\Sempsub[\Gamma]}{t}$.
\item For $t \in \STm[\Gamma]{A}$, let $\Ssingsub{t} : \Gamma \to \Sextcon{\Gamma}{A}$ denote the \emph{singleton substitution} $\Sextsub{\Sidsub[\Gamma]}{t}$.
\item For $\gamma : \Delta \to \Gamma$ and $A \in \STy$, let $\Sliftsub[A]{\gamma} : \Sextcon{\Delta}{A} \to \Sextcon{\Gamma}{A}$ denote the \emph{lifted substitution} $\Sextsub{(\Scompsub{\gamma}{\Spsub})}{\Sq}$.
\end{enum}
\end{notn}

To model type formers, we need to require additional structure on scwfs. This structure essentially consists of operations on types and terms, corresponding to the type formation, introduction, and elimination rules of type systems, satisfying certain equations corresponding to $\beta$ and optionally $\eta$-rules for the type former. For our minimalistic simply typed $\lambda$-calculus presented in Section~\ref{sec:stlc2-syntax}, it suffices to define what it means for an scwf to support function types.

%TODO what to call this? Castellan, Claraimbault and Dybjer call it \To-structure
%TODO notation for function type, maybe use \To?
\begin{defn}[Function-structure] \label{def:function-structure}
A function-structure on an scwf $\cat{C}$ consists of a type former $\Sfuncty{-}{-} : \cartprod{\STy}{\STy} \to \STy$ and for each $\Gamma \in \cat{C}$ and $A, B \in \STy$, term formers
\begin{align*}
\Sapp[\Gamma, A, B]{-}{-} &: \cartprod{\STm[\Gamma]{\Sfuncty{A}{B}}}{\STm[\Gamma]{A}} \to \STm[\Gamma]{B} \\
\Slam{A}[\Gamma, B]{-} &: \STm[\Sextcon{\Gamma}{A}]{B} \to \STm[\Gamma]{\Sfuncty{A}{B}}
\end{align*}
satisfying the equations
\begin{align}
\Sapp[\Gamma, A, B]{\Slam{A}[\Gamma, B]{s}}{u}
    &= \Ssubst{s}{\Ssingsub{u}} \label{eq:scwf-beta} \\
\Slam{A}[\Gamma, B]{\Sapp[\Sextcon{\Gamma}{A}, A, B]{\Ssubst{t}{\Spsub}}{\Sq}}
    &= t \label{eq:scwf-eta} \\
\Ssubst{(\Sapp[\Gamma, A, B]{t}{u})}{\gamma}
    &= \Sapp[\Delta, A, B]{\Ssubst{t}{\gamma}}{\Ssubst{u}{\gamma}} \label{eq:scwf-app-subst}
\end{align}
for all $s \in \STm[\Sextcon{\Gamma}{A}]{B}$, $t \in \STm[\Gamma]{\Sfuncty{A}{B}}$, $u \in \STm[\Gamma]{A}$, and $\gamma : \Delta \to \Gamma$.
\end{defn}

Often, we drop the subscripts of $\Slam{A}[\Gamma, B]{-}$ and $\Sapp[\Gamma, A, B]{}{}$ to improve readability.
%Occasionally, we may also omit the type of abstraction in $\Slam{A}{-}$ written as a superscript.

In Definition~\ref{def:function-structure}, the operation $\Sapp[\Gamma, A, B]{}{}$ corresponds to function application in $\lambda$-calculus, and $\Slam{A}[\Gamma,B]{-}$ corresponds to $\lambda$-abstraction. The first and second equations correspond to the $\beta$ and $\eta$-rules, respectively. The third equation describes how to perform a substitution in an application.

There is also a substitution law for abstraction: for each $t \in \STm[\Sextcon{\Gamma}{A}]{B}$ and $\gamma : \Delta \to \Gamma$, we have
\begin{equation} \label{eq:scwf-lam-subst}
\Ssubst{\Slam{A}[\Gamma, B]{t}}{\gamma} = \Slam{A}[\Delta, B]{\Ssubst{t}{\Sliftsub[A]{\gamma}}}.
\end{equation}
This law can be derived from the axioms of scwfs and function structures. Compare these equations with the corresponding rules in Figure~\ref{fig:stlc2-equations} (\textsc{beta}, \textsc{eta}, \textsc{app-subst}, and \textsc{lam-subst}).

%TODO elaborate more on this?
A function-structure on an scwf $\cat{C}$ is a tuple $(\Sfuncty{}{}$, $\Sapp{}{}$, $\Slam{}{-})$. To simplify notation, we usually omit the explicit reference to the function structure.

\begin{ex}
We define the function-structure on the scwf of sets as follows. The function type $\Sfuncty{A}{B}$ is given by the set of functions $\funcset{A}{B}$. The operations $\Slam{A}[\Gamma, B]{-}$ and $\Sapp[\Gamma,A,B]{}{}$ are given by currying and function application, respectively.
\end{ex}

We now introduce a notion of morphism between scwfs.

%TODO fix ugly notation
\begin{defn}[Strict scwf-morphism] \label{def:strict-scwf-morphism}
Let $\cat{C}$ and $\cat{D}$ be scwfs. A \emph{strict scwf-morphism} from $\cat{C}$ to $\cat{D}$ consists of
\begin{enum}
    \item a functor $F : \cat{C} \to \cat{D}$ between the base categories,
    \item a function $T : \STy[\cat{C}] \to \STy[\cat{D}]$, and
    \item for each $A \in \STy[\cat{C}]$, a natural transformation $\tau_A : \STm{A}[\cat{C}] \to \STm{TA}[\cat{D}] \circ F$
\end{enum}
such that $F$ strictly preserves the empty context and $\tau$ strictly preserves context comprehension.
\end{defn}

Again, the compactness of the categorical definition above might obscure the intuitive idea behind strict scwf-morphisms. Thus, we spell out Definition~\ref{def:strict-scwf-morphism} in detail while also introducing notation for the components of a strict scwf-morphism.
%TODO extremely verbose notation. Already start dropping the superscripts while spelling out the definition?
\begin{enum}
    \item We have a mapping $\scwfmorcon*{F} : \SCon[\cat{C}] \to \SCon[\cat{D}]$ of contexts, corresponding to the object part of the functor $F : \cat{C} \to \cat{D}$. The morphism part of $F$ is given by mappings $\scwfmorsub*{F}{\Delta}{\Gamma} : \SSub{\Delta}{\Gamma}[\cat{C}] \to \SSub{\scwfmorcon*{F}\Delta}{\scwfmorcon*{F}\Gamma}[\cat{D}]$ of substitutions for each $\Delta, \Gamma \in \SCon[\cat{C}]$.

    The functoriality of $F$ means that composition and identities are preserved: we have
    \[ \scwfmorsub*{F}{\Theta}{\Gamma}(\Scompsub{\gamma}{\delta})
        = \Scompsub{\scwfmorsub*{F}{\Delta}{\Gamma}(\gamma)}{\scwfmorsub*{F}{\Theta}{\Delta}(\delta)} \]
    for all $\gamma \in \SSub{\Delta}{\Gamma}[\cat{C}], \delta \in \SSub{\Theta}{\Delta}[\cat{C}]$, and
    \[ \scwfmorsub*{F}{\Gamma}{\Gamma}(\Sidsub[\Gamma]) = \Sidsub[\scwfmorcon*{F}\Gamma] \]
    for all $\Gamma \in \SCon[\cat{C}]$.

    \item The function $T : \STy[\cat{C}] \to \STy[\cat{D}]$ provides a mapping from the types of $\cat{C}$ to the types of $\cat{D}$ and is written as $\scwfmorty*{F}$.

    \item For each $A \in \STy[\cat{C}]$, the natural transformation $\tau_A : \STm{A}[\cat{C}] \to \STm{TA}[\cat{D}] \circ F$ is denoted by $\scwfmortm*{F}{A}$. Its components $(\tau_A)_\Gamma : \STm[\Gamma]{A}[\cat{C}] \to \STm[\scwfmorcon*{F}\Gamma]{\scwfmorty*{F}A}$ for $\Gamma \in \SCon[\cat{C}]$ are mappings of terms and are denoted by $\scwfmortm*{F}[\Gamma]{A}$. Naturality of $\scwfmortm*{F}{A}$ amounts to preservation of substitution in terms. That is, for all $t \in \STm[\Gamma]{A}[\cat{C}]$ and $\gamma \in \SSub{\Delta}{\Gamma}[\cat{C}]$, we have
    \[ \scwfmortm*{F}[\Delta]{A}(\Ssubst{t}{\gamma}) = \Ssubst{\scwfmortm*{F}[\Gamma]{A}(t)}{\scwfmorsub*{F}{\Delta}{\Gamma}(\gamma)}. \]

    \item Strict preservation of the empty context means that $\scwfmorcon*{F}(\Sempcon[\cat{C}]) = \Sempcon[\cat{D}]$.

    \item Finally, strict preservation of context comprehension means that
    \begin{align*}
    \scwfmorcon*{F}(\Sextcon{\Gamma}{A})
        &= \Sextcon{\scwfmorcon*{F}\Gamma}{\scwfmorty*{F}A} \\
    \scwfmorsub*{F}{\Sextcon{\Gamma}{A}}{\Gamma}(\Spsub[\Gamma][A])
        &= \Spsub[\scwfmorcon*{F}\Gamma][\scwfmorty*{F}A] \\
    \scwfmortm*{F}[\Sextcon{\Gamma}{A}]{A}(\Sq[\Gamma][A])
        &= \Sq[\scwfmorcon*{F}\Gamma][\scwfmorty*{F}A].
    \end{align*}
    We can also rephrase this condition by saying that, for all $\Gamma \in \SCon[\cat{C}]$ and $A \in \STy[\cat{C}]$, the triple
    \[ (\scwfmorcon*{F}(\Sextcon{\Gamma}{A}),
        \scwfmorsub*{F}{\Sextcon{\Gamma}{A}}{\Gamma}(\Spsub[\Gamma][A]),
        \scwfmortm*{F}{\Sextcon{\Gamma}{A}}{A}(\Sq[\Gamma][A])) \]
    is a context comprehension of $\scwfmorcon*{F}\Gamma \in \SCon[\cat{D}]$ and $\scwfmorty*{F}A \in \STy[\cat{D}]$.
\end{enum}

Similarly to scwfs, scwf-morphisms have several components. Formally, an scwf-morphism from $\cat{C}$ to $\cat{D}$ is a tuple $(F, T, \tau)$. By convention, we use the symbol for the base functor to denote all three components of an scwf-morphism. However, as demonstrated above, the notation can get quite verbose. To simplify notation, we drop the superscripts and subscripts most of the time, since they can usually be inferred from the argument of $F$.

Note that our notion of scwf-morphism is exactly the same as one would expect when viewing scwfs as generalized algebraic structures: they are structure preserving mappings between the sorts of the structures. If one thinks about the sorts (contexts, substitutions, types, terms) syntactically, then a good intuition for scwf-morphisms is that they are syntactic translations from one theory to another one.

\begin{defn} \label{def:scwf-morphism-preserves-function-structure}
Suppose $\cat{C}$ and $\cat{D}$ are scwfs with function-structures. An scwf-morphism $F : \cat{C} \to \cat{D}$ is said to \emph{strictly preserve the function-structure} if
\[ F(\Sfuncty{A}{B}) = \Sfuncty{FA}{FB} \quad\text{and}\quad
    F(\Sapp[\Gamma,A,B]{t}{u}) = \Sapp[F\Gamma,FA,FB]{Ft}{Fu} \]
for all $\Gamma \in \SCon[\cat{C}]$, $A, B \in \STy[\cat{C}]$, $t \in \STm[\Gamma]{\Sfuncty{A}{B}}$, and $u \in \STm[\Gamma]{A}$.
\end{defn}

In Definition~\ref{def:scwf-morphism-preserves-function-structure}, we only require $F$ to preserve the application operation. By the axioms of scwfs and function-structures, this already implies that it also preserves abstraction. That is, we have
\[ F(\Slam{A}[\Gamma,B]{t}) = \Slam{FA}[F\Gamma,FB]{Ft} \]
for all $t \in \STm[\Sextcon{\Gamma}{A}]{B}$.

Scwfs equipped with a function structure provide the main notion of semantic domain for the simply typed $\lambda$-calculus in this and the next chapter. Hence, we introduce the following terminology.

\begin{defn} \label{def:lambda-domain}
\begin{enum}
    \item A \emph{$\lambda$-domain} is an scwf equipped with a function-structure.
    \item A \emph{morphism of $\lambda$-domains} is an scwf-morphism preserving the function-structure.
\end{enum}
\end{defn}

%TODO replace 'stable under' by 'preserved by'?
It is easy to see that scwf-morphisms are componentwise composable, and that the identity morphism at each component gives an identity scwf-morphism on each scwf. Moreover, the property of preservation of function-structure is stable under composition, and the identity scwf-morphisms trivially preserve the function-structure. Hence:

\begin{defn} \label{def:cat-lambda-domain}
$\lambda$-domains and their morphisms form a category $\Ldom$.
\end{defn}

\subsection{Abstract syntax and models}
%TODO better title?
%TODO too many forward pointers? Too many references to figure with equations?

The goal of this section is to show that there exists a \textit{syntactic $\lambda$-domain} (Definition~\ref{def:syn-ldom}) which satisfies the universal property of being a free $\lambda$-domain over the set of base types $\Basetypes$ (Theorem~\ref{thm:syncat-free-ldom}). We then define a notion of model for the simply typed $\lambda$-calculus (Definition~\ref{def:stlc2-mod}) and show that there is a \textit{syntactic model} (Definition~\ref{def:syn-mod}) which is initial in the category of models (Theorem~\ref{thm:syn-mod-init}).

%TODO Analogies? (e.g. free abelian group)

As a first step, we discuss how to interpret the syntax of simply typed $\lambda$-calculus presented in Section~\ref{sec:stlc2-syntax} in a $\lambda$-domain $\cat{C}$. To do this, we need to choose an interpretation $J : \Basetypes \to \STy[\cat{C}]$ for the base types. Then, we can extend the interpretation $J$ to an interpretation $\cint[J]{-}$ of the syntax of $\lambda$-calculus by structural recursion on the syntax. The definitions are entirely straightforward: we simply replace each syntactic operation by the corresponding semantic operation. We spell out the interpretation in steps for the different kinds of syntactic entities.

\begin{defn}[Interpretation of $\lambda$-calculus in a $\lambda$-domain] \label{def:stlc2-int}
Let $\cat{C}$ be a $\lambda$-domain and let $J : \Basetypes \to \STy[\cat{C}]$ be an interpretation for the base types. The \emph{interpretation of $\lambda$-calculus in $\cat{C}$ with respect to $J$} is defined as follows.
%TODO drop the J (and superscripts) to make it less verbose?
\begin{enum}
\item The interpretation of types is defined by structural recursion on types:
\begin{align*}
\cintty[J]{-} &: \sTy \to \STy[\cat{C}] \\
\cintty[J]{\beta} &= J(\beta) \quad(\beta \in \Basetypes) \\
\cintty[J]{\functy{\sigma}{\tau}} &= \Sfuncty{\cintty[J]{\sigma}}{\cintty[J]{\tau}}
\end{align*}

\item The interpretation of contexts is defined by structural recursion on lists:
\begin{align*}
\cintcon[J]{-} &: \sCon \to \SCon[\cat{C}] \\
\cintcon[J]{\sempcon} &= \Sempcon[\cat{C}] \\
\cintcon[J]{\sextcon{\Gamma}{\sigma}}
    &= \Sextcon{\cintcon[J]{\Gamma}}{\cintty[J]{\sigma}}
\end{align*}

\item The interpretation of terms and substitutions is defined by mutual recursion on the generating rules for terms and substitutions (see Figure~\ref{fig:stlc2-terms-subs}).
\begin{align*}
\cinttm[J]{-} &: \sTm{\Gamma}{\sigma} \to \STm[\cintcon[J]{\Gamma}]{\cintty[J]{\sigma}}[\cat{C}] \\
\cintsub[J]{-} &: \sSub{\Delta}{\Gamma} \to \SSub{\cintcon[J]{\Delta}}{\cintcon[J]{\Gamma}}[\cat{C}] \\
\\
\cinttm[J]{\vz[\Gamma][\sigma]} &= \Sq[\cintcon[J]{\Gamma}][\cintty[J]{\sigma}] \\
\cinttm[J]{\app{t}{u}} &= \Sapp{\cinttm[J]{t}}{\cinttm[J]{u}} \\
\cinttm[J]{\slam{A}{t}} &= \Slam{\cintty[J]{A}}{\cinttm[J]{t}} \\
\cinttm[J]{\ssubst{t}{\gamma}} &= \Ssubst{\cinttm[J]{t}}{\cintsub[J]{\gamma}} \\
\\
\cintsub[J]{\sempsub[\Gamma]} &= \Sempsub[\cintcon[J]{\Gamma}] \\
\cintsub[J]{\sextsub{\gamma}{t}} &= \Sextsub{\cintsub[J]{\gamma}}{\cinttm[J]{t}} \\
\cintsub[J]{\sidsub[\Gamma]} &= \Sidsub[\cintcon[J]{\Gamma}] \\
\cintsub[J]{\scompsub{\gamma}{\delta}} &=
    \Scompsub{\cintsub[J]{\gamma}}{\cintsub[J]{\delta}} \\
\cintsub[J]{\spsub[\Gamma][\sigma]} &=
    \Spsub[\cintcon[J]{\Gamma}][\cintty[J]{\sigma}]
\end{align*}
\end{enum}
\end{defn}

To improve readability, we often drop the superscripts from $\cintty[J]{-}$, $\cintcon[J]{-}$, etc. and write $\cint[J]{-}$ uniformly for all the interpretation functions.

An important property of the interpretation is that convertible terms and substitutions are mapped to the same semantic object. This property is referred to as \textit{soundness}.

\begin{thm}[Soundness of the interpretation] \label{thm:cat-soundness}
Let $\cat{C}$ be a $\lambda$-domain and $J : \Basetypes \to \STy[\cat{C}]$ an interpretation for the base types.
\begin{enum}
\item For all $t$ and $t'$, if $\sconvtm{t}{t'}{\Gamma}{\sigma}$, then $\cint[J]{t} = \cint[J]{t'} \in \STm[\cint[J]{\Gamma}]{\cint[J]{\sigma}}[\cat{C}]$.
\item For all $\gamma$ and $\gamma'$, if $\sconvsub{\gamma}{\gamma'}{\Delta}{\Gamma}$, then $\cint[J]{\gamma} = \cint[J]{\gamma'} \in \SSub{\cint[J]{\Delta}}{\cint[J]{\Gamma}}[\cat{C}]$.
\end{enum}
\begin{proof}
Both statements are proved simultaneously by mutual induction on the derivation of $\sconvtm{t}{t'}{\Gamma}{\sigma}$, respectively $\sconvsub{\gamma}{\gamma'}{\Delta}{\Gamma}$.
%TODO give an example case?
We omit the unsurprising details.
\end{proof}
\end{thm}

The next step is to define the \textit{syntactic $\lambda$-domain}. We do this in stages corresponding to the structural complexity of $\lambda$-domains. First, we define the so called \textit{syntactic category} (Definition~\ref{def:syn-cat}), which serves as the base category for the syntactic $\lambda$-domain. Next, we endow the syntactic category with the structure of an scwf to obtain the \textit{syntactic scwf} (Definition~\ref{def:syn-scwf}. Finally, we equip the syntactic scwf with the \textit{syntactic function-structure} resulting in the syntactic $\lambda$-domain (Definition~\ref{def:syn-ldom}).

\begin{defn}[Syntactic category] \label{def:syn-cat}
The \emph{syntactic category} $\syncat$ of $\lambda$-calculus is defined as follows.
\begin{enum}
    \item Its objects are syntactic contexts, i.e. lists of types.
    
    \item For $\Gamma, \Delta \in \syncat$, $\Hom[\syncat]{\Delta}{\Gamma}$ is the set of syntactic substitutions $\sSub{\Delta}{\Gamma}$ quotiented by the convertibility relation $\sconvrelsub{\Delta}{\Gamma}$. That is,
    \[ \Hom[\syncat]{\Delta}{\Gamma} = \setof{[\gamma]}{\stypedsub{\gamma}{\Delta}{\Gamma}}, \]
    where
    \[ [\gamma] = \setof{\stypedsub{\gamma'}{\Delta}{\Gamma}}{\sconvsub{\gamma}{\gamma'}{\Delta}{\Gamma}} \]
    denotes the equivalence class of $\stypedsub{\gamma}{\Delta}{\Gamma}$.
    
    \item Composition of substitutions is done on representatives of equivalences classes:
    \[ \Scompsub{[\gamma]}{[\delta]}[\syncat] = [\scompsub{\gamma}{\delta}] \]
    for $\stypedsub{\gamma}{\Delta}{\Gamma}$ and $\stypedsub{\delta}{\Theta}{\Delta}$. This operation is well-defined since, by the congruence rule \textsc{cong-comp} (Figure~\ref{fig:stlc2-equations}), if $\gamma'$ and $\delta'$ are also representatives of the equivalences classes $[\gamma]$ and $[\delta]$, respectively, then $[\scompsub{\gamma}{\delta}] = [\scompsub{\gamma'}{\delta'}]$.

    \item The identity substitution of $\syncat$ on the context $\Gamma$ is the equivalence class of the syntactic identity substitution on $\Gamma$:
    \[ \Sidsub[\Gamma][\syncat] = [\sidsub[\Gamma]] \]
\end{enum}
%TODO category axioms vs axioms of categories
The category axioms are satisfied due to the associativity (\textsc{comp-assoc}) and identity (\textsc{id-l}, \textsc{id-r}) laws of the syntax (Figure~\ref{fig:stlc2-equations}).
\end{defn}

%TODO rephrase this paragraph to make it clearer?
%TODO category axioms vs axioms of categories
To summarize the construction of Definition~\ref{def:syn-cat}, the syntactic category $\syncat$ is obtained by using corresponding syntactic objects and operations for each component of the category. That is, contexts are syntactic contexts, substitutions are syntactic substitutions, etc. However, there is a subtlety involved for morphisms: instead of using the substitutions themselves, we quotient them by convertibility, thereby identifying convertible substitutions. Taking the quotient is necessary to satisfy the category axioms: without quotienting, they would only hold up to convertibility instead of equality.

The downside of quotienting is that all operations defined on equivalence classes must be proved to be well-defined, that is, independent of the chosen representatives. In the syntactic category, this was ensured by the congruence law \textsc{cong-comp}. In the rest of this section, we omit such verifications, as they all follow immediately from one of the congruence rules (\textsc{cong-app}, \textsc{cong-lam}, \textsc{cong-subst}, \textsc{cong-ext}, \textsc{cong-comp} in Figure~\ref{fig:stlc2-equations}) for the syntax.

\begin{defn}[Syntactic scwf] \label{def:syn-scwf}
The \emph{syntactic scwf} has as base category the syntactic category $\syncat$. The remaining structure is defined as follows.
\begin{enum}
    \item The terminal object $\Sempcon[\syncat]$ is the empty list $\sempcon$ and the unique map $\Sempsub[\Gamma][\syncat] : \Gamma \to \sempcon$ is the equivalence class of the empty tuple $\sempsub[\Gamma]$. (Note that this equivalence class is a singleton set since $\sempsub[\Gamma]$ is the unique element of $\sSub{\Gamma}{\sempcon}$.)

    \item The set $\STy[\syncat]$ is the set $\sTy$ of types of the simply typed $\lambda$-calculus.

    \item Given $\sigma \in \sTy$, the presheaf $\STm{\sigma}[\syncat]$ is defined as follows. It sends a context $\Gamma$ to the set of syntactic terms $\sTm{\Gamma}{\sigma}$ quotiented by the convertibility relation $\sconvreltm{\Gamma}{\sigma}$, and it sends an equivalence class $[\gamma] : \Delta \to \Gamma$ of substitutions to the function
    \[ \sTm{\Gamma}{\sigma}/\sconvreltm{\Gamma}{\sigma} \ \to\ 
        \sTm{\Delta}{\sigma}/\sconvreltm{\Delta}{\sigma} \]
    given by $[t] \mapsto [\ssubst{t}{\gamma}]$, where $\stypedsub{\gamma}{\Delta}{\Gamma}$ and $\stypedtm{t}{\Gamma}{\sigma}$. The presheaf axioms are satisfied due to the rules \textsc{subst-id} and \textsc{subst-comp} (Figure~\ref{fig:stlc2-equations}).

    \item Context comprehension is given by context extension, and the operations on substitution are defined by the analogous operations on the syntax.
\end{enum}
\end{defn}

\begin{defn}[Syntactic $\lambda$-domain] \label{def:syn-ldom}
The \emph{syntactic function-structure} on the syntactic scwf $\syncat$ is defined as follows.
\begin{enum}
\item We define $\Sfuncty{\sigma}{\tau}[\syncat]$ to be $\functy{\sigma}{\tau}$.
\item We define $\Sapp[\Gamma,\sigma,\tau]{[t]}{[u]}$ to be $[\app{t}{u}]$.
\item Finally, we define $\Slam{\sigma}[\Gamma,\tau]{[t]}$ to be $[\slam{\sigma}{t}]$.
\end{enum}
The \emph{syntactic $\lambda$-domain} is the syntactic scwf $\syncat$ equipped with the syntactic function-structure.
\end{defn}

We are now ready to state the universal property of the syntactic $\lambda$-domain.

\begin{thm} \label{thm:syncat-free-ldom}
For every $\lambda$-domain $\cat{C}$ and function $J : \Basetypes \to \STy[\cat{C}]$, there is a unique morphism of $\lambda$-domains $\cint[J]{-}[\cat{C}] : \syncat \to \cat{C}$ such that $\cint[J]{\beta}[\cat{C}] = J(\beta)$ for all $\beta \in \Basetypes$.
\begin{proof}
The components of the desired morphism of $\lambda$-domains $\cint[J]{-}[\cat{C}]$ are given by the interpretation functions of Definition~\ref{def:stlc2-int}. These definitions on the equivalence classes of terms and substitutions are well-defined due to the soundness of the interpretation (Theorem~\ref{thm:cat-soundness}). The structure of $\lambda$-domains is preserved by the definition of $\cint[J]{-}[\cat{C}]$. Uniqueness of the morphism follows by induction on the syntax of the $\lambda$-calculus given in Section~\ref{sec:stlc2-syntax}.
\end{proof}
\end{thm}

In categorical terms, the above theorem states that $\syncat$ is a free $\lambda$-domain over the set $\Basetypes$ of base types. We can rephrase the result in terms of initiality. For this, we introduce a category of models (Definition~\ref{def:cat-mod}) and show that there is a \textit{syntactic model} which is an initial object in this category.

\begin{defn}[Model] \label{def:stlc2-mod}
A \emph{model} of simply typed $\lambda$-calculus is a $\lambda$-domain $\cat{C}$ together with an interpretation function $J : \Basetypes \to \STy[\cat{C}]$ for the base types.
\end{defn}

\begin{defn}[Model morphism]
Let $(\cat{C}, J)$ and $(\cat{D}, K)$ be models. A \emph{model morphism} from $(\cat{C}, J)$ to $(\cat{D}, K)$ is a morphism of $\lambda$-domains $F : \cat{C} \to \cat{D}$ such that $F(J(\beta)) = K(\beta)$ for all $\beta \in \Basetypes$.
\end{defn}

Obviously, the composite of model morphisms is a model morphism, and the identity morphism on each $\lambda$-domain is a model morphism. Hence:

\begin{defn} \label{def:cat-mod}
Models and model morphisms form a category $\Mod$.
\end{defn}

\begin{defn} \label{def:syn-mod}
\begin{enum}
\item The \emph{canonical interpretation} $\canint : \Basetypes \to \STy[\syncat]$ of base types is the inclusion of $\Basetypes$ into $\sTy$.
\item The \emph{syntactic model} is the syntactic $\lambda$-domain $\syncat$ together with the canonical interpretation $\canint$ of base types.
\end{enum}
\end{defn}

\begin{thm} \label{thm:syn-mod-init}
The syntactic model $(\syncat, \canint)$ is an initial object in $\Mod$.
\begin{proof}
This theorem is a rephrasing of Theorem~\ref{thm:syncat-free-ldom}.
\end{proof}
\end{thm}

%TODO Remark: Initiality vs biinitiality?

\subsection{Semantics in cartesian closed categories}

In this section, we show that cartesian closed categories provide a wide range of models for the simply typed $\lambda$-calculus. Instead of defining the interpretation directly, we use the framework of the previous section.

\begin{prop} \label{prop:ccc-to-ldom}
We have a functor $S : \CCC \to \Ldom$.
\begin{proof}
We only discuss the object part of the functor. Given a CCC $\cat{C}$, we let $S(\cat{C})$ be the $\lambda$-domain defined as follows:
\begin{enum}
    \item Its base category is $\cat{C}$.
    \item Types are objects of $\cat{C}$.
    \item Terms $\STm[\Gamma]{A}$ are morphisms $\Gamma \to A$ in $\cat{C}$. Substitution in terms is given by precomposition of morphisms.
    \item Context comprehension is given by the categorical product in $\cat{C}$.
    \item The function type $\Sfuncty{A}{B}$ is the exponential $\cexp{A}{B}$ in $\cat{C}$. The operation $\Slam{A}[\Gamma,B]{-}$ sends a term $\cprod{\Gamma}{A} \to B$ to its exponential transpose. Finally, the operation $\Sapp[\Gamma,A,B]{}{}$ maps the pair $(t : \Gamma \to (\cexp{A}{B}), u : \Gamma \to A)$ to $\mev \circ \mpair{t}{u} : \Gamma \to B$. \qedhere
\end{enum}
\end{proof}
\end{prop}

\begin{cor}[Interpretation of $\lambda$-calculus in a CCC]
Let $\cat{C}$ be a CCC and let $J : \Basetypes \to \Ob{\cat{C}}$ be an interpretation for the base types. The interpretation of $\lambda$-calculus in $\cat{C}$ with respect to $J$ is given by the following clauses.
%TODO drop the superscripts and use overloaded notation everywhere?
\begin{enum}
\item Types:
\begin{align*}
\cintty[J]{-} &: \sTy \to \Ob{\cat{C}} \\
\cint[J]{\beta} &= J(\beta) \quad(\beta \in \Basetypes) \\
\cint[J]{\functy{\sigma}{\tau}} &= \cexp{\cint[J]{\sigma}}{\cint[J]{\tau}}
\end{align*}

\item Contexts:
\begin{align*}
\cintcon[J]{-} &: \sCon \to \Ob{\cat{C}} \\
\cint[J]{\sempcon} &= \1[\cat{C}] \\
\cint[J]{\sextcon{\Gamma}{\sigma}} &= \cprod{\cint[J]{\Gamma}}{\cint[J]{\sigma}}
\end{align*}

\item Terms and substitutions:
\begin{align*}
\cinttm[J]{-} &: \sTm{\Gamma}{\sigma} \to \Hom[\cat{C}]{\cint[J]{\Gamma}}{\cint[J]{\sigma}} \\
\cintsub[J]{-} &: \sSub{\Delta}{\Gamma} \to \Hom[\cat{C}]{\cint[J]{\Delta}}{\cint[J]{\Gamma}} \\
\\
\cint[J]{\vz[\Gamma][\sigma]} &= \msnd[\cint[J]{\Gamma}, \cint[J]{\sigma}] \\
\cint[J]{\app{t}{u}} &= \mev \circ \mpair{\cint[J]{t}}{\cint[J]{u}} \\
\cint[J]{\slam{A}{t}} &= \mcurry{\cint[J]{t}} \\
\cint[J]{\ssubst{t}{\gamma}} &= \cint[J]{t} \circ \cint[J]{\gamma} \\
\\
\cint[J]{\sempsub[\Gamma]} &= \mterm[\cint[J]{\Gamma}] \\
\cint[J]{\sextsub{\gamma}{t}} &= \mpair{\cint[J]{\gamma}}{\cint[J]{t}} \\
\cint[J]{\sidsub[\Gamma]} &= \id[\cint[J]{\Gamma}] \\
\cint[J]{\scompsub{\gamma}{\delta}} &= \cint[J]{\gamma} \circ \cint[J]{\delta} \\
\cint[J]{\spsub[\Gamma][\sigma]} &= \mfst[\cint[J]{\Gamma}, \cint[J]{\sigma}]
\end{align*}
\end{enum}
\begin{proof}
These clauses arise from the interpretation morphism $\cint[J]{-}[S(\cat{C})] : \syncat \to S(\cat{C})$.
\end{proof}
\end{cor}

\begin{comment}
\begin{defn} \label{def:scwf-from-products}
Any small cartesian category $\cat{C}$ gives rise to an scwf as follows:
\begin{items}
    \item The base category is $\cat{C}$ itself.
    \item The types are the objects of $\cat{C}$.
    \item For any object $A$, the terms of type $A$ are morphisms with codomain $A$, that is $\name{Tm}(-, A) = \cat{C}(-, A)$.
    \item Context comprehension is given by products. More precisely, $\Sextcon{\Gamma}{A}$ is the product $\cprod{\Gamma}{A}$, $p : \cprod{\Gamma}{A} \to \Gamma$ and $q : \cprod{\Gamma}{A} \to A$ are the projections, and $\Sextsub{-}{-}$ is given by the universal property of the product.
\end{items}

%TODO define cartesian categories and cartesian functors?
This construction extends to a functor $L : \CC \to \Scwf$: a strict cartesian functor $F : \cat{C} \to \cat{D}$ gives rise to a strict scwf-morphism $LF : L\cat{C} \to L\cat{D}$ all whose components identical to either the action on objects or morphisms of $F$. The preservation of products ensures that context comprehension is preserved.
\end{defn}

\begin{defn} \label{def:scwffun-from-ccc}
We extend Definition \ref{def:scwf-from-products} and define an scwf with function-structure from a cartesian closed category. We use the exponentials in $\cat{C}$ to define a function-structure the scwf. Concretely, $A \funcstruct B$ is given by the exponential $\cexp{A}{B}$. If $t : \cprod{\Gamma}{A} \to B$, then $\lambda_{\Gamma, A, B}(t) : \Gamma \to \cexp{A}{B}$ is the exponential transpose of $t$. For $t : \Gamma \to \cexp{A}{B}$ and $u : \Gamma \to A$, $\name{app}$ is defined as
\[ \name{app}_{\Gamma, A, B}(t, u) = \mev[\Gamma,A] \circ \Sextsub{t}{u} \]

Similarly to Definition \ref{def:scwf-from-products}, this construction extends to a functor $L : \CCC \to \Scwffun$, since if $F : \cat{C} \to \cat{D}$ preserves exponentials, then $LF : L\cat{C} \to \cat{D}$ preserves the function-structure.
\end{defn}

\end{comment}


\chapter{Gluing} \label{chap:gluing}
\begin{comment}
thoughts for intro:
- motivation: how to categorify Tait proof/NbE?
- presheaves are used to keep track of free variables (contexts)
- the inductive structure of logical predicates suggests that they form a model
- gluing category is the category of (proof relevant, Kripke) logical predicates
- the Tait proof states that every term has a normal form, but does not provide a procedure for computing them: extensional vs intensional normalization, proof-irrelevance vs proof-relevance
- Categorically: proof-irrelevant predicate is a monomorphism, proof-relevant one is any morphism
\end{comment}

\begin{comment}
Chap: Gluing
Main goals: present categorical proof, relate to NbE

overview of proof
- explain the main steps
- relate the steps to NbE (on a high level)
preparation
- category of renamings/weakenings, or poset of contexts under extension
- presheaf of terms via Yoneda embedding
- presheaves of normals and neutrals (cannot be defined in presheaves over classifying category)
HOAS?
gluing category and its properties (CCC, projection is strict CC), subcategory on monos?
- give intuition for the gluing category: category of logical predicates
- explain/spell out its ccc structure?
- add special cases for exponentiating with a representable?
quote & unquote
- define functions
- show that they are morphisms in the slice category (quote-unquote yoga)
def of normalization function + correctness
- soundness
- completeness
- naturality
- stability?
\end{comment}

In this chapter, we look at a categorical reconstruction of normalization by evaluation using categorical gluing. The presentation is similar to and is inspired by the work of Fiore \cite{fiore:2002:ppdp, fiore:2022:mscs} and Sterling and Spitters \cite{sterling:2018:arxiv}.

The structure of the proof follows closely the components of normalization by evaluation (Section~\ref{sec:nbe-alg}).
\begin{enum}
    \item First, we define a suitable model, namely the gluing category $\gluecat$ (Definition~\ref{def:gluing-cat}) together with the interpretation $\glueint$ of base types (Definition~\ref{def:gluing-interpretation}). The gluing category $\gluecat$ can be seen as the category of logical predicates.
    \item The model $(\gluecat, \glueint)$ gives rise to an interpretation functor $\cint[\glueint]{-}[\gluecat] : \syncat \to \gluecat$ (Definition~\ref{def:gluing-interpretation}).
    \item Next, we define families of natural transformations $\gquote{\sigma}$ and $\gunquote{\sigma}$ (Definition~\ref{def:gluing-quote-unquote}). These natural transformations correspond to the `quote' and `unquote' functions of normalization by evaluation.
    \item Finally, we define the normalization function as interpretation followed by quote (Definition~\ref{def:gluing-norm-fun}). For this, we need to extend the unquote map to contexts (Definition~\ref{def:gluing-unquote-context}), which is used to embed substitutions into the semantics. The environment $\idenv{\Gamma}$ corresponds to unquoting the identity substitution.
\end{enum}

\section{Neutral and normal forms via presheaves}

In this section, we define presheaves $\PNe{\sigma}$ and $\PNf{\sigma}$ of neutral and normal terms, respectively, for all types $\sigma \in \sTy$. Intuitively, $\PNe{\sigma}$ and $\PNf{\sigma}$ are variable sets indexed by contexts, i.e. $\PNe{\sigma}(\Gamma) = \sNe{\Gamma}{\sigma}$ and $\PNf{\sigma}(\Gamma) = \sNf{\Gamma}{\sigma}$. The presheaf action on morphisms corresponds to substitution. Since neutral and normal terms are not closed under arbitrary substitutions, we need to restrict the base category of the presheaves to an appropriate subcategory of $\syncat$.

In this thesis, we choose the category of \textit{weakenings} for the restricted base category, which is isomorphic to the dual of the poset of contexts under the prefix ordering. Alternative base categories are the category of \textit{order-preserving embeddings} or the category of \textit{renamings}. The former is used in \cite{altenkirch:1995:ctcs} (in which it is confusingly called the category of weakenings) and \cite{kovacs:2017:msc}, and the latter in \cite{fiore:2002:ppdp, sterling:2018:arxiv, fiore:2022:mscs}.

\begin{defn}[Poset of contexts] \label{def:contexts-poset}
We order the set $\sCon$ of contexts by the \emph{prefix ordering}: $\Gamma \le \Delta$ iff $\Gamma$ is a prefix of $\Delta$. We write $\ge$ for the dual order, which we call the \emph{extension order}: $\Delta \ge \Gamma$ iff $\Gamma \le \Delta$. If $\Delta \ge \Gamma$, we also say that $\Delta$ is an extension of $\Gamma$.
\end{defn}

From now on, the set $\sCon$ is assumed to carry the ordering $\ge$ of Definition~\ref{def:contexts-poset} unless otherwise stated. We view the poset $\pCon$ as a category whose objects are contexts and such that $\Hom[\pCon]{\Delta}{\Gamma}$ is a singleton $\singset$ if $\Delta \ge \Gamma$ and $\Hom[\pCon]{\Delta}{\Gamma}$ is empty otherwise.

\begin{lem} \label{lem:context-extension-inversion}
For every $\Delta, \Gamma \in \sCon$, we have
\[ \Delta \ge \Gamma \quad\iff\quad
    \Delta = \Gamma \vee \exists \Delta' \in \sCon, \sigma \in \sTy.\;
                            \Delta = (\Delta', \sigma) \wedge \Delta' \ge \Gamma \]
\begin{proof}
Follows by induction on the difference of the lengths of $\Delta$ and $\Gamma$.
\end{proof}
\end{lem}

In what follows, we often employ Lemma~\ref{lem:context-extension-inversion} to define mathematical objects depending on a pair of contexts $\Delta$ and $\Gamma$ such that $\Delta \ge \Gamma$ by case distinction (for instance, Definition~\ref{def:general-weakening}). In the second case, the definition may refer recursively to the object being defined but depending on the ``smaller" pair $\Delta' \ge \Gamma$. This type of definition can be seen as a sort of recursion principle for the partial order $\ge$. Alternatively, one may think of this definition scheme as recursion on lists but where the base case is $\Gamma$ instead of the empty list.

%TODO move to next section?
\begin{defn} \label{def:general-weakening}
We define substitutions $\stypedsub{\Spsub[\Gamma][\Delta]}{\Delta}{\Gamma}$ for any pair of contexts $\Delta \ge \Gamma$:
\[ \spsub[\Gamma][\Delta] =
    \begin{cases}
        \sidsub[\Gamma] & \text{ if } \Delta = \Gamma \\
        \scompsub{\spsub[\Gamma][\Delta']}{\spsub[\Gamma][\sigma]} & \text{ if } \Delta = (\sextcon{\Delta'}{\sigma}) \wedge \Delta' \ge \Gamma
    \end{cases}. \]
\end{defn}

The substitution $\spsub[\Gamma][\Delta]$ is a sort of generalized weakening. The primitive substitution $\spsub[\Gamma][\sigma]$ only adds the single type $\sigma$ to a context $\Gamma$. In contrast, $\spsub[\Gamma][\Delta]$ extends $\Gamma$ with any number of types.

\begin{lem} \label{lem:psub-comp}
$\sconvsub{\scompsub{\Spsub[\Gamma][\Delta]}{\Spsub[\Delta][\Theta]}}{\Spsub[\Gamma][\Theta]}{\Theta}{\Gamma}$
\begin{proof}
Using Lemma~\ref{lem:context-extension-inversion} on the proof that $\Theta \ge \Delta$.
\end{proof}
\end{lem}

\begin{prop}
There is an \emph{inclusion} functor $\inclcon : \pCon \to \syncat$.
\begin{proof}
The functor $\inclcon : \pCon \to \syncat$ is the identity on contexts, and it sends the unique morphism $\Delta \ge \Gamma$ in $\pCon$ to the equivalence class of $\spsub[\Gamma][\Delta] \in \sSub{\Delta}{\Gamma}$. Composition is preserved by Lemma~\ref{lem:psub-comp}, and identities are preserved by the definition of $\spsub[\Gamma][\Delta]$.
\end{proof}
\end{prop}

The functor $\inclcon$ is bijective on objects and faithful. Hence, it allows us to view $\pCon$ as a wide subcategory of $\syncat$, containing only morphisms of the form $\spsub[\Gamma][\Delta]$ for $\Delta \ge \Gamma$. This subcategory is also called the \textit{category of weakenings}.

Recall from Definition~\ref{def:reindexing} that the inclusion $\inclcon : \pCon \to \syncat$ gives rise to a \textit{reindexing functor} $\reind{\inclcon} : \PSh{\syncat} \to \PSh{\pCon}$.

\begin{defn}
We define the functor $\FSub : \syncat \to \PSh{\pCon}$ as the composite
\[ \syncat \xto{\y} \PSh{\syncat} \xto{\reind{\inclcon}} \PSh{\pCon}. \]
\end{defn}

We write $\FSub[\Gamma]$ for $\FSub(\Gamma)$. Intuitively, $\FSub[\Gamma]$ is the presheaf of substitutions with codomain $\Gamma$. The presheaf action weakens the domain of a substitution, allowing extra free variables in the context.

We are now ready to define the presheaves of neutral and normal terms \cite{altenkirch:1995:ctcs, sterling:2018:arxiv}.

\begin{prop}
We have presheaves $\PNe{\sigma} \in \PSh{\pCon}$ and $\PNf{\sigma} \in \PSh{\pCon}$ whose object parts are given by
\[ \PNe{\sigma}(\Gamma) = \sNe{\Gamma}{\sigma} \quad\text{and}\quad
   \PNf{\sigma}(\Gamma) = \sNf{\Gamma}{\sigma}. \]
\begin{proof}
The morphism parts of the presheaves $\PNe{\sigma}$ and $\PNf{\sigma}$ are essentially given by
\[ \PNe{\sigma}(\Delta \ge \Gamma)(\stypedne{m}{\Gamma}{\sigma})
        = \ssubst{m}{\spsub[\Gamma][\Delta]} \]
and
\[ \PNf{\sigma}(\Delta \ge \Gamma)(\stypednf{n}{\Gamma}{\sigma})
        = \ssubst{n}{\spsub[\Gamma][\Delta]}, \]
where, abusing notation slightly, we write $\Delta \ge \Gamma$ for the unique element of $\Hom[\pCon]{\Delta}{\Gamma}$. However, we cannot use the substitution term former (\textsc{subst} in Figure~\ref{fig:stlc2-terms-subs}) since terms of the form $\subst{t}{\gamma}$ for a term $t$ and substitution $\gamma$ are neither neutral nor normal. Instead, we need to define these actions using mutual recursion on the syntax of neutral terms and normal forms. We omit the details.
\end{proof}
\end{prop}

\begin{prop} \label{prop:presheaf-syntax}
We have the following morphisms:
\begin{items}
    \item $\PNe{\beta} \xto{\cong} \PNf{\beta}$
    \item $\nvar{\sigma} : \FSub[\singlist{\sigma}] \to \PNe{\sigma}$
    \item $\napp{\sigma}{\tau} : \cprod{\PNe{\functy{\sigma}{\tau}}}{\PNf{\sigma}} \to \PNe{\tau}$
    \item $\nlam{\sigma}{\tau} : (\cexp{\FSub[\singlist{\sigma}]}{\PNf{\tau}}) \xto{\cong} \PNf{\functy{\sigma}{\tau}}$
\end{items}
\begin{proof}
The first morphism coincides with the morphism in \cite[Figure 13]{fiore:2022:mscs}. The final three morphisms coincide with those in \cite[Figure 14]{fiore:2022:mscs}.
\end{proof}
\end{prop}

Similarly to the presheaves of neutral terms, we have presheaves $\PNe{\Gamma} \in \PSh{\pCon}$ of neutral substitutions. These are necessary for Definition~\ref{def:gluing-unquote-context}.

\section{Gluing category}

In this section, we construct the gluing category (Definition~\ref{def:gluing-cat}) and prove that it is cartesian closed and hence a model of the simply typed $\lambda$-calculus. We also define the gluing interpretation (Definition~\ref{def:gluing-interpretation}) and discuss how to interpret the syntax of $\lambda$-calculus in the gluing category (Corollary~\ref{cor:gluing-interpretation}).

As a first step, we need to show that the syntactic category $\syncat$ is cartesian closed.

\begin{notn}
Let $\Gamma = [A_1, \ldots, A_n] \in \sCon$ and $A \in \sTy$. Then $\functy{\Gamma}{A} \in \sTy$ is shorthand for $\functy{A_1}{\functy{\ldots}{\functy{A_n}{A}}}$.
\end{notn}

\begin{defn} \label{def:syncat-ccc-structure}
\begin{enum}
\item The terminal object in $\syncat$ is the empty context $\sempcon$.
\item The product $\cprod{\Gamma}{\Delta}$ of contexts $\Gamma$ and $\Delta$ in $\syncat$ is their concatenation $\concat{\Gamma}{\Delta}$.
%The first projection $\mfst[\Gamma,\Delta]$ is $\spsub[\Gamma][\concat{\Gamma}{\Delta}] : \concat{\Gamma}{\Delta} \to \Gamma$.
%TODO for the second projection, we need to define $\Delta \\ \Gamma$ and $\Sq[\Gamma][\Delta]$ by recursion on \Delta \ge \Gamma
\item If $\Delta = [B_1, \ldots, B_n]$, then the exponential $\cexp{\Gamma}{\Delta}$ is the context $[\functy{\Gamma}{B_1}, \ldots, \functy{\Gamma}{B_n}]$.
%TODO: evaluation
\end{enum}
\end{defn}

\begin{lem} \label{lem:syncat-ccc}
The syntactic category $\syncat$ is cartesian closed with structure given in Definition~\ref{def:syncat-ccc-structure}.
\begin{proof}
The lemma can be proved using constructions similar to {\v{C}}ubri{\'c} et al. \cite[Proposition 3.6]{cubric:1998:mscs}. There are two differences: they use variable names in the context of P-category theory, whereas we use de Bruijn indices and explicit substitutions in the context of ordinary category theory.
\end{proof}
\end{lem}

We can now define the gluing category $\gluecat$.

\begin{defn}[Gluing category] \label{def:gluing-cat}
We define the \emph{gluing category} $\gluecat$ to be the comma category $\comma{\PSh{\pCon}}{\FSub}$. The functors $\gluefst$ and $\gluesnd$ are given by following diagram:
% https://q.uiver.app/#q=WzAsNCxbMCwwLCJcXGdsdWVjYXQiXSxbMSwwLCJcXFBTaHtcXHBDb259Il0sWzEsMSwiXFxQU2h7XFxwQ29ufSJdLFswLDEsIlxcc3luY2F0Il0sWzMsMiwiXFxGU3ViIiwyXSxbMSwyLCJcXGlkZnVuYyJdLFswLDEsIlxcZ2x1ZWZzdCJdLFswLDMsIlxcZ2x1ZXNuZCIsMl1d
\[\begin{tikzcd}
	\gluecat & {\PSh{\pCon}} \\
	\syncat & {\PSh{\pCon}}
	\arrow["\FSub"', from=2-1, to=2-2]
	\arrow["\idfunc", from=1-2, to=2-2]
	\arrow["\gluefst", from=1-1, to=1-2]
	\arrow["\gluesnd"', from=1-1, to=2-1]
\end{tikzcd}\]
\end{defn}

Recall that objects of $\gluecat$ are triples $(P, \Gamma, p)$ where $P \in \PSh{\pCon}$ is a presheaf over $\pCon$, $\Gamma \in \syncat$ is a syntactic context, and $p : P \to \FSub[\Gamma]$ is a natural transformation. The morphism $p$ can be seen as an indexed predicate over substitutions (where the index is the domain of the substitutions). Naturality of $p$ means that the predicate is stable under weakening.

Now we show that $\gluecat$ is cartesian closed and that $\gluesnd$ is strict cartesian closed. For this, we need the following lemma.

\begin{lem} \label{lem:fsub-weak-preservation}
The functor $\FSub : \syncat \to \PSh{\pCon}$ weakly preserves the terminal object and products.
\begin{proof}
This is a consequence of Proposition~\ref{prop:yoneda-preservation} and Proposition~\ref{prop:reindexing-preservation}.
\end{proof}
\end{lem}

\begin{prop}
\begin{enum}
\item The category $\gluecat$ is cartesian closed.
\item The functor $\gluesnd : \gluecat \to \syncat$ is strict cartesian closed.
\end{enum}
\begin{proof}
Follows immediately from Proposition~\ref{prop:gluing-category-ccc}, Lemma~\ref{lem:syncat-ccc}, Proposition~\ref{prop:presheaf-cat-is-ccc}, Lemma~\ref{lem:fsub-weak-preservation}, and Proposition~\ref{prop:presheaf-cat-pullbacks}.
\end{proof}
\end{prop}

\begin{comment}
\begin{enum}
\item We define $\gluevar{\sigma} \in \gluecat$ as
\[ \gluevar{\sigma} = (\FSub[\singlist{\sigma}], \singlist{\sigma}, \id[\FSub[\singlist{\sigma}]]). \]
\item We define $\glueneut{\sigma} \in \gluecat$ as
\[ \glueneut{\sigma} = (\PNe{\sigma}, \singlist{\sigma}, \predneut{\sigma}), \]
where $\predneut{\sigma} : \PNe{\sigma} \to \FSub[\singlist{\sigma}]$ sends a neutral term $\stypedne{m}{\Gamma}{\sigma}$ to $[\stmsub{m}] \in \SSub{\Gamma}{\singlist{\sigma}}[\syncat]$.

\item We define $\gluenorm{\sigma} \in \gluecat$ as
\[ \gluenorm{\sigma} = (\PNf{\sigma}, \singlist{\sigma}, \prednorm{\sigma}), \]
where $\prednorm{\sigma} : \PNf{\sigma} \to \FSub[\singlist{\sigma}]$ sends a normal term $\stypednf{n}{\Gamma}{\sigma}$ to $[\stmsub{n}] \in \SSub{\Gamma}{\singlist{\sigma}}[\syncat]$.
\end{enum}
\end{comment}

We now define the gluing interpretation. Recall that for every term $\stypedtm{t}{\Gamma}{\sigma}$ we have the substitution $\stmsub{t} \in \sSub{\Gamma}{\singlist{\sigma}}$ (Notation~\ref{not:special-subs}).

\begin{defn}[Gluing interpretation] \label{def:gluing-interpretation}
We define the \emph{gluing interpretation} $\glueint : \Basetypes \to \gluecat$ as follows:
\[ \glueint(\beta) = (\PNf{\beta}, \singlist{\beta}, \prednorm{\beta}), \]
where $\prednorm{\beta} : \PNf{\beta} \to \FSub[\singlist{\beta}]$ sends a normal form $s \in \PNf{\beta}(\Gamma)$ to $[\stmsub{n}] \in \Hom[\syncat]{\Gamma}{\singlist{\beta}}$.
\end{defn}

By Theorem~\ref{thm:syncat-free-ldom}, we get an interpretation functor $\cint[\glueint]{-}[\gluecat] : \syncat \to \gluecat$ which is a morphism of $\lambda$-domains. This satisfies the following proposition.

\begin{prop} \label{prop:proj-int-is-id}
The following diagram of categories and functors commutes:
% https://q.uiver.app/#q=WzAsMyxbMCwwLCJcXHN5bmNhdCJdLFsxLDAsIlxcZ2x1ZWNhdCJdLFsxLDEsIlxcc3luY2F0Il0sWzAsMiwiXFxpZGZ1bmNbXFxzeW5jYXRdIiwyXSxbMCwxLCJcXGNpbnRbXFxnbHVlaW50XXstfVtcXGdsdWVjYXRdIl0sWzEsMiwiXFxnbHVlc25kIl1d
\[\begin{tikzcd}
	\syncat & \gluecat \\
	& \syncat
	\arrow["{\idfunc[\syncat]}"', from=1-1, to=2-2]
	\arrow["{\cint[\glueint]{-}[\gluecat]}", from=1-1, to=1-2]
	\arrow["\gluesnd", from=1-2, to=2-2]
\end{tikzcd}\]
\begin{proof}
By definition, the interpretation functor $\cint[\glueint]{-}[\gluecat]$ is the interpretation morphism $\cint[\glueint]{-}[S(\gluecat)]$, where $S$ is the functor from Proposition~\ref{prop:ccc-to-ldom}. Applying the functor $S$ to $\gluesnd : \gluecat \to \syncat$, we get the following commutative diagram of $\lambda$-domains due to the initiality of $\syncat$:
% https://q.uiver.app/#q=WzAsMyxbMCwwLCJcXHN5bmNhdCJdLFsxLDAsIlMoXFxnbHVlY2F0KSJdLFsxLDEsIlMoXFxzeW5jYXQpIl0sWzAsMiwiIiwyLHsic3R5bGUiOnsiYm9keSI6eyJuYW1lIjoiZGFzaGVkIn19fV0sWzAsMSwiXFxjaW50W1xcZ2x1ZWludF17LX1bXFxnbHVlY2F0XSJdLFsxLDIsIlMoXFxnbHVlc25kKSJdXQ==
\[\begin{tikzcd}
	\syncat & {S(\gluecat)} \\
	& {S(\syncat)}
	\arrow[dashed, from=1-1, to=2-2]
	\arrow["{\cint[\glueint]{-}[\gluecat]}", from=1-1, to=1-2]
	\arrow["{S(\gluesnd)}", from=1-2, to=2-2]
\end{tikzcd}\]
We now obtain the desired diagram by forgetting the extra structure of $\lambda$-domains and their morphisms.
\end{proof}
\end{prop}

\begin{rem}
The above proof crucially depends on the strictness of the functor $\gluesnd : \gluecat \to \syncat$. This is why we decided to adopt CCCs with structure and strict cartesian closed functors in Chapter~\ref{chap:cat-prelims}. See also Remark~\ref{rem:ccc-strict-vs-weak-preservation}.
\end{rem}

We now characterize how the syntax of $\lambda$-calculus is interpreted in the gluing category.

\begin{cor} \label{cor:gluing-interpretation}
The following statements hold for the interpretation functor $\cint[\glueint]{-}[\gluecat]$.
\begin{enum}
\item For every $\sigma \in \sTy$, we have
\[ \cintty[\glueint]{\sigma} = (\gluepsh{\sigma}, \singlist{\sigma}, \gluepred{\sigma}) \]
for some $\gluepsh{\sigma} \in \PSh{\pCon}$ and $\gluepred{\sigma} : \gluepsh{\sigma} \to \FSub[[\sigma]]$.

\item For every $\Gamma \in \sCon$, we have
\[ \cintcon[\glueint]{\Gamma} = (\gluepsh{\Gamma}, \Gamma, \gluepred{\Gamma}) \]
for some $\gluepsh{\Gamma} \in \PSh{\pCon}$ and $\gluepred{\Gamma} : \gluepsh{\Gamma} \to \FSub[\Gamma]$.

\item For every $\stypedtm{t}{\Gamma}{\sigma}$, we have
\[ \cinttm[\glueint]{t} = (\gluemortm{t}, [\stmsub{t}]) : (\gluepsh{\Gamma}, \Gamma, \gluepred{\Gamma}) \to (\gluepsh{\sigma}, \singlist{\sigma}, \gluepred{\sigma}) \]
for some $\gluemortm{t} : \gluepsh{\Gamma} \to \gluepsh{\sigma}$. That is, the diagram
% https://q.uiver.app/#q=WzAsNCxbMCwwLCJcXGdsdWVwc2h7XFxHYW1tYX0iXSxbMSwwLCJcXGdsdWVwc2h7XFxzaWdtYX0iXSxbMCwxLCJcXEZTdWJbXFxHYW1tYV0iXSxbMSwxLCJcXEZTdWJbXFxzaW5nbGlzdHtcXHNpZ21hfV0iXSxbMCwyLCJcXGdsdWVwcmVke1xcR2FtbWF9IiwyXSxbMSwzLCJcXGdsdWVwcmVke1xcc2lnbWF9Il0sWzAsMSwiXFxnbHVlbW9ydG17dH0iXSxbMiwzLCJcXFNjb21wc3Vie1tcXHN0bXN1Ynt0fV19ey19IiwyXV0=
\[\begin{tikzcd}
	{\gluepsh{\Gamma}} & {\gluepsh{\sigma}} \\
	{\FSub[\Gamma]} & {\FSub[\singlist{\sigma}]}
	\arrow["{\gluepred{\Gamma}}"', from=1-1, to=2-1]
	\arrow["{\gluepred{\sigma}}", from=1-2, to=2-2]
	\arrow["{\gluemortm{t}}", from=1-1, to=1-2]
	\arrow["{\Scompsub{[\stmsub{t}]}{-}}"', from=2-1, to=2-2]
\end{tikzcd}\]
commutes.

\item For every $\stypedsub{\gamma}{\Delta}{\Gamma}$, we have
\[ \cintsub[\glueint]{\gamma} = (\gluemorsub{\gamma}, [\gamma]) : (\gluepsh{\Delta}, \Delta, \gluepred{\Delta}) \to (\gluepsh{\Gamma}, \Gamma, \gluepred{\Gamma}) \]
for some $\gluemorsub{\gamma} : \gluepsh{\Delta} \to \gluepsh{\Gamma}$. That is, the diagram
% https://q.uiver.app/#q=WzAsNCxbMCwwLCJcXGdsdWVwc2h7XFxEZWx0YX0iXSxbMSwwLCJcXGdsdWVwc2h7XFxHYW1tYX0iXSxbMCwxLCJcXEZTdWJbXFxEZWx0YV0iXSxbMSwxLCJcXEZTdWJbXFxHYW1tYV0iXSxbMCwyLCJcXGdsdWVwcmVke1xcRGVsdGF9IiwyXSxbMSwzLCJcXGdsdWVwcmVke1xcR2FtbWF9Il0sWzAsMSwiXFxnbHVlbW9yc3Vie1xcZ2FtbWF9Il0sWzIsMywiXFxTY29tcHN1YntbXFxnYW1tYV19ey19IiwyXV0=
\[\begin{tikzcd}
	{\gluepsh{\Delta}} & {\gluepsh{\Gamma}} \\
	{\FSub[\Delta]} & {\FSub[\Gamma]}
	\arrow["{\gluepred{\Delta}}"', from=1-1, to=2-1]
	\arrow["{\gluepred{\Gamma}}", from=1-2, to=2-2]
	\arrow["{\gluemorsub{\gamma}}", from=1-1, to=1-2]
	\arrow["{\Scompsub{[\gamma]}{-}}"', from=2-1, to=2-2]
\end{tikzcd}\]
commutes.
\end{enum}
\begin{proof}
This follows from Proposition~\ref{prop:proj-int-is-id}, because $\gluesnd$ picks out the second coordinate from the objects of the gluing category.
\end{proof}
\end{cor}

\section{Quote and unquote}

The goal of this section is to define the quote and unquote functions of normalization by evaluation.

\begin{prop} \label{def:gluing-quote-unquote}
We have two families of natural transformations
\[ \gunquote{\sigma} : \PNe{\sigma} \to \gluepsh{\sigma} \quad\text{and}\quad    \gquote{\sigma} : \gluepsh{\sigma} \to \PNf{\sigma}. \]
\begin{proof}
We define $\gunquote{\sigma}$ and $\gquote{\sigma}$ by mutual recursion on the type $\sigma$. For the definitions, we use the morphisms in Proposition~\ref{prop:presheaf-syntax}.
\begin{enum}
\item For base types $\beta \in \Basetypes$, we have $\gluepsh{\beta} = \PNf{\beta}$. Thus, we put $\gunquote{\beta} = (\PNe{\beta} \xto{\cong} \PNf{\beta})$ and $\gquote{\beta} = \id[\PNf{\beta}]$.

\item For function types $\functy{\sigma}{\tau}$, by Corollary~\ref{cor:gluing-interpretation}, we have that $\gluepsh{\functy{\sigma}{\tau}}$ is given by the pullback diagram
% https://q.uiver.app/#q=WzAsNSxbMCwwLCJcXGdsdWVwc2h7XFxmdW5jdHl7XFxzaWdtYX17XFx0YXV9fSJdLFswLDEsIlxcRlN1YltcXHNpbmdsaXN0e1xcZnVuY3R5e1xcc2lnbWF9e1xcdGF1fX1dIl0sWzEsMSwiKFxcY2V4cHtcXEZTdWJbXFxzaW5nbGlzdHtcXHNpZ21hfV19e1xcRlN1YltcXHNpbmdsaXN0e1xcdGF1fV19KSJdLFszLDAsIlxcY2V4cHtcXGdsdWVwc2h7XFxzaWdtYX19e1xcZ2x1ZXBzaHtcXHRhdX19Il0sWzMsMSwiXFxjZXhwe1xcZ2x1ZXBzaHtcXHNpZ21hfX17XFxGU3ViW1xcc2luZ2xpc3R7XFx0YXV9XX0iXSxbMCwxLCJcXGdsdWVwcmVke1xcZnVuY3R5e1xcc2lnbWF9e1xcdGF1fX0iLDJdLFsxLDIsIlxcZXhwY21wW1xcc2luZ2xpc3R7XFxzaWdtYX0sXFxzaW5nbGlzdHtcXHRhdX1dIiwyXSxbMCwzLCJrIl0sWzMsNCwiXFxjZXhwe1xcZ2x1ZXBzaHtcXHNpZ21hfX17XFxnbHVlcHJlZHtcXHRhdX19Il0sWzIsNCwiXFxjZXhwe1xcZ2x1ZXByZWR7XFxzaWdtYX19e1xcRlN1YltcXHNpbmdsaXN0e1xcdGF1fV19IiwyXV0=
\begin{equation} \label{diag:quote-unquote-pb}
\begin{tikzcd}
	{\gluepsh{\functy{\sigma}{\tau}}} &&& {\cexp{\gluepsh{\sigma}}{\gluepsh{\tau}}} \\
	{\FSub[\singlist{\functy{\sigma}{\tau}}]} & {(\cexp{\FSub[\singlist{\sigma}]}{\FSub[\singlist{\tau}]})} && {\cexp{\gluepsh{\sigma}}{\FSub[\singlist{\tau}]}}
	\arrow["{\gluepred{\functy{\sigma}{\tau}}}"', from=1-1, to=2-1]
	\arrow["{\expcmp[\singlist{\sigma},\singlist{\tau}]}"', from=2-1, to=2-2]
	\arrow["k", from=1-1, to=1-4]
	\arrow["{\cexp{\gluepsh{\sigma}}{\gluepred{\tau}}}", from=1-4, to=2-4]
	\arrow["{\cexp{\gluepred{\sigma}}{\FSub[\singlist{\tau}]}}"', from=2-2, to=2-4]
\end{tikzcd}
\end{equation}
\begin{items}
    \item We define $\gunquote{\functy{\sigma}{\tau}} : \PNe{\functy{\sigma}{\tau}} \to \gluepsh{\functy{\sigma}{\tau}}$ using the universal property of the pullback square (\ref{diag:quote-unquote-pb}). For that, we need natural transformations $f : \PNe{\functy{\sigma}{\tau}} \to \FSub[\singlist{\functy{\sigma}{\tau}}]$ and $g : \PNe{\functy{\sigma}{\tau}} \to \cexp{\gluepsh{\sigma}}{\gluepsh{\tau}}$ such that the square
    % https://q.uiver.app/#q=WzAsNSxbMCwxLCJcXEZTdWJbXFxzaW5nbGlzdHtcXGZ1bmN0eXtcXHNpZ21hfXtcXHRhdX19XSJdLFsxLDEsIihcXGNleHB7XFxGU3ViW1xcc2luZ2xpc3R7XFxzaWdtYX1dfXtcXEZTdWJbXFxzaW5nbGlzdHtcXHRhdX1dfSkiXSxbMywwLCJcXGNleHB7XFxnbHVlcHNoe1xcc2lnbWF9fXtcXGdsdWVwc2h7XFx0YXV9fSJdLFszLDEsIlxcY2V4cHtcXGdsdWVwc2h7XFxzaWdtYX19e1xcRlN1YltcXHNpbmdsaXN0e1xcdGF1fV19Il0sWzAsMCwiXFxQTmV7XFxmdW5jdHl7XFxzaWdtYX17XFx0YXV9fSJdLFswLDEsIlxcZXhwY21wW1xcc2luZ2xpc3R7XFxzaWdtYX0sXFxzaW5nbGlzdHtcXHRhdX1dIiwyXSxbMiwzLCJcXGNleHB7XFxnbHVlcHNoe1xcc2lnbWF9fXtcXGdsdWVwcmVke1xcdGF1fX0iXSxbMSwzLCJcXGNleHB7XFxnbHVlcHJlZHtcXHNpZ21hfX17XFxGU3ViW1xcc2luZ2xpc3R7XFx0YXV9XX0iLDJdLFs0LDAsImYiLDJdLFs0LDIsImciXV0=
    \[\begin{tikzcd}
    	{\PNe{\functy{\sigma}{\tau}}} &&& {\cexp{\gluepsh{\sigma}}{\gluepsh{\tau}}} \\
    	{\FSub[\singlist{\functy{\sigma}{\tau}}]} & {(\cexp{\FSub[\singlist{\sigma}]}{\FSub[\singlist{\tau}]})} && {\cexp{\gluepsh{\sigma}}{\FSub[\singlist{\tau}]}}
    	\arrow["{\expcmp[\singlist{\sigma},\singlist{\tau}]}"', from=2-1, to=2-2]
    	\arrow["{\cexp{\gluepsh{\sigma}}{\gluepred{\tau}}}", from=1-4, to=2-4]
    	\arrow["{\cexp{\gluepred{\sigma}}{\FSub[\singlist{\tau}]}}"', from=2-2, to=2-4]
    	\arrow["f"', from=1-1, to=2-1]
    	\arrow["g", from=1-1, to=1-4]
    \end{tikzcd}\]
    commutes. For $f$, we choose the morphism $\predneut{\functy{\sigma}{\tau}}$ that sends a neutral term $m \in \PNe{\functy{\sigma}{\tau}}(\Gamma)$ to $[\stmsub{m}] \in \Hom[\syncat]{\Gamma}{\singlist{\functy{\sigma}{\tau}}}$. For $g$, we take the exponential transpose of the composite
    % https://q.uiver.app/#q=WzAsNCxbMCwwLCJcXGNwcm9ke1xcUE5le1xcZnVuY3R5e1xcc2lnbWF9e1xcdGF1fX19e1xcZ2x1ZXBzaHtcXHNpZ21hfX0iXSxbMiwwLCJcXGNwcm9ke1xcUE5le1xcZnVuY3R5e1xcc2lnbWF9e1xcdGF1fX19e1xcUE5me1xcc2lnbWF9fSJdLFszLDAsIlxcUE5le1xcdGF1fSJdLFs0LDAsIlxcZ2x1ZXBzaHtcXHRhdX0iXSxbMCwxLCJcXGNwcm9ke1xcaWR9e1xcZ3F1b3Rle1xcc2lnbWF9fSJdLFsxLDIsIlxcbmFwcHtcXHNpZ21hfXtcXHRhdX0iXSxbMiwzLCJcXGd1bnF1b3Rle1xcdGF1fSJdXQ==
    \[\begin{tikzcd}
    	{\cprod{\PNe{\functy{\sigma}{\tau}}}{\gluepsh{\sigma}}} && {\cprod{\PNe{\functy{\sigma}{\tau}}}{\PNf{\sigma}}} & {\PNe{\tau}} & {\gluepsh{\tau}}
    	\arrow["{\cprod{\id}{\gquote{\sigma}}}", from=1-1, to=1-3]
    	\arrow["{\napp{\sigma}{\tau}}", from=1-3, to=1-4]
    	\arrow["{\gunquote{\tau}}", from=1-4, to=1-5]
    \end{tikzcd}\]

    \item We define $\gquote{\functy{\sigma}{\tau}} : \gluepsh{\functy{\sigma}{\tau}} \to \PNf{\functy{\sigma}{\tau}}$ as the composite
    % https://q.uiver.app/#q=WzAsNCxbMCwwLCJcXGdsdWVwc2h7XFxmdW5jdHl7XFxzaWdtYX17XFx0YXV9fSJdLFsxLDAsIihcXGNleHB7XFxnbHVlcHNoe1xcc2lnbWF9fXtcXGdsdWVwc2h7XFx0YXV9fSkiXSxbMywwLCIoXFxjZXhwe1xcRlN1YltcXHNpbmdsaXN0e1xcc2lnbWF9XX17XFxQTmZ7XFx0YXV9fSkiXSxbNCwwLCJcXFBOZntcXGZ1bmN0eXtcXHNpZ21hfXtcXHRhdX19Il0sWzAsMSwiayJdLFsxLDIsIlxcY2V4cHtcXGd1bnF1b3Rle1xcc2lnbWF9XFxudmFye1xcc2lnbWF9fXtcXGdxdW90ZXtcXHRhdX19Il0sWzIsMywiXFxubGFte1xcc2lnbWF9e1xcdGF1fSJdXQ==
    \[\begin{tikzcd}
    	{\gluepsh{\functy{\sigma}{\tau}}} & {(\cexp{\gluepsh{\sigma}}{\gluepsh{\tau}})} && {(\cexp{\FSub[\singlist{\sigma}]}{\PNf{\tau}})} & {\PNf{\functy{\sigma}{\tau}}}
    	\arrow["k", from=1-1, to=1-2]
    	\arrow["{\cexp{\gunquote{\sigma}\nvar{\sigma}}{\gquote{\tau}}}", from=1-2, to=1-4]
    	\arrow["{\nlam{\sigma}{\tau}}", from=1-4, to=1-5]
    \end{tikzcd}\] \qedhere
\end{items}
\end{enum}
\end{proof}
\end{prop}

\begin{comment}
\begin{prop}
For all types $\sigma \in \sTy$, the following diagram commutes:
% https://q.uiver.app/#q=WzAsNCxbMSwwLCJcXGdsdWVwc2h7XFxzaWdtYX0iXSxbMiwwLCJcXFBOZntcXHNpZ21hfSJdLFswLDAsIlxcUE5le1xcc2lnbWF9Il0sWzEsMSwiXFxGU3ViW1xcc2luZ2xpc3R7XFxzaWdtYX1dIl0sWzIsMCwiXFxndW5xdW90ZXtcXHNpZ21hfSJdLFswLDEsIlxcZ3F1b3Rle1xcc2lnbWF9Il0sWzEsMywiXFxwcmVkbm9ybXtcXHNpZ21hfSJdLFsyLDMsIlxccHJlZG5ldXR7XFxzaWdtYX0iLDJdLFswLDMsIlxcZ2x1ZXByZWR7XFxzaWdtYX0iLDFdXQ==
\[\begin{tikzcd}
	{\PNe{\sigma}} & {\gluepsh{\sigma}} & {\PNf{\sigma}} \\
	& {\FSub[\singlist{\sigma}]}
	\arrow["{\gunquote{\sigma}}", from=1-1, to=1-2]
	\arrow["{\gquote{\sigma}}", from=1-2, to=1-3]
	\arrow["{\prednorm{\sigma}}", from=1-3, to=2-2]
	\arrow["{\predneut{\sigma}}"', from=1-1, to=2-2]
	\arrow["{\gluepred{\sigma}}"{description}, from=1-2, to=2-2]
\end{tikzcd}\]
\begin{proof}
We follow the same construction as Fiore \cite[Figure 25]{fiore:2022:mscs}. Fiore constructs quote and unquote as maps in the gluing category, whereas we construct these maps in the presheaf category. However, since the second coordinate of the maps in the gluing category are identities \cite[Proposition 18]{fiore:2022:mscs}, we can reconstruct the desired maps in the gluing category from the maps that we constructed for presheaves.
\end{proof}
\end{prop}
\end{comment}

We now extend the unquote function to contexts. This is necessary to define the normalization function. Note that $\gluepsh{\sempcon} = \1$ and $\gluepsh{\sextcon{\Gamma}{\sigma}} = \cprod{\gluepsh{\Gamma}}{\gluepsh{\sigma}}$ by Corollary~\ref{cor:gluing-interpretation}.

\begin{defn} \label{def:gluing-unquote-context}
We define $\gunquote{\Gamma} : \PNe{\Gamma} \to \gluepsh{\Gamma}$ by recursion on the context $\Gamma$.
\begin{align*}
\gunquote{\sempcon}_\Delta(\sempsub[\Gamma])
    &= \singel \\
\gunquote{\sextcon{\Gamma}{\sigma}}_\Delta(\sextsub{\gamma}{t})
    &= (\gunquote{\Gamma}_\Delta(\gamma), \gunquote{\sigma}_\Delta(t))
\end{align*}
\end{defn}

\section{Normalization function}

In this section, we construct the normalization function and prove its correctness.

Recall from Corollary~\ref{cor:gluing-interpretation} that the interpretation $\cint[\glueint]{t}[\gluecat]$ of a term $\stypedtm{t}{\Gamma}{\sigma}$ in the category $\gluecat$ takes the form $(\gluemortm{t}, [\stmsub{t}])$ for some $\gluemortm{t} : \gluepsh{\Gamma} \to \gluepsh{\sigma}$.

\begin{defn} \label{def:gluing-norm-fun}
We define the normalization function $\gnorm{\Gamma}{\sigma} : \sTm{\Gamma}{\sigma} \to \sNf{\Gamma}{\sigma}$ as
\[ \gnorm{\Gamma}{\sigma}(t) = \gquote{\sigma}_\Gamma(
    \gluemortm{t}_\Gamma(\gunquote{\Gamma}_\Gamma(\sidsub[\Gamma]))). \]
\end{defn}

In essence, the definition of the normalization function has the same form as the one for normalization by evaluation in Section~\ref{sec:nbe-alg}.

Diagrammatically, the normalization function is the composite
% https://q.uiver.app/#q=WzAsNSxbMCwwLCJcXHNUbXtcXEdhbW1hfXtcXHNpZ21hfSJdLFsxLDAsIlxcSG9tW1xcZ2x1ZWNhdF17XFxjaW50W1xcZ2x1ZWludF17XFxHYW1tYX19e1xcY2ludFtcXGdsdWVpbnRde1xcc2lnbWF9fSJdLFsyLDAsIlxcSG9tW1xcUFNoe1xccENvbn1de1xcZ2x1ZXBzaHtcXEdhbW1hfX17XFxnbHVlcHNoe1xcc2lnbWF9fSJdLFszLDAsIlxcZ2x1ZXBzaHtcXHNpZ21hfShcXEdhbW1hKSJdLFs0LDAsIlxcc05me1xcR2FtbWF9e1xcc2lnbWF9Il0sWzAsMSwiXFxjaW50dG1bXFxnbHVlaW50XXstfSJdLFsxLDIsIlxcZ2x1ZWZzdCJdLFszLDQsIlxcZ3F1b3Rle1xcc2lnbWF9X1xcR2FtbWEiXSxbMiwzLCJcXHBoaSJdXQ==
\[\begin{tikzcd}
	{\sTm{\Gamma}{\sigma}} & {\Hom[\gluecat]{\cint[\glueint]{\Gamma}}{\cint[\glueint]{\sigma}}} & {\Hom[\PSh{\pCon}]{\gluepsh{\Gamma}}{\gluepsh{\sigma}}} & {\gluepsh{\sigma}(\Gamma)} & {\sNf{\Gamma}{\sigma}}
	\arrow["{\cinttm[\glueint]{-}}", from=1-1, to=1-2]
	\arrow["\gluefst", from=1-2, to=1-3]
	\arrow["{\gquote{\sigma}_\Gamma}", from=1-4, to=1-5]
	\arrow["\phi", from=1-3, to=1-4]
\end{tikzcd}\]
where $\phi$ is given by
\[ \phi(\tilde{t}) = \tilde{t}_\Gamma(\gunquote{\Gamma}_\Gamma(\sidsub[\Gamma])). \]

The correctness of the normalization function can be expressed via the following two propositions.

\begin{prop}
If $\sconvtm{t}{u}{\Gamma}{\sigma}$, then $\gnorm{\Gamma}{\sigma}(t) = \gnorm{\Gamma}{\sigma}(u)$.
\begin{proof}
Note that $\gnorm{\Gamma}{\sigma}(t)$ is defined in terms of the interpretation $\cint[\glueint]{t}[\gluecat]$. Hence, by the soundness of the interpretation (Theorem~\ref{thm:cat-soundness}), convertible terms have equal normal forms.
\end{proof}
\end{prop}

\begin{prop}
$\sconvtm{\gnorm{\Gamma}{\sigma}(t)}{t}{\Gamma}{\sigma}$.
\begin{proof}
This is Theorem 3.6 in \cite{sterling:2018:arxiv}.
\end{proof}
\end{prop}

\begin{comment}
\section{Using the internal language}

We can simplify the construction of the normalization function using the internal language of the presheaf topos $\PSh{\rencat}$.
\end{comment}

\begin{comment}
Further applications

Gluing abstraction includes:
\begin{items}
    \item Classical (unary) logical predicates (source is classifying category, target is Set);
    e.g. canonicity (gluing along global sections functor), existence/disjunction property
    \item $n$-ary ogical relations (source is product of classifying categories, target is Set);
    e.g. parametricity/free theorems?
    \item Kripke logical relations (of varying arity) (target is presheaves over a poset/category);
    e.g. normalization, definability, completeness
    \item Beth/Grothendieck logical relations (target is sheaves over poset/category);
    + sum types
\end{items}
Most useful if target is a locally cartesian closed category
\end{comment}


\chapter{Conclusion}
\begin{comment}
\begin{itemize}
    \item Give a short summary of what I did:
    \begin{itemize}
        \item I presented and compared 3 normalization proofs: Tait, NbE, gluing
        \item On the way introduction to the model theory lambda calculus: Henkin models, Kripke models, simply typed categories with families
    \end{itemize}
    \item Future work
    \begin{itemize}
        \item generalize NbE and gluing in the same way I generalized Tait
        \item Consider more complex type theories (e.g. dependent types, inductive types)
    \end{itemize}
\end{itemize}
\end{comment}

We presented and compared three reduction-free normalization proofs for the simply typed $\lambda$-calculus, all of which share a similar structure. The proofs have advantages and disadvantages.
\begin{enum}
    \item First, we proved weak normalization using Tait's idea of a convertibility predicate (Section~\ref{sec:Tait-proof}). An advantage of this proof is that it is rather short and easy to understand. A disadvantage is that the weak normalization theorem does not tell us how to compute the normal form of a term.
    \item The second proof constructed a normalization function which computes the normal form of a given input term (Section~\ref{sec:nbe-alg}). This proof used the technique of normalization by evaluation, whereby normalization is implemented by interpreting the term in a suitable model and then turning the semantic object back into a normal form. An advantage of this proof is that it provides an efficient normalization algorithm which does not rely on concepts from term rewriting. A disadvantage is that the proof itself is more involved.
    \item The third proof was a categorical reconstruction of normalization by evaluation using the categorical gluing construction (Chapter~\ref{chap:gluing}). The main advantage of this proof is that it is independent of the syntactic presentation of $\lambda$-calculus: it employs categorical semantics where the syntax can be characterized as an initial model. A disadvantage is that the constructions in this proof are complicated and abstract, and thus this proof is much harder to understand.
\end{enum}

On the way to presenting the proofs, we also provided introductions to various flavors of syntax and semantics for the simply typed $\lambda$-calculus. In particular, we discussed Henkin models, Kripke models, simply typed categories with families, and cartesian closed categories.

As an attempt to generalize the structure of the Tait-like normalization proof, we introduced the notion of \textit{bilogical predicates}. Using this notion, we were able to give a very short proof of weak normalization for the simply typed $\lambda$-calculus. As future work, one could consider a similar generalization for normalization by evaluation.

Another, orthogonal direction of research is to consider more complex type theories, such as dependent type theory or type theory with simple inductive types. A comparison between the different proof methods for such systems would be interesting.


% You can choose a citation style, 'plain' is the default
% See:
% https://www.overleaf.com/learn/latex/Bibtex_bibliography_styles

\bibliographystyle{plain}
\bibliography{refs}

\end{document}

% Have fun!
% -fons

% http://www2.washjeff.edu/users/rhigginbottom/latex/resources/symbols.pdf